% Generated by Sphinx.
\def\sphinxdocclass{report}
\documentclass[letterpaper,10pt,english]{sphinxmanual}
\usepackage[utf8]{inputenc}
\DeclareUnicodeCharacter{00A0}{\nobreakspace}
\usepackage{cmap}
\usepackage[T1]{fontenc}
\usepackage{babel}
\usepackage{times}
\usepackage[Sonny]{fncychap}
\usepackage{longtable}
\usepackage{sphinx}
\usepackage{multirow}

\addto\captionsenglish{\renewcommand{\figurename}{Fig. }}
\addto\captionsenglish{\renewcommand{\tablename}{Table }}
\floatname{literal-block}{Listing }



\title{ESCAPEv2 Documentation}
\date{June 29, 2015}
\release{2.0.0}
\author{János Czentye}
\newcommand{\sphinxlogo}{}
\renewcommand{\releasename}{Release}
\makeindex

\makeatletter
\def\PYG@reset{\let\PYG@it=\relax \let\PYG@bf=\relax%
    \let\PYG@ul=\relax \let\PYG@tc=\relax%
    \let\PYG@bc=\relax \let\PYG@ff=\relax}
\def\PYG@tok#1{\csname PYG@tok@#1\endcsname}
\def\PYG@toks#1+{\ifx\relax#1\empty\else%
    \PYG@tok{#1}\expandafter\PYG@toks\fi}
\def\PYG@do#1{\PYG@bc{\PYG@tc{\PYG@ul{%
    \PYG@it{\PYG@bf{\PYG@ff{#1}}}}}}}
\def\PYG#1#2{\PYG@reset\PYG@toks#1+\relax+\PYG@do{#2}}

\expandafter\def\csname PYG@tok@gd\endcsname{\def\PYG@tc##1{\textcolor[rgb]{0.63,0.00,0.00}{##1}}}
\expandafter\def\csname PYG@tok@gu\endcsname{\let\PYG@bf=\textbf\def\PYG@tc##1{\textcolor[rgb]{0.50,0.00,0.50}{##1}}}
\expandafter\def\csname PYG@tok@gt\endcsname{\def\PYG@tc##1{\textcolor[rgb]{0.00,0.27,0.87}{##1}}}
\expandafter\def\csname PYG@tok@gs\endcsname{\let\PYG@bf=\textbf}
\expandafter\def\csname PYG@tok@gr\endcsname{\def\PYG@tc##1{\textcolor[rgb]{1.00,0.00,0.00}{##1}}}
\expandafter\def\csname PYG@tok@cm\endcsname{\let\PYG@it=\textit\def\PYG@tc##1{\textcolor[rgb]{0.25,0.50,0.56}{##1}}}
\expandafter\def\csname PYG@tok@vg\endcsname{\def\PYG@tc##1{\textcolor[rgb]{0.73,0.38,0.84}{##1}}}
\expandafter\def\csname PYG@tok@m\endcsname{\def\PYG@tc##1{\textcolor[rgb]{0.13,0.50,0.31}{##1}}}
\expandafter\def\csname PYG@tok@mh\endcsname{\def\PYG@tc##1{\textcolor[rgb]{0.13,0.50,0.31}{##1}}}
\expandafter\def\csname PYG@tok@cs\endcsname{\def\PYG@tc##1{\textcolor[rgb]{0.25,0.50,0.56}{##1}}\def\PYG@bc##1{\setlength{\fboxsep}{0pt}\colorbox[rgb]{1.00,0.94,0.94}{\strut ##1}}}
\expandafter\def\csname PYG@tok@ge\endcsname{\let\PYG@it=\textit}
\expandafter\def\csname PYG@tok@vc\endcsname{\def\PYG@tc##1{\textcolor[rgb]{0.73,0.38,0.84}{##1}}}
\expandafter\def\csname PYG@tok@il\endcsname{\def\PYG@tc##1{\textcolor[rgb]{0.13,0.50,0.31}{##1}}}
\expandafter\def\csname PYG@tok@go\endcsname{\def\PYG@tc##1{\textcolor[rgb]{0.20,0.20,0.20}{##1}}}
\expandafter\def\csname PYG@tok@cp\endcsname{\def\PYG@tc##1{\textcolor[rgb]{0.00,0.44,0.13}{##1}}}
\expandafter\def\csname PYG@tok@gi\endcsname{\def\PYG@tc##1{\textcolor[rgb]{0.00,0.63,0.00}{##1}}}
\expandafter\def\csname PYG@tok@gh\endcsname{\let\PYG@bf=\textbf\def\PYG@tc##1{\textcolor[rgb]{0.00,0.00,0.50}{##1}}}
\expandafter\def\csname PYG@tok@ni\endcsname{\let\PYG@bf=\textbf\def\PYG@tc##1{\textcolor[rgb]{0.84,0.33,0.22}{##1}}}
\expandafter\def\csname PYG@tok@nl\endcsname{\let\PYG@bf=\textbf\def\PYG@tc##1{\textcolor[rgb]{0.00,0.13,0.44}{##1}}}
\expandafter\def\csname PYG@tok@nn\endcsname{\let\PYG@bf=\textbf\def\PYG@tc##1{\textcolor[rgb]{0.05,0.52,0.71}{##1}}}
\expandafter\def\csname PYG@tok@no\endcsname{\def\PYG@tc##1{\textcolor[rgb]{0.38,0.68,0.84}{##1}}}
\expandafter\def\csname PYG@tok@na\endcsname{\def\PYG@tc##1{\textcolor[rgb]{0.25,0.44,0.63}{##1}}}
\expandafter\def\csname PYG@tok@nb\endcsname{\def\PYG@tc##1{\textcolor[rgb]{0.00,0.44,0.13}{##1}}}
\expandafter\def\csname PYG@tok@nc\endcsname{\let\PYG@bf=\textbf\def\PYG@tc##1{\textcolor[rgb]{0.05,0.52,0.71}{##1}}}
\expandafter\def\csname PYG@tok@nd\endcsname{\let\PYG@bf=\textbf\def\PYG@tc##1{\textcolor[rgb]{0.33,0.33,0.33}{##1}}}
\expandafter\def\csname PYG@tok@ne\endcsname{\def\PYG@tc##1{\textcolor[rgb]{0.00,0.44,0.13}{##1}}}
\expandafter\def\csname PYG@tok@nf\endcsname{\def\PYG@tc##1{\textcolor[rgb]{0.02,0.16,0.49}{##1}}}
\expandafter\def\csname PYG@tok@si\endcsname{\let\PYG@it=\textit\def\PYG@tc##1{\textcolor[rgb]{0.44,0.63,0.82}{##1}}}
\expandafter\def\csname PYG@tok@s2\endcsname{\def\PYG@tc##1{\textcolor[rgb]{0.25,0.44,0.63}{##1}}}
\expandafter\def\csname PYG@tok@vi\endcsname{\def\PYG@tc##1{\textcolor[rgb]{0.73,0.38,0.84}{##1}}}
\expandafter\def\csname PYG@tok@nt\endcsname{\let\PYG@bf=\textbf\def\PYG@tc##1{\textcolor[rgb]{0.02,0.16,0.45}{##1}}}
\expandafter\def\csname PYG@tok@nv\endcsname{\def\PYG@tc##1{\textcolor[rgb]{0.73,0.38,0.84}{##1}}}
\expandafter\def\csname PYG@tok@s1\endcsname{\def\PYG@tc##1{\textcolor[rgb]{0.25,0.44,0.63}{##1}}}
\expandafter\def\csname PYG@tok@gp\endcsname{\let\PYG@bf=\textbf\def\PYG@tc##1{\textcolor[rgb]{0.78,0.36,0.04}{##1}}}
\expandafter\def\csname PYG@tok@sh\endcsname{\def\PYG@tc##1{\textcolor[rgb]{0.25,0.44,0.63}{##1}}}
\expandafter\def\csname PYG@tok@ow\endcsname{\let\PYG@bf=\textbf\def\PYG@tc##1{\textcolor[rgb]{0.00,0.44,0.13}{##1}}}
\expandafter\def\csname PYG@tok@sx\endcsname{\def\PYG@tc##1{\textcolor[rgb]{0.78,0.36,0.04}{##1}}}
\expandafter\def\csname PYG@tok@bp\endcsname{\def\PYG@tc##1{\textcolor[rgb]{0.00,0.44,0.13}{##1}}}
\expandafter\def\csname PYG@tok@c1\endcsname{\let\PYG@it=\textit\def\PYG@tc##1{\textcolor[rgb]{0.25,0.50,0.56}{##1}}}
\expandafter\def\csname PYG@tok@kc\endcsname{\let\PYG@bf=\textbf\def\PYG@tc##1{\textcolor[rgb]{0.00,0.44,0.13}{##1}}}
\expandafter\def\csname PYG@tok@c\endcsname{\let\PYG@it=\textit\def\PYG@tc##1{\textcolor[rgb]{0.25,0.50,0.56}{##1}}}
\expandafter\def\csname PYG@tok@mf\endcsname{\def\PYG@tc##1{\textcolor[rgb]{0.13,0.50,0.31}{##1}}}
\expandafter\def\csname PYG@tok@err\endcsname{\def\PYG@bc##1{\setlength{\fboxsep}{0pt}\fcolorbox[rgb]{1.00,0.00,0.00}{1,1,1}{\strut ##1}}}
\expandafter\def\csname PYG@tok@mb\endcsname{\def\PYG@tc##1{\textcolor[rgb]{0.13,0.50,0.31}{##1}}}
\expandafter\def\csname PYG@tok@ss\endcsname{\def\PYG@tc##1{\textcolor[rgb]{0.32,0.47,0.09}{##1}}}
\expandafter\def\csname PYG@tok@sr\endcsname{\def\PYG@tc##1{\textcolor[rgb]{0.14,0.33,0.53}{##1}}}
\expandafter\def\csname PYG@tok@mo\endcsname{\def\PYG@tc##1{\textcolor[rgb]{0.13,0.50,0.31}{##1}}}
\expandafter\def\csname PYG@tok@kd\endcsname{\let\PYG@bf=\textbf\def\PYG@tc##1{\textcolor[rgb]{0.00,0.44,0.13}{##1}}}
\expandafter\def\csname PYG@tok@mi\endcsname{\def\PYG@tc##1{\textcolor[rgb]{0.13,0.50,0.31}{##1}}}
\expandafter\def\csname PYG@tok@kn\endcsname{\let\PYG@bf=\textbf\def\PYG@tc##1{\textcolor[rgb]{0.00,0.44,0.13}{##1}}}
\expandafter\def\csname PYG@tok@o\endcsname{\def\PYG@tc##1{\textcolor[rgb]{0.40,0.40,0.40}{##1}}}
\expandafter\def\csname PYG@tok@kr\endcsname{\let\PYG@bf=\textbf\def\PYG@tc##1{\textcolor[rgb]{0.00,0.44,0.13}{##1}}}
\expandafter\def\csname PYG@tok@s\endcsname{\def\PYG@tc##1{\textcolor[rgb]{0.25,0.44,0.63}{##1}}}
\expandafter\def\csname PYG@tok@kp\endcsname{\def\PYG@tc##1{\textcolor[rgb]{0.00,0.44,0.13}{##1}}}
\expandafter\def\csname PYG@tok@w\endcsname{\def\PYG@tc##1{\textcolor[rgb]{0.73,0.73,0.73}{##1}}}
\expandafter\def\csname PYG@tok@kt\endcsname{\def\PYG@tc##1{\textcolor[rgb]{0.56,0.13,0.00}{##1}}}
\expandafter\def\csname PYG@tok@sc\endcsname{\def\PYG@tc##1{\textcolor[rgb]{0.25,0.44,0.63}{##1}}}
\expandafter\def\csname PYG@tok@sb\endcsname{\def\PYG@tc##1{\textcolor[rgb]{0.25,0.44,0.63}{##1}}}
\expandafter\def\csname PYG@tok@k\endcsname{\let\PYG@bf=\textbf\def\PYG@tc##1{\textcolor[rgb]{0.00,0.44,0.13}{##1}}}
\expandafter\def\csname PYG@tok@se\endcsname{\let\PYG@bf=\textbf\def\PYG@tc##1{\textcolor[rgb]{0.25,0.44,0.63}{##1}}}
\expandafter\def\csname PYG@tok@sd\endcsname{\let\PYG@it=\textit\def\PYG@tc##1{\textcolor[rgb]{0.25,0.44,0.63}{##1}}}

\def\PYGZbs{\char`\\}
\def\PYGZus{\char`\_}
\def\PYGZob{\char`\{}
\def\PYGZcb{\char`\}}
\def\PYGZca{\char`\^}
\def\PYGZam{\char`\&}
\def\PYGZlt{\char`\<}
\def\PYGZgt{\char`\>}
\def\PYGZsh{\char`\#}
\def\PYGZpc{\char`\%}
\def\PYGZdl{\char`\$}
\def\PYGZhy{\char`\-}
\def\PYGZsq{\char`\'}
\def\PYGZdq{\char`\"}
\def\PYGZti{\char`\~}
% for compatibility with earlier versions
\def\PYGZat{@}
\def\PYGZlb{[}
\def\PYGZrb{]}
\makeatother

\renewcommand\PYGZsq{\textquotesingle}

\begin{document}

\maketitle
\tableofcontents
\phantomsection\label{index::doc}


Welcome! This is the API documentation for \textbf{ESCAPEv2}.


\chapter{Overview}
\label{index:welcome-to-escapev2-s-documentation}\label{index:overview}
\href{http://mininet.org/}{Mininet} is a great prototyping tool which takes
existing SDN-related software components (e.g. Open vSwitch, OpenFlow
controllers, network namespaces, cgroups, etc.) and combines them into a
framework, which can automatically set up and configure customized OpenFlow
testbeds scaling up to hundreds of nodes. Standing on the shoulders of Mininet,
we have implemented a similar prototyping system called ESCAPE, which can be
used to develop and test various components of the service chaining
architecture. Our framework incorporates
\href{http://www.read.cs.ucla.edu/click/}{Click} for implementing Virtual Network
Functions (VNF), NETCONF (\index{RFC!RFC 6241}\href{https://tools.ietf.org/html/rfc6241.html}{\textbf{RFC 6241}}) for managing Click-based VNFs and
\href{https://openflow.stanford.edu/display/ONL/POX+Wiki}{POX} for taking care of
traffic steering. We also add our extensible Orchestrator module, which can
accommodate mapping algorithms from abstract service descriptions to deployed
and running service chains.


\strong{See also:}


The source code of previous ESCAPE version is available at our \href{https://github.com/nemethf/escape}{github page}. For more information we first suggest
to read our paper:

Attila Csoma, Balazs Sonkoly, Levente Csikor, Felician Nemeth, Andras Gulyas,
Wouter Tavernier, and Sahel Sahhaf: \textbf{ESCAPE: Extensible Service ChAin
Prototyping Environment using Mininet, Click, NETCONF and POX}.
Demo paper at Sigcomm`14.
\begin{itemize}
\item {} 
\href{http://dl.acm.org/authorize?N71297}{Download the paper}

\item {} 
\href{http://sb.tmit.bme.hu/mediawiki/images/b/ba/Sigcomm2014\_poster.png}{Accompanying poster}

\end{itemize}

For further information contact \href{mailto:csoma@tmit.bme.hu}{csoma@tmit.bme.hu}, \href{mailto:sonkoly@tmit.bme.hu}{sonkoly@tmit.bme.hu}




\chapter{ESCAPEv2 structure}
\label{index:escapev2-structure}

\section{Dependencies:}
\label{index:dependencies}
\begin{Verbatim}[commandchars=\\\{\}]
\PYG{n+nv}{\PYGZdl{} }sudo apt\PYGZhy{}get install libxml2 libxslt1\PYGZhy{}dev python\PYGZhy{}setuptools python\PYGZhy{}pip \PYG{l+s+se}{\PYGZbs{}}
python\PYGZhy{}paramiko python\PYGZhy{}lxml python\PYGZhy{}libxml2 python\PYGZhy{}libxslt1

sudo pip install networkx ncclient requests
\end{Verbatim}


\section{Class structure}
\label{index:class-structure}

\subsection{The \emph{escape} package}
\label{top:the-escape-package}\label{top::doc}

\subsubsection{\emph{escape} package}
\label{escape:escape-package}\label{escape::doc}\label{escape:module-escape}\index{escape (module)}
Unifying package for ESCAPEv2 functions

\emph{CONFIG} contains the ESCAPEv2 dependent configuration such as the concrete
RequestHandler and strategy classes, the initial Adapter classes, etc.


\paragraph{Submodules}
\label{escape:submodules}

\subparagraph{\emph{escape.service} package}
\label{service/service:escape-service-package}\label{service/service::doc}\label{service/service:module-escape.service}\index{escape.service (module)}
Subpackage for classes related mostly to Service (Graph) Adaptation sublayer


\subparagraph{Submodules}
\label{service/service:submodules}

\subparagraph{\emph{element\_mgmt.py} module}
\label{service/element_mgmt:element-mgmt-py-module}\label{service/element_mgmt::doc}
{\hyperref[service/element_mgmt:escape.service.element_mgmt.AbstractElementManager]{\emph{\code{AbstractElementManager}}}} is an abstract class for element managers

{\hyperref[service/element_mgmt:escape.service.element_mgmt.ClickManager]{\emph{\code{ClickManager}}}} represent the interface to Click elements


\subparagraph{Module contents}
\label{service/element_mgmt:module-escape.service.element_mgmt}\label{service/element_mgmt:module-contents}\index{escape.service.element\_mgmt (module)}
Contains classes relevant to element management
\index{AbstractElementManager (class in escape.service.element\_mgmt)}

\begin{fulllineitems}
\phantomsection\label{service/element_mgmt:escape.service.element_mgmt.AbstractElementManager}\pysigline{\strong{class }\code{escape.service.element\_mgmt.}\bfcode{AbstractElementManager}}
Bases: \href{https://docs.python.org/2.7/library/functions.html\#object}{\code{object}}

Abstract class for element management components (EM)

\begin{notice}{warning}{Warning:}
Not implemented yet!
\end{notice}
\index{\_\_init\_\_() (escape.service.element\_mgmt.AbstractElementManager method)}

\begin{fulllineitems}
\phantomsection\label{service/element_mgmt:escape.service.element_mgmt.AbstractElementManager.__init__}\pysiglinewithargsret{\bfcode{\_\_init\_\_}}{}{}
Init

\end{fulllineitems}


\end{fulllineitems}

\index{ClickManager (class in escape.service.element\_mgmt)}

\begin{fulllineitems}
\phantomsection\label{service/element_mgmt:escape.service.element_mgmt.ClickManager}\pysigline{\strong{class }\code{escape.service.element\_mgmt.}\bfcode{ClickManager}}
Bases: {\hyperref[service/element_mgmt:escape.service.element_mgmt.AbstractElementManager]{\emph{\code{escape.service.element\_mgmt.AbstractElementManager}}}}

Manager class for specific VNF management based on Clicky

\begin{notice}{warning}{Warning:}
Not implemented yet!
\end{notice}
\index{\_\_init\_\_() (escape.service.element\_mgmt.ClickManager method)}

\begin{fulllineitems}
\phantomsection\label{service/element_mgmt:escape.service.element_mgmt.ClickManager.__init__}\pysiglinewithargsret{\bfcode{\_\_init\_\_}}{}{}
Init

\end{fulllineitems}


\end{fulllineitems}



\subparagraph{\emph{sas\_mapping.py} module}
\label{service/sas_mapping:sas-mapping-py-module}\label{service/sas_mapping::doc}
{\hyperref[service/sas_mapping:escape.service.sas_mapping.DefaultServiceMappingStrategy]{\emph{\code{DefaultServiceMappingStrategy}}}} implements a default mapping algorithm
which map given SG on a single Bis-Bis

{\hyperref[service/sas_mapping:escape.service.sas_mapping.ServiceGraphMapper]{\emph{\code{ServiceGraphMapper}}}} perform the supplementary tasks for SG mapping


\subparagraph{Module contents}
\label{service/sas_mapping:module-contents}\label{service/sas_mapping:module-escape.service.sas_mapping}\index{escape.service.sas\_mapping (module)}
Contains classes which implement SG mapping functionality
\index{DefaultServiceMappingStrategy (class in escape.service.sas\_mapping)}

\begin{fulllineitems}
\phantomsection\label{service/sas_mapping:escape.service.sas_mapping.DefaultServiceMappingStrategy}\pysigline{\strong{class }\code{escape.service.sas\_mapping.}\bfcode{DefaultServiceMappingStrategy}}
Bases: {\hyperref[util/mapping:escape.util.mapping.AbstractMappingStrategy]{\emph{\code{escape.util.mapping.AbstractMappingStrategy}}}}

Mapping class which maps given Service Graph into a single BiS-BiS
\index{\_\_init\_\_() (escape.service.sas\_mapping.DefaultServiceMappingStrategy method)}

\begin{fulllineitems}
\phantomsection\label{service/sas_mapping:escape.service.sas_mapping.DefaultServiceMappingStrategy.__init__}\pysiglinewithargsret{\bfcode{\_\_init\_\_}}{}{}
Init

\end{fulllineitems}

\index{map() (escape.service.sas\_mapping.DefaultServiceMappingStrategy class method)}

\begin{fulllineitems}
\phantomsection\label{service/sas_mapping:escape.service.sas_mapping.DefaultServiceMappingStrategy.map}\pysiglinewithargsret{\strong{classmethod }\bfcode{map}}{\emph{graph}, \emph{resource}}{}
Default mapping algorithm which maps given Service Graph on one BiS-BiS
\begin{quote}\begin{description}
\item[{Parameters}] \leavevmode\begin{itemize}
\item {} 
\textbf{\texttt{graph}} ({\hyperref[util/nffg:escape.util.nffg.NFFG]{\emph{\emph{NFFG}}}}) -- Service Graph

\item {} 
\textbf{\texttt{resource}} ({\hyperref[util/nffg:escape.util.nffg.NFFG]{\emph{\emph{NFFG}}}}) -- virtual resource

\end{itemize}

\item[{Returns}] \leavevmode
Network Function Forwarding Graph

\item[{Return type}] \leavevmode
{\hyperref[util/nffg:escape.util.nffg.NFFG]{\emph{NFFG}}}

\end{description}\end{quote}

\end{fulllineitems}


\end{fulllineitems}

\index{SGMappingFinishedEvent (class in escape.service.sas\_mapping)}

\begin{fulllineitems}
\phantomsection\label{service/sas_mapping:escape.service.sas_mapping.SGMappingFinishedEvent}\pysiglinewithargsret{\strong{class }\code{escape.service.sas\_mapping.}\bfcode{SGMappingFinishedEvent}}{\emph{nffg}}{}
Bases: \code{pox.lib.revent.revent.Event}

Event for signaling the end of SG mapping
\index{\_\_init\_\_() (escape.service.sas\_mapping.SGMappingFinishedEvent method)}

\begin{fulllineitems}
\phantomsection\label{service/sas_mapping:escape.service.sas_mapping.SGMappingFinishedEvent.__init__}\pysiglinewithargsret{\bfcode{\_\_init\_\_}}{\emph{nffg}}{}
Init
\begin{quote}\begin{description}
\item[{Parameters}] \leavevmode
\textbf{\texttt{nffg}} ({\hyperref[util/nffg:escape.util.nffg.NFFG]{\emph{\emph{NFFG}}}}) -- NF-FG need to be initiated

\end{description}\end{quote}

\end{fulllineitems}


\end{fulllineitems}

\index{ServiceGraphMapper (class in escape.service.sas\_mapping)}

\begin{fulllineitems}
\phantomsection\label{service/sas_mapping:escape.service.sas_mapping.ServiceGraphMapper}\pysigline{\strong{class }\code{escape.service.sas\_mapping.}\bfcode{ServiceGraphMapper}}
Bases: {\hyperref[util/mapping:escape.util.mapping.AbstractMapper]{\emph{\code{escape.util.mapping.AbstractMapper}}}}

Helper class for mapping Service Graph to NF-FG
\index{\_eventMixin\_events (escape.service.sas\_mapping.ServiceGraphMapper attribute)}

\begin{fulllineitems}
\phantomsection\label{service/sas_mapping:escape.service.sas_mapping.ServiceGraphMapper._eventMixin_events}\pysigline{\bfcode{\_eventMixin\_events}\strong{ = set({[}\textless{}class `escape.service.sas\_mapping.SGMappingFinishedEvent'\textgreater{}{]})}}
\end{fulllineitems}

\index{\_\_init\_\_() (escape.service.sas\_mapping.ServiceGraphMapper method)}

\begin{fulllineitems}
\phantomsection\label{service/sas_mapping:escape.service.sas_mapping.ServiceGraphMapper.__init__}\pysiglinewithargsret{\bfcode{\_\_init\_\_}}{}{}
Init Service mapper
\begin{quote}\begin{description}
\item[{Returns}] \leavevmode
None

\end{description}\end{quote}

\end{fulllineitems}

\index{orchestrate() (escape.service.sas\_mapping.ServiceGraphMapper method)}

\begin{fulllineitems}
\phantomsection\label{service/sas_mapping:escape.service.sas_mapping.ServiceGraphMapper.orchestrate}\pysiglinewithargsret{\bfcode{orchestrate}}{\emph{input\_graph}, \emph{resource\_view}}{}
Orchestrate mapping of given service graph on given virtual resource
\begin{quote}\begin{description}
\item[{Parameters}] \leavevmode\begin{itemize}
\item {} 
\textbf{\texttt{input\_graph}} ({\hyperref[util/nffg:escape.util.nffg.NFFG]{\emph{\emph{NFFG}}}}) -- Service Graph

\item {} 
\textbf{\texttt{resource\_view}} -- virtual resource view

\item {} 
\textbf{\texttt{resource\_view}} -- ESCAPEVirtualizer

\end{itemize}

\item[{Returns}] \leavevmode
Network Function Forwarding Graph

\item[{Return type}] \leavevmode
{\hyperref[util/nffg:escape.util.nffg.NFFG]{\emph{NFFG}}}

\end{description}\end{quote}

\end{fulllineitems}

\index{\_mapping\_finished() (escape.service.sas\_mapping.ServiceGraphMapper method)}

\begin{fulllineitems}
\phantomsection\label{service/sas_mapping:escape.service.sas_mapping.ServiceGraphMapper._mapping_finished}\pysiglinewithargsret{\bfcode{\_mapping\_finished}}{\emph{nffg}}{}
Called from a separate thread when the mapping process is finished
\begin{quote}\begin{description}
\item[{Parameters}] \leavevmode
\textbf{\texttt{nffg}} ({\hyperref[util/nffg:escape.util.nffg.NFFG]{\emph{\emph{NFFG}}}}) -- geenrated NF-FG

\item[{Returns}] \leavevmode
None

\end{description}\end{quote}

\end{fulllineitems}


\end{fulllineitems}



\subparagraph{\emph{sas\_API.py} module}
\label{service/sas_API:sas-api-py-module}\label{service/sas_API::doc}
{\hyperref[service/sas_API:escape.service.sas_API.InstantiateNFFGEvent]{\emph{\code{InstantiateNFFGEvent}}}} can send NF-FG to the lower layer

{\hyperref[service/sas_API:escape.service.sas_API.GetVirtResInfoEvent]{\emph{\code{GetVirtResInfoEvent}}}} can request virtual resource info from lower layer

{\hyperref[service/sas_API:escape.service.sas_API.ServiceRequestHandler]{\emph{\code{ServiceRequestHandler}}}} implement the specific RESTful API functionality
thereby realizes the UNIFY's U - Sl API

{\hyperref[service/sas_API:escape.service.sas_API.ServiceLayerAPI]{\emph{\code{ServiceLayerAPI}}}} represents the SAS layer and implement all related
functionality


\subparagraph{Module contents}
\label{service/sas_API:module-contents}\label{service/sas_API:module-escape.service.sas_API}\index{escape.service.sas\_API (module)}
Implements the platform and POX dependent logic for the Service Adaptation
Sublayer
\index{InstantiateNFFGEvent (class in escape.service.sas\_API)}

\begin{fulllineitems}
\phantomsection\label{service/sas_API:escape.service.sas_API.InstantiateNFFGEvent}\pysiglinewithargsret{\strong{class }\code{escape.service.sas\_API.}\bfcode{InstantiateNFFGEvent}}{\emph{nffg}}{}
Bases: \code{pox.lib.revent.revent.Event}

Event for passing NFFG (mapped SG) to Orchestration layer
\index{\_\_init\_\_() (escape.service.sas\_API.InstantiateNFFGEvent method)}

\begin{fulllineitems}
\phantomsection\label{service/sas_API:escape.service.sas_API.InstantiateNFFGEvent.__init__}\pysiglinewithargsret{\bfcode{\_\_init\_\_}}{\emph{nffg}}{}
Init
\begin{quote}\begin{description}
\item[{Parameters}] \leavevmode
\textbf{\texttt{nffg}} ({\hyperref[util/nffg:escape.util.nffg.NFFG]{\emph{\emph{NFFG}}}}) -- NF-FG need to be initiated

\end{description}\end{quote}

\end{fulllineitems}


\end{fulllineitems}

\index{GetVirtResInfoEvent (class in escape.service.sas\_API)}

\begin{fulllineitems}
\phantomsection\label{service/sas_API:escape.service.sas_API.GetVirtResInfoEvent}\pysiglinewithargsret{\strong{class }\code{escape.service.sas\_API.}\bfcode{GetVirtResInfoEvent}}{\emph{sid}}{}
Bases: \code{pox.lib.revent.revent.Event}

Event for requesting virtual resource info from Orchestration layer
\index{\_\_init\_\_() (escape.service.sas\_API.GetVirtResInfoEvent method)}

\begin{fulllineitems}
\phantomsection\label{service/sas_API:escape.service.sas_API.GetVirtResInfoEvent.__init__}\pysiglinewithargsret{\bfcode{\_\_init\_\_}}{\emph{sid}}{}
Init
\begin{quote}\begin{description}
\item[{Parameters}] \leavevmode
\textbf{\texttt{sid}} (\href{https://docs.python.org/2.7/library/functions.html\#int}{\emph{int}}) -- Service layer ID

\end{description}\end{quote}

\end{fulllineitems}


\end{fulllineitems}

\index{ServiceRequestHandler (class in escape.service.sas\_API)}

\begin{fulllineitems}
\phantomsection\label{service/sas_API:escape.service.sas_API.ServiceRequestHandler}\pysiglinewithargsret{\strong{class }\code{escape.service.sas\_API.}\bfcode{ServiceRequestHandler}}{\emph{request}, \emph{client\_address}, \emph{server}}{}
Bases: {\hyperref[util/api:escape.util.api.AbstractRequestHandler]{\emph{\code{escape.util.api.AbstractRequestHandler}}}}

Request Handler for Service Adaptation SubLayer

\begin{notice}{warning}{Warning:}
This class is out of the context of the recoco's co-operative thread
context! While you don't need to worry much about synchronization between
recoco tasks, you do need to think about synchronization between recoco task
and normal threads. Synchronisation is needed to take care manually: use
relevant helper function of core object: \emph{callLater}/\emph{raiseLater} or use
\emph{schedule\_as\_coop\_task} decorator defined in util.misc on the called
function
\end{notice}
\index{request\_perm (escape.service.sas\_API.ServiceRequestHandler attribute)}

\begin{fulllineitems}
\phantomsection\label{service/sas_API:escape.service.sas_API.ServiceRequestHandler.request_perm}\pysigline{\bfcode{request\_perm}\strong{ = \{`PUT': (`echo',), `POST': (`echo', `sg'), `DELETE': (`echo',), `GET': (`echo', `version', `operations')\}}}
\end{fulllineitems}

\index{bounded\_layer (escape.service.sas\_API.ServiceRequestHandler attribute)}

\begin{fulllineitems}
\phantomsection\label{service/sas_API:escape.service.sas_API.ServiceRequestHandler.bounded_layer}\pysigline{\bfcode{bounded\_layer}\strong{ = `service'}}
\end{fulllineitems}

\index{log (escape.service.sas\_API.ServiceRequestHandler attribute)}

\begin{fulllineitems}
\phantomsection\label{service/sas_API:escape.service.sas_API.ServiceRequestHandler.log}\pysigline{\bfcode{log}\strong{ = \textless{}logging.Logger object\textgreater{}}}
\end{fulllineitems}

\index{echo() (escape.service.sas\_API.ServiceRequestHandler method)}

\begin{fulllineitems}
\phantomsection\label{service/sas_API:escape.service.sas_API.ServiceRequestHandler.echo}\pysiglinewithargsret{\bfcode{echo}}{}{}
Test function for REST-API
\begin{quote}\begin{description}
\item[{Returns}] \leavevmode
None

\end{description}\end{quote}

\end{fulllineitems}

\index{sg() (escape.service.sas\_API.ServiceRequestHandler method)}

\begin{fulllineitems}
\phantomsection\label{service/sas_API:escape.service.sas_API.ServiceRequestHandler.sg}\pysiglinewithargsret{\bfcode{sg}}{}{}
Main API function for Service Graph initiation

Bounded to POST HTTP verb

\end{fulllineitems}

\index{version() (escape.service.sas\_API.ServiceRequestHandler method)}

\begin{fulllineitems}
\phantomsection\label{service/sas_API:escape.service.sas_API.ServiceRequestHandler.version}\pysiglinewithargsret{\bfcode{version}}{}{}
Return with version
\begin{quote}\begin{description}
\item[{Returns}] \leavevmode
None

\end{description}\end{quote}

\end{fulllineitems}

\index{operations() (escape.service.sas\_API.ServiceRequestHandler method)}

\begin{fulllineitems}
\phantomsection\label{service/sas_API:escape.service.sas_API.ServiceRequestHandler.operations}\pysiglinewithargsret{\bfcode{operations}}{}{}
Return with allowed operations
\begin{quote}\begin{description}
\item[{Returns}] \leavevmode
None

\end{description}\end{quote}

\end{fulllineitems}


\end{fulllineitems}

\index{ServiceLayerAPI (class in escape.service.sas\_API)}

\begin{fulllineitems}
\phantomsection\label{service/sas_API:escape.service.sas_API.ServiceLayerAPI}\pysiglinewithargsret{\strong{class }\code{escape.service.sas\_API.}\bfcode{ServiceLayerAPI}}{\emph{standalone=False}, \emph{**kwargs}}{}
Bases: {\hyperref[util/api:escape.util.api.AbstractAPI]{\emph{\code{escape.util.api.AbstractAPI}}}}

Entry point for Service Adaptation Sublayer

Maintain the contact with other UNIFY layers

Implement the U - Sl reference point
\index{\_core\_name (escape.service.sas\_API.ServiceLayerAPI attribute)}

\begin{fulllineitems}
\phantomsection\label{service/sas_API:escape.service.sas_API.ServiceLayerAPI._core_name}\pysigline{\bfcode{\_core\_name}\strong{ = `service'}}
\end{fulllineitems}

\index{dependencies (escape.service.sas\_API.ServiceLayerAPI attribute)}

\begin{fulllineitems}
\phantomsection\label{service/sas_API:escape.service.sas_API.ServiceLayerAPI.dependencies}\pysigline{\bfcode{dependencies}\strong{ = (`orchestration',)}}
\end{fulllineitems}

\index{\_\_init\_\_() (escape.service.sas\_API.ServiceLayerAPI method)}

\begin{fulllineitems}
\phantomsection\label{service/sas_API:escape.service.sas_API.ServiceLayerAPI.__init__}\pysiglinewithargsret{\bfcode{\_\_init\_\_}}{\emph{standalone=False}, \emph{**kwargs}}{}~

\strong{See also:}


{\hyperref[util/api:escape.util.api.AbstractAPI.__init__]{\emph{\code{AbstractAPI.\_\_init\_\_()}}}}



\end{fulllineitems}

\index{initialize() (escape.service.sas\_API.ServiceLayerAPI method)}

\begin{fulllineitems}
\phantomsection\label{service/sas_API:escape.service.sas_API.ServiceLayerAPI.initialize}\pysiglinewithargsret{\bfcode{initialize}}{}{}~

\strong{See also:}


{\hyperref[util/api:escape.util.api.AbstractAPI.initialize]{\emph{\code{AbstractAPI.initialze()}}}}



\end{fulllineitems}

\index{shutdown() (escape.service.sas\_API.ServiceLayerAPI method)}

\begin{fulllineitems}
\phantomsection\label{service/sas_API:escape.service.sas_API.ServiceLayerAPI.shutdown}\pysiglinewithargsret{\bfcode{shutdown}}{\emph{event}}{}~

\strong{See also:}


{\hyperref[util/api:escape.util.api.AbstractAPI.shutdown]{\emph{\code{AbstractAPI.shutdown()}}}}



\end{fulllineitems}

\index{\_initiate\_rest\_api() (escape.service.sas\_API.ServiceLayerAPI method)}

\begin{fulllineitems}
\phantomsection\label{service/sas_API:escape.service.sas_API.ServiceLayerAPI._initiate_rest_api}\pysiglinewithargsret{\bfcode{\_initiate\_rest\_api}}{\emph{handler=\textless{}class escape.service.sas\_API.ServiceRequestHandler\textgreater{}}, \emph{address='localhost'}, \emph{port=8008}}{}
Initialize and set up REST API in a different thread
\begin{quote}\begin{description}
\item[{Parameters}] \leavevmode\begin{itemize}
\item {} 
\textbf{\texttt{address}} (\href{https://docs.python.org/2.7/library/functions.html\#str}{\emph{str}}) -- server address, default localhost

\item {} 
\textbf{\texttt{port}} (\href{https://docs.python.org/2.7/library/functions.html\#int}{\emph{int}}) -- port number, default 8008

\end{itemize}

\item[{Returns}] \leavevmode
None

\end{description}\end{quote}

\end{fulllineitems}

\index{\_initiate\_gui() (escape.service.sas\_API.ServiceLayerAPI method)}

\begin{fulllineitems}
\phantomsection\label{service/sas_API:escape.service.sas_API.ServiceLayerAPI._initiate_gui}\pysiglinewithargsret{\bfcode{\_initiate\_gui}}{}{}
Initiate and set up GUI

\end{fulllineitems}

\index{\_handle\_SGMappingFinishedEvent() (escape.service.sas\_API.ServiceLayerAPI method)}

\begin{fulllineitems}
\phantomsection\label{service/sas_API:escape.service.sas_API.ServiceLayerAPI._handle_SGMappingFinishedEvent}\pysiglinewithargsret{\bfcode{\_handle\_SGMappingFinishedEvent}}{\emph{event}}{}
Handle SGMappingFinishedEvent and proceed with  {\hyperref[util/nffg:escape.util.nffg.NFFG]{\emph{\code{NFFG}}}} instantiation
\begin{quote}\begin{description}
\item[{Parameters}] \leavevmode
\textbf{\texttt{event}} ({\hyperref[service/sas_mapping:escape.service.sas_mapping.SGMappingFinishedEvent]{\emph{\emph{SGMappingFinishedEvent}}}}) -- event object

\item[{Returns}] \leavevmode
None

\end{description}\end{quote}

\end{fulllineitems}

\index{request\_service() (escape.service.sas\_API.ServiceLayerAPI method)}

\begin{fulllineitems}
\phantomsection\label{service/sas_API:escape.service.sas_API.ServiceLayerAPI.request_service}\pysiglinewithargsret{\bfcode{request\_service}}{\emph{*args}, \emph{**kwargs}}{}
Initiate a Service Graph (UNIFY U-Sl API)
\begin{quote}\begin{description}
\item[{Parameters}] \leavevmode
\textbf{\texttt{sg}} ({\hyperref[util/nffg:escape.util.nffg.NFFG]{\emph{\emph{NFFG}}}}) -- service graph instance

\item[{Returns}] \leavevmode
None

\end{description}\end{quote}

\end{fulllineitems}

\index{\_instantiate\_NFFG() (escape.service.sas\_API.ServiceLayerAPI method)}

\begin{fulllineitems}
\phantomsection\label{service/sas_API:escape.service.sas_API.ServiceLayerAPI._instantiate_NFFG}\pysiglinewithargsret{\bfcode{\_instantiate\_NFFG}}{\emph{nffg}}{}
Send NFFG to Resource Orchestration Sublayer in an implementation-specific
way

General function which is used from microtask and Python thread also
\begin{quote}\begin{description}
\item[{Parameters}] \leavevmode
\textbf{\texttt{nffg}} ({\hyperref[util/nffg:escape.util.nffg.NFFG]{\emph{\emph{NFFG}}}}) -- mapped Service Graph

\item[{Returns}] \leavevmode
None

\end{description}\end{quote}

\end{fulllineitems}

\index{\_handle\_MissingVirtualViewEvent() (escape.service.sas\_API.ServiceLayerAPI method)}

\begin{fulllineitems}
\phantomsection\label{service/sas_API:escape.service.sas_API.ServiceLayerAPI._handle_MissingVirtualViewEvent}\pysiglinewithargsret{\bfcode{\_handle\_MissingVirtualViewEvent}}{\emph{event}}{}
Request virtual resource info from Orchestration layer (UNIFY Sl - Or API)

Invoked when a \code{MissingVirtualViewEvent} raised

Service layer is identified with the sid value automatically
\begin{quote}\begin{description}
\item[{Parameters}] \leavevmode
\textbf{\texttt{event}} ({\hyperref[service/sas_orchestration:escape.service.sas_orchestration.MissingVirtualViewEvent]{\emph{\emph{MissingVirtualViewEvent}}}}) -- event object

\item[{Returns}] \leavevmode
None

\end{description}\end{quote}

\end{fulllineitems}

\index{\_handle\_VirtResInfoEvent() (escape.service.sas\_API.ServiceLayerAPI method)}

\begin{fulllineitems}
\phantomsection\label{service/sas_API:escape.service.sas_API.ServiceLayerAPI._handle_VirtResInfoEvent}\pysiglinewithargsret{\bfcode{\_handle\_VirtResInfoEvent}}{\emph{event}}{}
Save requested virtual resource info as an {\hyperref[orchest/virtualization_mgmt:escape.orchest.virtualization_mgmt.ESCAPEVirtualizer]{\emph{\code{ESCAPEVirtualizer}}}}
\begin{quote}\begin{description}
\item[{Parameters}] \leavevmode
\textbf{\texttt{event}} ({\hyperref[orchest/ros_API:escape.orchest.ros_API.VirtResInfoEvent]{\emph{\emph{VirtResInfoEvent}}}}) -- event object

\item[{Returns}] \leavevmode
None

\end{description}\end{quote}

\end{fulllineitems}

\index{\_handle\_InstantiationFinishedEvent() (escape.service.sas\_API.ServiceLayerAPI method)}

\begin{fulllineitems}
\phantomsection\label{service/sas_API:escape.service.sas_API.ServiceLayerAPI._handle_InstantiationFinishedEvent}\pysiglinewithargsret{\bfcode{\_handle\_InstantiationFinishedEvent}}{\emph{event}}{}
\end{fulllineitems}


\end{fulllineitems}



\subparagraph{\emph{sas\_orchestration.py} module}
\label{service/sas_orchestration:sas-orchestration-py-module}\label{service/sas_orchestration::doc}
{\hyperref[service/sas_orchestration:escape.service.sas_orchestration.ServiceOrchestrator]{\emph{\code{ServiceOrchestrator}}}} orchestrates SG mapping and centralize layer logic

{\hyperref[service/sas_orchestration:escape.service.sas_orchestration.SGManager]{\emph{\code{SGManager}}}} stores and handles Service Graphs

{\hyperref[service/sas_orchestration:escape.service.sas_orchestration.VirtualResourceManager]{\emph{\code{VirtualResourceManager}}}} contains the functionality tided to the
layer's virtual view and virtual resources

{\hyperref[orchest/ros_orchestration:escape.orchest.ros_orchestration.NFIBManager]{\emph{\code{NFIBManager}}}} handles the Network Function Information Base and hides
implementation dependent logic


\subparagraph{Module contents}
\label{service/sas_orchestration:module-contents}\label{service/sas_orchestration:module-escape.service.sas_orchestration}\index{escape.service.sas\_orchestration (module)}
Contains classes relevant to Service Adaptation Sublayer functionality
\index{ServiceOrchestrator (class in escape.service.sas\_orchestration)}

\begin{fulllineitems}
\phantomsection\label{service/sas_orchestration:escape.service.sas_orchestration.ServiceOrchestrator}\pysiglinewithargsret{\strong{class }\code{escape.service.sas\_orchestration.}\bfcode{ServiceOrchestrator}}{\emph{layer\_API}}{}
Bases: \href{https://docs.python.org/2.7/library/functions.html\#object}{\code{object}}

Main class for the actual Service Graph processing
\index{\_\_init\_\_() (escape.service.sas\_orchestration.ServiceOrchestrator method)}

\begin{fulllineitems}
\phantomsection\label{service/sas_orchestration:escape.service.sas_orchestration.ServiceOrchestrator.__init__}\pysiglinewithargsret{\bfcode{\_\_init\_\_}}{\emph{layer\_API}}{}
Initialize main Service Layer components
\begin{quote}\begin{description}
\item[{Parameters}] \leavevmode
\textbf{\texttt{layer\_API}} ({\hyperref[service/sas_API:escape.service.sas_API.ServiceLayerAPI]{\emph{\emph{ServiceLayerAPI}}}}) -- layer API instance

\item[{Returns}] \leavevmode
None

\end{description}\end{quote}

\end{fulllineitems}

\index{initiate\_service\_graph() (escape.service.sas\_orchestration.ServiceOrchestrator method)}

\begin{fulllineitems}
\phantomsection\label{service/sas_orchestration:escape.service.sas_orchestration.ServiceOrchestrator.initiate_service_graph}\pysiglinewithargsret{\bfcode{initiate\_service\_graph}}{\emph{sg}}{}
Main function for initiating Service Graphs
\begin{quote}\begin{description}
\item[{Parameters}] \leavevmode
\textbf{\texttt{sg}} ({\hyperref[util/nffg:escape.util.nffg.NFFG]{\emph{\emph{NFFG}}}}) -- service graph stored in NFFG instance

\item[{Returns}] \leavevmode
NF-FG description

\item[{Return type}] \leavevmode
{\hyperref[util/nffg:escape.util.nffg.NFFG]{\emph{NFFG}}}

\end{description}\end{quote}

\end{fulllineitems}


\end{fulllineitems}

\index{SGManager (class in escape.service.sas\_orchestration)}

\begin{fulllineitems}
\phantomsection\label{service/sas_orchestration:escape.service.sas_orchestration.SGManager}\pysigline{\strong{class }\code{escape.service.sas\_orchestration.}\bfcode{SGManager}}
Bases: \href{https://docs.python.org/2.7/library/functions.html\#object}{\code{object}}

Store, handle and organize Service Graphs

Currently it just stores SGs in one central place
\index{\_\_init\_\_() (escape.service.sas\_orchestration.SGManager method)}

\begin{fulllineitems}
\phantomsection\label{service/sas_orchestration:escape.service.sas_orchestration.SGManager.__init__}\pysiglinewithargsret{\bfcode{\_\_init\_\_}}{}{}
Init

\end{fulllineitems}

\index{save() (escape.service.sas\_orchestration.SGManager method)}

\begin{fulllineitems}
\phantomsection\label{service/sas_orchestration:escape.service.sas_orchestration.SGManager.save}\pysiglinewithargsret{\bfcode{save}}{\emph{sg}}{}
Save SG in a dict
\begin{quote}\begin{description}
\item[{Parameters}] \leavevmode
\textbf{\texttt{sg}} ({\hyperref[util/nffg:escape.util.nffg.NFFG]{\emph{\emph{NFFG}}}}) -- Service Graph

\item[{Returns}] \leavevmode
computed id of given Service Graph

\item[{Return type}] \leavevmode
\href{https://docs.python.org/2.7/library/functions.html\#int}{int}

\end{description}\end{quote}

\end{fulllineitems}

\index{get() (escape.service.sas\_orchestration.SGManager method)}

\begin{fulllineitems}
\phantomsection\label{service/sas_orchestration:escape.service.sas_orchestration.SGManager.get}\pysiglinewithargsret{\bfcode{get}}{\emph{graph\_id}}{}
Return service graph with given id
\begin{quote}\begin{description}
\item[{Parameters}] \leavevmode
\textbf{\texttt{graph\_id}} (\href{https://docs.python.org/2.7/library/functions.html\#int}{\emph{int}}) -- graph ID

\item[{Returns}] \leavevmode
stored Service Graph

\item[{Return type}] \leavevmode
{\hyperref[util/nffg:escape.util.nffg.NFFG]{\emph{NFFG}}}

\end{description}\end{quote}

\end{fulllineitems}


\end{fulllineitems}

\index{MissingVirtualViewEvent (class in escape.service.sas\_orchestration)}

\begin{fulllineitems}
\phantomsection\label{service/sas_orchestration:escape.service.sas_orchestration.MissingVirtualViewEvent}\pysigline{\strong{class }\code{escape.service.sas\_orchestration.}\bfcode{MissingVirtualViewEvent}}
Bases: \code{pox.lib.revent.revent.Event}

Event for signaling missing virtual resource view

\end{fulllineitems}

\index{VirtualResourceManager (class in escape.service.sas\_orchestration)}

\begin{fulllineitems}
\phantomsection\label{service/sas_orchestration:escape.service.sas_orchestration.VirtualResourceManager}\pysigline{\strong{class }\code{escape.service.sas\_orchestration.}\bfcode{VirtualResourceManager}}
Bases: \code{pox.lib.revent.revent.EventMixin}

Support Service Graph mapping, follow the used virtual resources according to
the Service Graph(s) in effect

Handles object derived from :class{}`AbstractVirtualizer{}` and requested from
lower layer
\index{\_eventMixin\_events (escape.service.sas\_orchestration.VirtualResourceManager attribute)}

\begin{fulllineitems}
\phantomsection\label{service/sas_orchestration:escape.service.sas_orchestration.VirtualResourceManager._eventMixin_events}\pysigline{\bfcode{\_eventMixin\_events}\strong{ = set({[}\textless{}class `escape.service.sas\_orchestration.MissingVirtualViewEvent'\textgreater{}{]})}}
\end{fulllineitems}

\index{\_\_init\_\_() (escape.service.sas\_orchestration.VirtualResourceManager method)}

\begin{fulllineitems}
\phantomsection\label{service/sas_orchestration:escape.service.sas_orchestration.VirtualResourceManager.__init__}\pysiglinewithargsret{\bfcode{\_\_init\_\_}}{}{}
Initialize virtual resource manager
\begin{quote}\begin{description}
\item[{Returns}] \leavevmode
None

\end{description}\end{quote}

\end{fulllineitems}

\index{virtual\_view (escape.service.sas\_orchestration.VirtualResourceManager attribute)}

\begin{fulllineitems}
\phantomsection\label{service/sas_orchestration:escape.service.sas_orchestration.VirtualResourceManager.virtual_view}\pysigline{\bfcode{virtual\_view}}
Return resource info of actual layer as an {\hyperref[util/nffg:escape.util.nffg.NFFG]{\emph{\code{NFFG}}}} instance

If it isn't exist requires it from Orchestration layer
\begin{quote}\begin{description}
\item[{Returns}] \leavevmode
resource info as a Virtualizer

\item[{Return type}] \leavevmode
{\hyperref[orchest/virtualization_mgmt:escape.orchest.virtualization_mgmt.AbstractVirtualizer]{\emph{AbstractVirtualizer}}}

\end{description}\end{quote}

\end{fulllineitems}


\end{fulllineitems}



\subparagraph{\emph{escape.orchest} package}
\label{orchest/orchest:module-escape.orchest}\label{orchest/orchest::doc}\label{orchest/orchest:escape-orchest-package}\index{escape.orchest (module)}
Subpackage for classes related to UNIFY's Resource Orchestration Sublayer (ROS)


\subparagraph{Submodules}
\label{orchest/orchest:submodules}

\subparagraph{\emph{policy\_enforcement.py} module}
\label{orchest/policy_enforcement:policy-enforcement-py-module}\label{orchest/policy_enforcement::doc}
{\hyperref[orchest/policy_enforcement:escape.orchest.policy_enforcement.PolicyEnforcementError]{\emph{\code{PolicyEnforcementError}}}} represent a violation during the policy
checking process

{\hyperref[orchest/policy_enforcement:escape.orchest.policy_enforcement.PolicyEnforcementMetaClass]{\emph{\code{PolicyEnforcementMetaClass}}}} contains the main general logic which
handles the Virtualizers and enforce policies

{\hyperref[orchest/policy_enforcement:escape.orchest.policy_enforcement.PolicyEnforcement]{\emph{\code{PolicyEnforcement}}}} implements the actual enforcement logic


\subparagraph{Module contents}
\label{orchest/policy_enforcement:module-escape.orchest.policy_enforcement}\label{orchest/policy_enforcement:module-contents}\index{escape.orchest.policy\_enforcement (module)}
Contains functionality related to policy enforcement
\index{PolicyEnforcementError}

\begin{fulllineitems}
\phantomsection\label{orchest/policy_enforcement:escape.orchest.policy_enforcement.PolicyEnforcementError}\pysigline{\strong{exception }\code{escape.orchest.policy\_enforcement.}\bfcode{PolicyEnforcementError}}
Bases: \href{https://docs.python.org/2.7/library/exceptions.html\#exceptions.RuntimeError}{\code{exceptions.RuntimeError}}

Exception class to signal policy enforcement error

\end{fulllineitems}

\index{PolicyEnforcementMetaClass (class in escape.orchest.policy\_enforcement)}

\begin{fulllineitems}
\phantomsection\label{orchest/policy_enforcement:escape.orchest.policy_enforcement.PolicyEnforcementMetaClass}\pysigline{\strong{class }\code{escape.orchest.policy\_enforcement.}\bfcode{PolicyEnforcementMetaClass}}
Bases: \href{https://docs.python.org/2.7/library/functions.html\#type}{\code{type}}

Meta class for handling policy enforcement in the context of classes inherited
from {\hyperref[orchest/virtualization_mgmt:escape.orchest.virtualization_mgmt.AbstractVirtualizer]{\emph{\code{AbstractVirtualizer}}}}

If the {\hyperref[orchest/policy_enforcement:escape.orchest.policy_enforcement.PolicyEnforcement]{\emph{\code{PolicyEnforcement}}}} class contains a function which name
matches one in the actual Virtualizer then PolicyEnforcement's function will
be called first.

\begin{notice}{warning}{Warning:}
Therefore the function names must be identical!
\end{notice}

\begin{notice}{note}{Note:}
If policy checking fails a {\hyperref[orchest/policy_enforcement:escape.orchest.policy_enforcement.PolicyEnforcementError]{\emph{\code{PolicyEnforcementError}}}} should be
raised and handled in a higher layer..
\end{notice}

To use policy checking set the following class attribute:

\begin{Verbatim}[commandchars=\\\{\}]
\PYG{n}{\PYGZus{}\PYGZus{}metaclass\PYGZus{}\PYGZus{}} \PYG{o}{=} \PYG{n}{PolicyEnforcementMetaClass}
\end{Verbatim}
\index{\_\_new\_\_() (escape.orchest.policy\_enforcement.PolicyEnforcementMetaClass static method)}

\begin{fulllineitems}
\phantomsection\label{orchest/policy_enforcement:escape.orchest.policy_enforcement.PolicyEnforcementMetaClass.__new__}\pysiglinewithargsret{\strong{static }\bfcode{\_\_new\_\_}}{\emph{mcs}, \emph{name}, \emph{bases}, \emph{attrs}}{}
Magic function called before subordinated class even created
\begin{quote}\begin{description}
\item[{Parameters}] \leavevmode\begin{itemize}
\item {} 
\textbf{\texttt{name}} (\href{https://docs.python.org/2.7/library/functions.html\#str}{\emph{str}}) -- given class name

\item {} 
\textbf{\texttt{bases}} (\href{https://docs.python.org/2.7/library/functions.html\#tuple}{\emph{tuple}}) -- bases of the class

\item {} 
\textbf{\texttt{attrs}} (\href{https://docs.python.org/2.7/library/stdtypes.html\#dict}{\emph{dict}}) -- given attributes

\end{itemize}

\item[{Returns}] \leavevmode
inferred class instance

\item[{Return type}] \leavevmode
{\hyperref[orchest/virtualization_mgmt:escape.orchest.virtualization_mgmt.AbstractVirtualizer]{\emph{AbstractVirtualizer}}}

\end{description}\end{quote}

\end{fulllineitems}

\index{get\_wrapper() (escape.orchest.policy\_enforcement.PolicyEnforcementMetaClass class method)}

\begin{fulllineitems}
\phantomsection\label{orchest/policy_enforcement:escape.orchest.policy_enforcement.PolicyEnforcementMetaClass.get_wrapper}\pysiglinewithargsret{\strong{classmethod }\bfcode{get\_wrapper}}{\emph{mcs}, \emph{orig\_func}, \emph{hooks}}{}
Return a decorator function which do the policy enforcement check
\begin{quote}\begin{description}
\item[{Parameters}] \leavevmode\begin{itemize}
\item {} 
\textbf{\texttt{orig\_func}} (\emph{func}) -- original function

\item {} 
\textbf{\texttt{hooks}} (\href{https://docs.python.org/2.7/library/functions.html\#tuple}{\emph{tuple}}) -- tuple of pre and post checking functions

\end{itemize}

\item[{Raise}] \leavevmode
PolicyEnforcementError

\item[{Returns}] \leavevmode
decorator function

\item[{Return type}] \leavevmode
func

\end{description}\end{quote}

\end{fulllineitems}


\end{fulllineitems}

\index{PolicyEnforcement (class in escape.orchest.policy\_enforcement)}

\begin{fulllineitems}
\phantomsection\label{orchest/policy_enforcement:escape.orchest.policy_enforcement.PolicyEnforcement}\pysigline{\strong{class }\code{escape.orchest.policy\_enforcement.}\bfcode{PolicyEnforcement}}
Bases: \href{https://docs.python.org/2.7/library/functions.html\#object}{\code{object}}

Proxy class for policy checking

Contains the policy checking function

Binding is based on function name (checking function have to exist in this
class and its name have to stand for the \emph{pre\_} or \emph{post\_} prefix and the
name of the checked function)

\begin{notice}{warning}{Warning:}
Every PRE policy checking function is classmethod and need to have two
parameter for nameless (args) and named(kwargs) params:
\end{notice}

Example:

\begin{Verbatim}[commandchars=\\\{\}]
def pre\PYGZus{}sanity\PYGZus{}check (cls, args, kwargs):
\end{Verbatim}

\begin{notice}{warning}{Warning:}
Every POST policy checking function is classmethod and need to have three
parameter for nameless (args), named (kwargs) params and return value:
\end{notice}

Example:

\begin{Verbatim}[commandchars=\\\{\}]
def post\PYGZus{}sanity\PYGZus{}check (cls, args, kwargs, ret\PYGZus{}value):
\end{Verbatim}

\begin{notice}{note}{Note:}
The first element of args is the supervised Virtualizer (`self' param in the
original function)
\end{notice}
\index{\_\_init\_\_() (escape.orchest.policy\_enforcement.PolicyEnforcement method)}

\begin{fulllineitems}
\phantomsection\label{orchest/policy_enforcement:escape.orchest.policy_enforcement.PolicyEnforcement.__init__}\pysiglinewithargsret{\bfcode{\_\_init\_\_}}{}{}
Init

\end{fulllineitems}

\index{pre\_sanity\_check() (escape.orchest.policy\_enforcement.PolicyEnforcement class method)}

\begin{fulllineitems}
\phantomsection\label{orchest/policy_enforcement:escape.orchest.policy_enforcement.PolicyEnforcement.pre_sanity_check}\pysiglinewithargsret{\strong{classmethod }\bfcode{pre\_sanity\_check}}{\emph{args}, \emph{kwargs}}{}
Implements the the sanity check before virtualizer's sanity check is called
\begin{quote}\begin{description}
\item[{Parameters}] \leavevmode\begin{itemize}
\item {} 
\textbf{\texttt{args}} (\href{https://docs.python.org/2.7/library/functions.html\#tuple}{\emph{tuple}}) -- original nameless arguments

\item {} 
\textbf{\texttt{kwargs}} (\href{https://docs.python.org/2.7/library/stdtypes.html\#dict}{\emph{dict}}) -- original named arguments

\end{itemize}

\item[{Returns}] \leavevmode
None

\end{description}\end{quote}

\end{fulllineitems}

\index{post\_sanity\_check() (escape.orchest.policy\_enforcement.PolicyEnforcement class method)}

\begin{fulllineitems}
\phantomsection\label{orchest/policy_enforcement:escape.orchest.policy_enforcement.PolicyEnforcement.post_sanity_check}\pysiglinewithargsret{\strong{classmethod }\bfcode{post\_sanity\_check}}{\emph{args}, \emph{kwargs}, \emph{ret\_value}}{}
Implements the the sanity check after virtualizer's sanity check is called
\begin{quote}\begin{description}
\item[{Parameters}] \leavevmode\begin{itemize}
\item {} 
\textbf{\texttt{args}} (\href{https://docs.python.org/2.7/library/functions.html\#tuple}{\emph{tuple}}) -- original nameless arguments

\item {} 
\textbf{\texttt{kwargs}} (\href{https://docs.python.org/2.7/library/stdtypes.html\#dict}{\emph{dict}}) -- original named arguments

\item {} 
\textbf{\texttt{ret\_value}} -- return value of Virtualizer's policy check function

\end{itemize}

\item[{Returns}] \leavevmode
None

\end{description}\end{quote}

\end{fulllineitems}


\end{fulllineitems}



\subparagraph{\emph{ros\_orchestration.py} module}
\label{orchest/ros_orchestration:ros-orchestration-py-module}\label{orchest/ros_orchestration::doc}
{\hyperref[orchest/ros_orchestration:escape.orchest.ros_orchestration.ResourceOrchestrator]{\emph{\code{ResourceOrchestrator}}}} orchestrates {\hyperref[util/nffg:escape.util.nffg.NFFG]{\emph{\code{NFFG}}}} mapping and centralize
layer logic

{\hyperref[orchest/ros_orchestration:escape.orchest.ros_orchestration.NFFGManager]{\emph{\code{NFFGManager}}}} stores and handles Network Function Forwarding Graphs


\subparagraph{Module contents}
\label{orchest/ros_orchestration:module-contents}\label{orchest/ros_orchestration:module-escape.orchest.ros_orchestration}\index{escape.orchest.ros\_orchestration (module)}
Contains classes relevant to Resource Orchestration Sublayer functionality
\index{ResourceOrchestrator (class in escape.orchest.ros\_orchestration)}

\begin{fulllineitems}
\phantomsection\label{orchest/ros_orchestration:escape.orchest.ros_orchestration.ResourceOrchestrator}\pysiglinewithargsret{\strong{class }\code{escape.orchest.ros\_orchestration.}\bfcode{ResourceOrchestrator}}{\emph{layer\_API}}{}
Bases: \href{https://docs.python.org/2.7/library/functions.html\#object}{\code{object}}

Main class for the handling of the ROS-level mapping functions
\index{\_\_init\_\_() (escape.orchest.ros\_orchestration.ResourceOrchestrator method)}

\begin{fulllineitems}
\phantomsection\label{orchest/ros_orchestration:escape.orchest.ros_orchestration.ResourceOrchestrator.__init__}\pysiglinewithargsret{\bfcode{\_\_init\_\_}}{\emph{layer\_API}}{}
Initialize main Resource Orchestration Layer components
\begin{quote}\begin{description}
\item[{Parameters}] \leavevmode
\textbf{\texttt{layer\_API}} ({\hyperref[orchest/ros_API:escape.orchest.ros_API.ResourceOrchestrationAPI]{\emph{\code{ResourceOrchestrationAPI}}}}) -- layer API instance

\item[{Returns}] \leavevmode
None

\end{description}\end{quote}

\end{fulllineitems}

\index{instantiate\_nffg() (escape.orchest.ros\_orchestration.ResourceOrchestrator method)}

\begin{fulllineitems}
\phantomsection\label{orchest/ros_orchestration:escape.orchest.ros_orchestration.ResourceOrchestrator.instantiate_nffg}\pysiglinewithargsret{\bfcode{instantiate\_nffg}}{\emph{nffg}}{}
Main API function for NF-FG instantiation
\begin{quote}\begin{description}
\item[{Parameters}] \leavevmode
\textbf{\texttt{nffg}} ({\hyperref[util/nffg:escape.util.nffg.NFFG]{\emph{\emph{NFFG}}}}) -- NFFG instance

\item[{Returns}] \leavevmode
mapped NFFG instance

\item[{Return type}] \leavevmode
{\hyperref[util/nffg:escape.util.nffg.NFFG]{\emph{NFFG}}}

\end{description}\end{quote}

\end{fulllineitems}


\end{fulllineitems}

\index{NFFGManager (class in escape.orchest.ros\_orchestration)}

\begin{fulllineitems}
\phantomsection\label{orchest/ros_orchestration:escape.orchest.ros_orchestration.NFFGManager}\pysigline{\strong{class }\code{escape.orchest.ros\_orchestration.}\bfcode{NFFGManager}}
Bases: \href{https://docs.python.org/2.7/library/functions.html\#object}{\code{object}}

Store, handle and organize Network Function Forwarding Graphs
\index{\_\_init\_\_() (escape.orchest.ros\_orchestration.NFFGManager method)}

\begin{fulllineitems}
\phantomsection\label{orchest/ros_orchestration:escape.orchest.ros_orchestration.NFFGManager.__init__}\pysiglinewithargsret{\bfcode{\_\_init\_\_}}{}{}
Init

\end{fulllineitems}

\index{save() (escape.orchest.ros\_orchestration.NFFGManager method)}

\begin{fulllineitems}
\phantomsection\label{orchest/ros_orchestration:escape.orchest.ros_orchestration.NFFGManager.save}\pysiglinewithargsret{\bfcode{save}}{\emph{nffg}}{}
Save NF-FG in a dict
\begin{quote}\begin{description}
\item[{Parameters}] \leavevmode
\textbf{\texttt{nffg}} ({\hyperref[util/nffg:escape.util.nffg.NFFG]{\emph{\emph{NFFG}}}}) -- Network Function Forwarding Graph

\item[{Returns}] \leavevmode
generated ID of given NF-FG

\item[{Return type}] \leavevmode
\href{https://docs.python.org/2.7/library/functions.html\#int}{int}

\end{description}\end{quote}

\end{fulllineitems}

\index{get() (escape.orchest.ros\_orchestration.NFFGManager method)}

\begin{fulllineitems}
\phantomsection\label{orchest/ros_orchestration:escape.orchest.ros_orchestration.NFFGManager.get}\pysiglinewithargsret{\bfcode{get}}{\emph{nffg\_id}}{}
Return NF-FG with given id
\begin{quote}\begin{description}
\item[{Parameters}] \leavevmode
\textbf{\texttt{nffg\_id}} (\href{https://docs.python.org/2.7/library/functions.html\#int}{\emph{int}}) -- ID of NF-FG

\item[{Returns}] \leavevmode
NF-Fg instance

\item[{Return type}] \leavevmode
{\hyperref[util/nffg:escape.util.nffg.NFFG]{\emph{NFFG}}}

\end{description}\end{quote}

\end{fulllineitems}


\end{fulllineitems}

\index{NFIBManager (class in escape.orchest.ros\_orchestration)}

\begin{fulllineitems}
\phantomsection\label{orchest/ros_orchestration:escape.orchest.ros_orchestration.NFIBManager}\pysigline{\strong{class }\code{escape.orchest.ros\_orchestration.}\bfcode{NFIBManager}}
Bases: \href{https://docs.python.org/2.7/library/functions.html\#object}{\code{object}}

Manage the handling of Network Function Information Base
\index{\_\_init\_\_() (escape.orchest.ros\_orchestration.NFIBManager method)}

\begin{fulllineitems}
\phantomsection\label{orchest/ros_orchestration:escape.orchest.ros_orchestration.NFIBManager.__init__}\pysiglinewithargsret{\bfcode{\_\_init\_\_}}{}{}
Init

\end{fulllineitems}

\index{add() (escape.orchest.ros\_orchestration.NFIBManager method)}

\begin{fulllineitems}
\phantomsection\label{orchest/ros_orchestration:escape.orchest.ros_orchestration.NFIBManager.add}\pysiglinewithargsret{\bfcode{add}}{\emph{nf}}{}
\end{fulllineitems}

\index{remove() (escape.orchest.ros\_orchestration.NFIBManager method)}

\begin{fulllineitems}
\phantomsection\label{orchest/ros_orchestration:escape.orchest.ros_orchestration.NFIBManager.remove}\pysiglinewithargsret{\bfcode{remove}}{\emph{nf\_id}}{}
\end{fulllineitems}

\index{getNF() (escape.orchest.ros\_orchestration.NFIBManager method)}

\begin{fulllineitems}
\phantomsection\label{orchest/ros_orchestration:escape.orchest.ros_orchestration.NFIBManager.getNF}\pysiglinewithargsret{\bfcode{getNF}}{\emph{nf\_id}}{}
\end{fulllineitems}


\end{fulllineitems}



\subparagraph{\emph{ros\_API.py} module}
\label{orchest/ros_API:ros-api-py-module}\label{orchest/ros_API::doc}
{\hyperref[orchest/ros_API:escape.orchest.ros_API.InstallNFFGEvent]{\emph{\code{InstallNFFGEvent}}}} can send mapped NF-FG to the lower layer

{\hyperref[orchest/ros_API:escape.orchest.ros_API.VirtResInfoEvent]{\emph{\code{VirtResInfoEvent}}}} can send back virtual resource info requested from
upper layer

{\hyperref[orchest/ros_API:escape.orchest.ros_API.GetGlobalResInfoEvent]{\emph{\code{GetGlobalResInfoEvent}}}} can request global resource info from lower layer

{\hyperref[orchest/ros_API:escape.orchest.ros_API.ResourceOrchestrationAPI]{\emph{\code{ResourceOrchestrationAPI}}}} represents the ROS layer and implement all
related functionality


\subparagraph{Module contents}
\label{orchest/ros_API:module-contents}\label{orchest/ros_API:module-escape.orchest.ros_API}\index{escape.orchest.ros\_API (module)}
Implements the platform and POX dependent logic for the Resource Orchestration
Sublayer
\index{InstallNFFGEvent (class in escape.orchest.ros\_API)}

\begin{fulllineitems}
\phantomsection\label{orchest/ros_API:escape.orchest.ros_API.InstallNFFGEvent}\pysiglinewithargsret{\strong{class }\code{escape.orchest.ros\_API.}\bfcode{InstallNFFGEvent}}{\emph{mapped\_nffg}}{}
Bases: \code{pox.lib.revent.revent.Event}

Event for passing mapped {\hyperref[util/nffg:escape.util.nffg.NFFG]{\emph{\code{NFFG}}}} to Controller
Adaptation Sublayer
\index{\_\_init\_\_() (escape.orchest.ros\_API.InstallNFFGEvent method)}

\begin{fulllineitems}
\phantomsection\label{orchest/ros_API:escape.orchest.ros_API.InstallNFFGEvent.__init__}\pysiglinewithargsret{\bfcode{\_\_init\_\_}}{\emph{mapped\_nffg}}{}
Init
\begin{quote}\begin{description}
\item[{Parameters}] \leavevmode
\textbf{\texttt{mapped\_nffg}} ({\hyperref[util/nffg:escape.util.nffg.NFFG]{\emph{\emph{NFFG}}}}) -- NF-FG graph need to be installed

\end{description}\end{quote}

\end{fulllineitems}


\end{fulllineitems}

\index{VirtResInfoEvent (class in escape.orchest.ros\_API)}

\begin{fulllineitems}
\phantomsection\label{orchest/ros_API:escape.orchest.ros_API.VirtResInfoEvent}\pysiglinewithargsret{\strong{class }\code{escape.orchest.ros\_API.}\bfcode{VirtResInfoEvent}}{\emph{resource\_info}}{}
Bases: \code{pox.lib.revent.revent.Event}

Event for sending back requested virtual resource info
\index{\_\_init\_\_() (escape.orchest.ros\_API.VirtResInfoEvent method)}

\begin{fulllineitems}
\phantomsection\label{orchest/ros_API:escape.orchest.ros_API.VirtResInfoEvent.__init__}\pysiglinewithargsret{\bfcode{\_\_init\_\_}}{\emph{resource\_info}}{}
Init
\begin{quote}\begin{description}
\item[{Parameters}] \leavevmode
\textbf{\texttt{resource\_info}} ({\hyperref[orchest/virtualization_mgmt:escape.orchest.virtualization_mgmt.ESCAPEVirtualizer]{\emph{\emph{ESCAPEVirtualizer}}}}) -- virtual resource info

\end{description}\end{quote}

\end{fulllineitems}


\end{fulllineitems}

\index{GetGlobalResInfoEvent (class in escape.orchest.ros\_API)}

\begin{fulllineitems}
\phantomsection\label{orchest/ros_API:escape.orchest.ros_API.GetGlobalResInfoEvent}\pysigline{\strong{class }\code{escape.orchest.ros\_API.}\bfcode{GetGlobalResInfoEvent}}
Bases: \code{pox.lib.revent.revent.Event}

Event for requesting \code{DomainVirtualizer} from CAS

\end{fulllineitems}

\index{InstantiationFinishedEvent (class in escape.orchest.ros\_API)}

\begin{fulllineitems}
\phantomsection\label{orchest/ros_API:escape.orchest.ros_API.InstantiationFinishedEvent}\pysiglinewithargsret{\strong{class }\code{escape.orchest.ros\_API.}\bfcode{InstantiationFinishedEvent}}{\emph{success}, \emph{error=None}}{}
Bases: \code{pox.lib.revent.revent.Event}

Event for signalling end of mapping process finished with success
\index{\_\_init\_\_() (escape.orchest.ros\_API.InstantiationFinishedEvent method)}

\begin{fulllineitems}
\phantomsection\label{orchest/ros_API:escape.orchest.ros_API.InstantiationFinishedEvent.__init__}\pysiglinewithargsret{\bfcode{\_\_init\_\_}}{\emph{success}, \emph{error=None}}{}
\end{fulllineitems}


\end{fulllineitems}

\index{ResourceOrchestrationAPI (class in escape.orchest.ros\_API)}

\begin{fulllineitems}
\phantomsection\label{orchest/ros_API:escape.orchest.ros_API.ResourceOrchestrationAPI}\pysiglinewithargsret{\strong{class }\code{escape.orchest.ros\_API.}\bfcode{ResourceOrchestrationAPI}}{\emph{standalone=False}, \emph{**kwargs}}{}
Bases: {\hyperref[util/api:escape.util.api.AbstractAPI]{\emph{\code{escape.util.api.AbstractAPI}}}}

Entry point for Resource Orchestration Sublayer (ROS)

Maintain the contact with other UNIFY layers

Implement the Sl - Or reference point
\index{\_core\_name (escape.orchest.ros\_API.ResourceOrchestrationAPI attribute)}

\begin{fulllineitems}
\phantomsection\label{orchest/ros_API:escape.orchest.ros_API.ResourceOrchestrationAPI._core_name}\pysigline{\bfcode{\_core\_name}\strong{ = `orchestration'}}
\end{fulllineitems}

\index{dependencies (escape.orchest.ros\_API.ResourceOrchestrationAPI attribute)}

\begin{fulllineitems}
\phantomsection\label{orchest/ros_API:escape.orchest.ros_API.ResourceOrchestrationAPI.dependencies}\pysigline{\bfcode{dependencies}\strong{ = (`adaptation',)}}
\end{fulllineitems}

\index{\_\_init\_\_() (escape.orchest.ros\_API.ResourceOrchestrationAPI method)}

\begin{fulllineitems}
\phantomsection\label{orchest/ros_API:escape.orchest.ros_API.ResourceOrchestrationAPI.__init__}\pysiglinewithargsret{\bfcode{\_\_init\_\_}}{\emph{standalone=False}, \emph{**kwargs}}{}~

\strong{See also:}


{\hyperref[util/api:escape.util.api.AbstractAPI.__init__]{\emph{\code{AbstractAPI.\_\_init\_\_()}}}}



\end{fulllineitems}

\index{initialize() (escape.orchest.ros\_API.ResourceOrchestrationAPI method)}

\begin{fulllineitems}
\phantomsection\label{orchest/ros_API:escape.orchest.ros_API.ResourceOrchestrationAPI.initialize}\pysiglinewithargsret{\bfcode{initialize}}{}{}~

\strong{See also:}


{\hyperref[util/api:escape.util.api.AbstractAPI.initialize]{\emph{\code{AbstractAPI.initialze()}}}}



\end{fulllineitems}

\index{shutdown() (escape.orchest.ros\_API.ResourceOrchestrationAPI method)}

\begin{fulllineitems}
\phantomsection\label{orchest/ros_API:escape.orchest.ros_API.ResourceOrchestrationAPI.shutdown}\pysiglinewithargsret{\bfcode{shutdown}}{\emph{event}}{}~

\strong{See also:}


{\hyperref[util/api:escape.util.api.AbstractAPI.shutdown]{\emph{\code{AbstractAPI.shutdown()}}}}



\end{fulllineitems}

\index{\_handle\_NFFGMappingFinishedEvent() (escape.orchest.ros\_API.ResourceOrchestrationAPI method)}

\begin{fulllineitems}
\phantomsection\label{orchest/ros_API:escape.orchest.ros_API.ResourceOrchestrationAPI._handle_NFFGMappingFinishedEvent}\pysiglinewithargsret{\bfcode{\_handle\_NFFGMappingFinishedEvent}}{\emph{event}}{}
Handle NFFGMappingFinishedEvent and proceed with  {\hyperref[util/nffg:escape.util.nffg.NFFG]{\emph{\code{NFFG}}}} installation
\begin{quote}\begin{description}
\item[{Parameters}] \leavevmode
\textbf{\texttt{event}} ({\hyperref[orchest/ros_mapping:escape.orchest.ros_mapping.NFFGMappingFinishedEvent]{\emph{\emph{NFFGMappingFinishedEvent}}}}) -- event object

\item[{Returns}] \leavevmode
None

\end{description}\end{quote}

\end{fulllineitems}

\index{\_handle\_InstantiateNFFGEvent() (escape.orchest.ros\_API.ResourceOrchestrationAPI method)}

\begin{fulllineitems}
\phantomsection\label{orchest/ros_API:escape.orchest.ros_API.ResourceOrchestrationAPI._handle_InstantiateNFFGEvent}\pysiglinewithargsret{\bfcode{\_handle\_InstantiateNFFGEvent}}{\emph{*args}, \emph{**kwargs}}{}
Instantiate given NF-FG (UNIFY Sl - Or API)
\begin{quote}\begin{description}
\item[{Parameters}] \leavevmode
\textbf{\texttt{event}} ({\hyperref[service/sas_API:escape.service.sas_API.InstantiateNFFGEvent]{\emph{\emph{InstantiateNFFGEvent}}}}) -- event object contains NF-FG

\item[{Returns}] \leavevmode
None

\end{description}\end{quote}

\end{fulllineitems}

\index{\_install\_NFFG() (escape.orchest.ros\_API.ResourceOrchestrationAPI method)}

\begin{fulllineitems}
\phantomsection\label{orchest/ros_API:escape.orchest.ros_API.ResourceOrchestrationAPI._install_NFFG}\pysiglinewithargsret{\bfcode{\_install\_NFFG}}{\emph{mapped\_nffg}}{}
Send mapped {\hyperref[util/nffg:escape.util.nffg.NFFG]{\emph{\code{NFFG}}}} to Controller Adaptation
Sublayer in an implementation-specific way

General function which is used from microtask and Python thread also
\begin{quote}\begin{description}
\item[{Parameters}] \leavevmode
\textbf{\texttt{mapped\_nffg}} ({\hyperref[util/nffg:escape.util.nffg.NFFG]{\emph{\emph{NFFG}}}}) -- mapped NF-FG

\item[{Returns}] \leavevmode
None

\end{description}\end{quote}

\end{fulllineitems}

\index{\_handle\_GetVirtResInfoEvent() (escape.orchest.ros\_API.ResourceOrchestrationAPI method)}

\begin{fulllineitems}
\phantomsection\label{orchest/ros_API:escape.orchest.ros_API.ResourceOrchestrationAPI._handle_GetVirtResInfoEvent}\pysiglinewithargsret{\bfcode{\_handle\_GetVirtResInfoEvent}}{\emph{event}}{}
Generate virtual resource info and send back to SAS
\begin{quote}\begin{description}
\item[{Parameters}] \leavevmode
\textbf{\texttt{event}} ({\hyperref[service/sas_API:escape.service.sas_API.GetVirtResInfoEvent]{\emph{\emph{GetVirtResInfoEvent}}}}) -- event object contains service layer id

\item[{Returns}] \leavevmode
None

\end{description}\end{quote}

\end{fulllineitems}

\index{\_handle\_MissingGlobalViewEvent() (escape.orchest.ros\_API.ResourceOrchestrationAPI method)}

\begin{fulllineitems}
\phantomsection\label{orchest/ros_API:escape.orchest.ros_API.ResourceOrchestrationAPI._handle_MissingGlobalViewEvent}\pysiglinewithargsret{\bfcode{\_handle\_MissingGlobalViewEvent}}{\emph{event}}{}
Request global resource info from CAS (UNIFY Or - CA API)

Invoked when a \code{MissingGlobalViewEvent} raised
\begin{quote}\begin{description}
\item[{Parameters}] \leavevmode
\textbf{\texttt{event}} ({\hyperref[orchest/virtualization_mgmt:escape.orchest.virtualization_mgmt.MissingGlobalViewEvent]{\emph{\emph{MissingGlobalViewEvent}}}}) -- event object

\item[{Returns}] \leavevmode
None

\end{description}\end{quote}

\end{fulllineitems}

\index{\_handle\_GlobalResInfoEvent() (escape.orchest.ros\_API.ResourceOrchestrationAPI method)}

\begin{fulllineitems}
\phantomsection\label{orchest/ros_API:escape.orchest.ros_API.ResourceOrchestrationAPI._handle_GlobalResInfoEvent}\pysiglinewithargsret{\bfcode{\_handle\_GlobalResInfoEvent}}{\emph{event}}{}
Save requested global resource info as the \code{DomainVirtualizer}
\begin{quote}\begin{description}
\item[{Parameters}] \leavevmode
\textbf{\texttt{event}} ({\hyperref[adapt/cas_API:escape.adapt.cas_API.GlobalResInfoEvent]{\emph{\emph{GlobalResInfoEvent}}}}) -- event object contains resource info

\item[{Returns}] \leavevmode
None

\end{description}\end{quote}

\end{fulllineitems}

\index{\_handle\_InstallationFinishedEvent() (escape.orchest.ros\_API.ResourceOrchestrationAPI method)}

\begin{fulllineitems}
\phantomsection\label{orchest/ros_API:escape.orchest.ros_API.ResourceOrchestrationAPI._handle_InstallationFinishedEvent}\pysiglinewithargsret{\bfcode{\_handle\_InstallationFinishedEvent}}{\emph{event}}{}
\end{fulllineitems}


\end{fulllineitems}



\subparagraph{\emph{ros\_mapping.py} module}
\label{orchest/ros_mapping::doc}\label{orchest/ros_mapping:ros-mapping-py-module}
{\hyperref[orchest/ros_mapping:escape.orchest.ros_mapping.ESCAPEMappingStrategy]{\emph{\code{ESCAPEMappingStrategy}}}} implements a default {\hyperref[util/nffg:escape.util.nffg.NFFG]{\emph{\code{NFFG}}}} mapping algorithm
of ESCAPEv2

{\hyperref[orchest/ros_mapping:escape.orchest.ros_mapping.ResourceOrchestrationMapper]{\emph{\code{ResourceOrchestrationMapper}}}} perform the supplementary tasks for
{\hyperref[util/nffg:escape.util.nffg.NFFG]{\emph{\code{NFFG}}}} mapping


\subparagraph{Module contents}
\label{orchest/ros_mapping:module-contents}\label{orchest/ros_mapping:module-escape.orchest.ros_mapping}\index{escape.orchest.ros\_mapping (module)}
Contains classes which implement {\hyperref[util/nffg:escape.util.nffg.NFFG]{\emph{\code{NFFG}}}}
mapping functionality
\index{ESCAPEMappingStrategy (class in escape.orchest.ros\_mapping)}

\begin{fulllineitems}
\phantomsection\label{orchest/ros_mapping:escape.orchest.ros_mapping.ESCAPEMappingStrategy}\pysigline{\strong{class }\code{escape.orchest.ros\_mapping.}\bfcode{ESCAPEMappingStrategy}}
Bases: {\hyperref[util/mapping:escape.util.mapping.AbstractMappingStrategy]{\emph{\code{escape.util.mapping.AbstractMappingStrategy}}}}

Implement a strategy to map initial {\hyperref[util/nffg:escape.util.nffg.NFFG]{\emph{\code{NFFG}}}}
into extended {\hyperref[util/nffg:escape.util.nffg.NFFG]{\emph{\code{NFFG}}}}
\index{\_\_init\_\_() (escape.orchest.ros\_mapping.ESCAPEMappingStrategy method)}

\begin{fulllineitems}
\phantomsection\label{orchest/ros_mapping:escape.orchest.ros_mapping.ESCAPEMappingStrategy.__init__}\pysiglinewithargsret{\bfcode{\_\_init\_\_}}{}{}
Init

\end{fulllineitems}

\index{map() (escape.orchest.ros\_mapping.ESCAPEMappingStrategy class method)}

\begin{fulllineitems}
\phantomsection\label{orchest/ros_mapping:escape.orchest.ros_mapping.ESCAPEMappingStrategy.map}\pysiglinewithargsret{\strong{classmethod }\bfcode{map}}{\emph{graph}, \emph{resource}}{}
Default mapping algorithm of ESCAPEv2
\begin{quote}\begin{description}
\item[{Parameters}] \leavevmode\begin{itemize}
\item {} 
\textbf{\texttt{graph}} ({\hyperref[util/nffg:escape.util.nffg.NFFG]{\emph{\emph{NFFG}}}}) -- Network Function forwarding Graph

\item {} 
\textbf{\texttt{resource}} ({\hyperref[util/nffg:escape.util.nffg.NFFG]{\emph{\emph{NFFG}}}}) -- global virtual resource info

\end{itemize}

\item[{Returns}] \leavevmode
mapped Network Function Forwarding Graph

\item[{Return type}] \leavevmode
{\hyperref[util/nffg:escape.util.nffg.NFFG]{\emph{NFFG}}}

\end{description}\end{quote}

\end{fulllineitems}


\end{fulllineitems}

\index{NFFGMappingFinishedEvent (class in escape.orchest.ros\_mapping)}

\begin{fulllineitems}
\phantomsection\label{orchest/ros_mapping:escape.orchest.ros_mapping.NFFGMappingFinishedEvent}\pysiglinewithargsret{\strong{class }\code{escape.orchest.ros\_mapping.}\bfcode{NFFGMappingFinishedEvent}}{\emph{nffg}}{}
Bases: \code{pox.lib.revent.revent.Event}

Event for signaling the end of NF-FG mapping
\index{\_\_init\_\_() (escape.orchest.ros\_mapping.NFFGMappingFinishedEvent method)}

\begin{fulllineitems}
\phantomsection\label{orchest/ros_mapping:escape.orchest.ros_mapping.NFFGMappingFinishedEvent.__init__}\pysiglinewithargsret{\bfcode{\_\_init\_\_}}{\emph{nffg}}{}
Init
\begin{quote}\begin{description}
\item[{Parameters}] \leavevmode
\textbf{\texttt{nffg}} ({\hyperref[util/nffg:escape.util.nffg.NFFG]{\emph{\emph{NFFG}}}}) -- NF-FG need to be installed

\end{description}\end{quote}

\end{fulllineitems}


\end{fulllineitems}

\index{ResourceOrchestrationMapper (class in escape.orchest.ros\_mapping)}

\begin{fulllineitems}
\phantomsection\label{orchest/ros_mapping:escape.orchest.ros_mapping.ResourceOrchestrationMapper}\pysigline{\strong{class }\code{escape.orchest.ros\_mapping.}\bfcode{ResourceOrchestrationMapper}}
Bases: {\hyperref[util/mapping:escape.util.mapping.AbstractMapper]{\emph{\code{escape.util.mapping.AbstractMapper}}}}

Helper class for mapping NF-FG on global virtual view
\index{\_eventMixin\_events (escape.orchest.ros\_mapping.ResourceOrchestrationMapper attribute)}

\begin{fulllineitems}
\phantomsection\label{orchest/ros_mapping:escape.orchest.ros_mapping.ResourceOrchestrationMapper._eventMixin_events}\pysigline{\bfcode{\_eventMixin\_events}\strong{ = set({[}\textless{}class `escape.orchest.ros\_mapping.NFFGMappingFinishedEvent'\textgreater{}{]})}}
\end{fulllineitems}

\index{\_\_init\_\_() (escape.orchest.ros\_mapping.ResourceOrchestrationMapper method)}

\begin{fulllineitems}
\phantomsection\label{orchest/ros_mapping:escape.orchest.ros_mapping.ResourceOrchestrationMapper.__init__}\pysiglinewithargsret{\bfcode{\_\_init\_\_}}{}{}
Init Resource Orchestrator mapper
\begin{quote}\begin{description}
\item[{Returns}] \leavevmode
None

\end{description}\end{quote}

\end{fulllineitems}

\index{orchestrate() (escape.orchest.ros\_mapping.ResourceOrchestrationMapper method)}

\begin{fulllineitems}
\phantomsection\label{orchest/ros_mapping:escape.orchest.ros_mapping.ResourceOrchestrationMapper.orchestrate}\pysiglinewithargsret{\bfcode{orchestrate}}{\emph{input\_graph}, \emph{resource\_view}}{}
Orchestrate mapping of given NF-FG on given global resource
\begin{quote}\begin{description}
\item[{Parameters}] \leavevmode\begin{itemize}
\item {} 
\textbf{\texttt{input\_graph}} ({\hyperref[util/nffg:escape.util.nffg.NFFG]{\emph{\emph{NFFG}}}}) -- Network Function Forwarding Graph

\item {} 
\textbf{\texttt{resource\_view}} ({\hyperref[adapt/adaptation:escape.adapt.adaptation.DomainVirtualizer]{\emph{\emph{DomainVirtualizer}}}}) -- global resource view

\end{itemize}

\item[{Returns}] \leavevmode
mapped Network Function Forwarding Graph

\item[{Return type}] \leavevmode
{\hyperref[util/nffg:escape.util.nffg.NFFG]{\emph{NFFG}}}

\end{description}\end{quote}

\end{fulllineitems}

\index{\_mapping\_finished() (escape.orchest.ros\_mapping.ResourceOrchestrationMapper method)}

\begin{fulllineitems}
\phantomsection\label{orchest/ros_mapping:escape.orchest.ros_mapping.ResourceOrchestrationMapper._mapping_finished}\pysiglinewithargsret{\bfcode{\_mapping\_finished}}{\emph{nffg}}{}
Called from a separate thread when the mapping process is finished
\begin{quote}\begin{description}
\item[{Parameters}] \leavevmode
\textbf{\texttt{nffg}} ({\hyperref[util/nffg:escape.util.nffg.NFFG]{\emph{\emph{NFFG}}}}) -- mapped NF-FG

\item[{Returns}] \leavevmode
None

\end{description}\end{quote}

\end{fulllineitems}


\end{fulllineitems}



\subparagraph{\emph{virtualization\_mgmt.py} module}
\label{orchest/virtualization_mgmt::doc}\label{orchest/virtualization_mgmt:virtualization-mgmt-py-module}
{\hyperref[orchest/virtualization_mgmt:escape.orchest.virtualization_mgmt.AbstractVirtualizer]{\emph{\code{AbstractVirtualizer}}}} contains the  central logic of Virtualizers

{\hyperref[orchest/virtualization_mgmt:escape.orchest.virtualization_mgmt.ESCAPEVirtualizer]{\emph{\code{ESCAPEVirtualizer}}}} implement the standard virtualization logic of the
Resource Orchestration Sublayer

{\hyperref[orchest/virtualization_mgmt:escape.orchest.virtualization_mgmt.VirtualizerManager]{\emph{\code{VirtualizerManager}}}} stores and handles the virtualizers


\subparagraph{Module contents}
\label{orchest/virtualization_mgmt:module-contents}\label{orchest/virtualization_mgmt:module-escape.orchest.virtualization_mgmt}\index{escape.orchest.virtualization\_mgmt (module)}
Contains components relevant to virtualization of resources and views
\index{AbstractVirtualizer (class in escape.orchest.virtualization\_mgmt)}

\begin{fulllineitems}
\phantomsection\label{orchest/virtualization_mgmt:escape.orchest.virtualization_mgmt.AbstractVirtualizer}\pysigline{\strong{class }\code{escape.orchest.virtualization\_mgmt.}\bfcode{AbstractVirtualizer}}
Bases: \href{https://docs.python.org/2.7/library/functions.html\#object}{\code{object}}

Abstract class for actual Virtualizers

Follows the Proxy design pattern
\index{\_\_metaclass\_\_ (escape.orchest.virtualization\_mgmt.AbstractVirtualizer attribute)}

\begin{fulllineitems}
\phantomsection\label{orchest/virtualization_mgmt:escape.orchest.virtualization_mgmt.AbstractVirtualizer.__metaclass__}\pysigline{\bfcode{\_\_metaclass\_\_}}
alias of \code{PolicyEnforcementMetaClass}

\end{fulllineitems}

\index{\_\_init\_\_() (escape.orchest.virtualization\_mgmt.AbstractVirtualizer method)}

\begin{fulllineitems}
\phantomsection\label{orchest/virtualization_mgmt:escape.orchest.virtualization_mgmt.AbstractVirtualizer.__init__}\pysiglinewithargsret{\bfcode{\_\_init\_\_}}{}{}
Init

\end{fulllineitems}

\index{get\_resource\_info() (escape.orchest.virtualization\_mgmt.AbstractVirtualizer method)}

\begin{fulllineitems}
\phantomsection\label{orchest/virtualization_mgmt:escape.orchest.virtualization_mgmt.AbstractVirtualizer.get_resource_info}\pysiglinewithargsret{\bfcode{get\_resource\_info}}{}{}
Hides object's mechanism and return with a resource object derived from
{\hyperref[util/nffg:escape.util.nffg.NFFG]{\emph{\code{NFFG}}}}

\begin{notice}{warning}{Warning:}
Derived class have to override this function
\end{notice}
\begin{quote}\begin{description}
\item[{Raise}] \leavevmode
NotImplementedError

\item[{Returns}] \leavevmode
resource info

\item[{Return type}] \leavevmode
{\hyperref[util/nffg:escape.util.nffg.NFFG]{\emph{NFFG}}}

\end{description}\end{quote}

\end{fulllineitems}

\index{sanity\_check() (escape.orchest.virtualization\_mgmt.AbstractVirtualizer method)}

\begin{fulllineitems}
\phantomsection\label{orchest/virtualization_mgmt:escape.orchest.virtualization_mgmt.AbstractVirtualizer.sanity_check}\pysiglinewithargsret{\bfcode{sanity\_check}}{\emph{*args}, \emph{**kwargs}}{}
Place-holder for sanity check which implemented in
\code{PolicyEnforcement}
\begin{quote}\begin{description}
\item[{Parameters}] \leavevmode
\textbf{\texttt{nffg}} ({\hyperref[util/nffg:escape.util.nffg.NFFG]{\emph{\emph{NFFG}}}}) -- NFFG instance

\item[{Returns}] \leavevmode
None

\end{description}\end{quote}

\end{fulllineitems}


\end{fulllineitems}

\index{ESCAPEVirtualizer (class in escape.orchest.virtualization\_mgmt)}

\begin{fulllineitems}
\phantomsection\label{orchest/virtualization_mgmt:escape.orchest.virtualization_mgmt.ESCAPEVirtualizer}\pysigline{\strong{class }\code{escape.orchest.virtualization\_mgmt.}\bfcode{ESCAPEVirtualizer}}
Bases: {\hyperref[orchest/virtualization_mgmt:escape.orchest.virtualization_mgmt.AbstractVirtualizer]{\emph{\code{escape.orchest.virtualization\_mgmt.AbstractVirtualizer}}}}

Actual Virtualizer class for ESCAPEv2
\index{\_\_init\_\_() (escape.orchest.virtualization\_mgmt.ESCAPEVirtualizer method)}

\begin{fulllineitems}
\phantomsection\label{orchest/virtualization_mgmt:escape.orchest.virtualization_mgmt.ESCAPEVirtualizer.__init__}\pysiglinewithargsret{\bfcode{\_\_init\_\_}}{}{}
Init

\end{fulllineitems}

\index{get\_resource\_info() (escape.orchest.virtualization\_mgmt.ESCAPEVirtualizer method)}

\begin{fulllineitems}
\phantomsection\label{orchest/virtualization_mgmt:escape.orchest.virtualization_mgmt.ESCAPEVirtualizer.get_resource_info}\pysiglinewithargsret{\bfcode{get\_resource\_info}}{}{}
Hides object's mechanism and return with a resource object derived from
{\hyperref[util/nffg:escape.util.nffg.NFFG]{\emph{\code{NFFG}}}}
\begin{quote}\begin{description}
\item[{Returns}] \leavevmode
virtual resource info

\item[{Return type}] \leavevmode
{\hyperref[util/nffg:escape.util.nffg.NFFG]{\emph{NFFG}}}

\end{description}\end{quote}

\end{fulllineitems}

\index{sanity\_check() (escape.orchest.virtualization\_mgmt.ESCAPEVirtualizer method)}

\begin{fulllineitems}
\phantomsection\label{orchest/virtualization_mgmt:escape.orchest.virtualization_mgmt.ESCAPEVirtualizer.sanity_check}\pysiglinewithargsret{\bfcode{sanity\_check}}{\emph{*args}, \emph{**kwargs}}{}
Placeholder method for policy checking.

Return the virtual resource info for the post checker function
\begin{quote}\begin{description}
\item[{Returns}] \leavevmode
virtual resource info

\item[{Return type}] \leavevmode
{\hyperref[util/nffg:escape.util.nffg.NFFG]{\emph{NFFG}}}

\end{description}\end{quote}

\end{fulllineitems}

\index{\_generate\_resource\_info() (escape.orchest.virtualization\_mgmt.ESCAPEVirtualizer method)}

\begin{fulllineitems}
\phantomsection\label{orchest/virtualization_mgmt:escape.orchest.virtualization_mgmt.ESCAPEVirtualizer._generate_resource_info}\pysiglinewithargsret{\bfcode{\_generate\_resource\_info}}{}{}
Private method to return with resouce info

\end{fulllineitems}


\end{fulllineitems}

\index{MissingGlobalViewEvent (class in escape.orchest.virtualization\_mgmt)}

\begin{fulllineitems}
\phantomsection\label{orchest/virtualization_mgmt:escape.orchest.virtualization_mgmt.MissingGlobalViewEvent}\pysigline{\strong{class }\code{escape.orchest.virtualization\_mgmt.}\bfcode{MissingGlobalViewEvent}}
Bases: \code{pox.lib.revent.revent.Event}

Event for signaling missing global resource view

\end{fulllineitems}

\index{VirtualizerManager (class in escape.orchest.virtualization\_mgmt)}

\begin{fulllineitems}
\phantomsection\label{orchest/virtualization_mgmt:escape.orchest.virtualization_mgmt.VirtualizerManager}\pysigline{\strong{class }\code{escape.orchest.virtualization\_mgmt.}\bfcode{VirtualizerManager}}
Bases: \code{pox.lib.revent.revent.EventMixin}

Store, handle and organize instances of derived classes of
{\hyperref[orchest/virtualization_mgmt:escape.orchest.virtualization_mgmt.AbstractVirtualizer]{\emph{\code{AbstractVirtualizer}}}}
\index{\_eventMixin\_events (escape.orchest.virtualization\_mgmt.VirtualizerManager attribute)}

\begin{fulllineitems}
\phantomsection\label{orchest/virtualization_mgmt:escape.orchest.virtualization_mgmt.VirtualizerManager._eventMixin_events}\pysigline{\bfcode{\_eventMixin\_events}\strong{ = set({[}\textless{}class `escape.orchest.virtualization\_mgmt.MissingGlobalViewEvent'\textgreater{}{]})}}
\end{fulllineitems}

\index{\_\_init\_\_() (escape.orchest.virtualization\_mgmt.VirtualizerManager method)}

\begin{fulllineitems}
\phantomsection\label{orchest/virtualization_mgmt:escape.orchest.virtualization_mgmt.VirtualizerManager.__init__}\pysiglinewithargsret{\bfcode{\_\_init\_\_}}{}{}
Initialize virtualizer manager
\begin{quote}\begin{description}
\item[{Returns}] \leavevmode
None

\end{description}\end{quote}

\end{fulllineitems}

\index{dov (escape.orchest.virtualization\_mgmt.VirtualizerManager attribute)}

\begin{fulllineitems}
\phantomsection\label{orchest/virtualization_mgmt:escape.orchest.virtualization_mgmt.VirtualizerManager.dov}\pysigline{\bfcode{dov}}
Getter method for the \code{DomainVirtualizer}

Request DoV from Adaptation if it hasn't set yet

Use: \emph{virtualizerManager.dov}
\begin{quote}\begin{description}
\item[{Returns}] \leavevmode
Domain Virtualizer (DoV)

\item[{Return type}] \leavevmode
{\hyperref[adapt/adaptation:escape.adapt.adaptation.DomainVirtualizer]{\emph{DomainVirtualizer}}}

\end{description}\end{quote}

\end{fulllineitems}

\index{get\_virtual\_view() (escape.orchest.virtualization\_mgmt.VirtualizerManager method)}

\begin{fulllineitems}
\phantomsection\label{orchest/virtualization_mgmt:escape.orchest.virtualization_mgmt.VirtualizerManager.get_virtual_view}\pysiglinewithargsret{\bfcode{get\_virtual\_view}}{\emph{layer\_id}}{}
Return the virtual view as a derived class of {\hyperref[orchest/virtualization_mgmt:escape.orchest.virtualization_mgmt.AbstractVirtualizer]{\emph{\code{AbstractVirtualizer}}}}
\begin{quote}\begin{description}
\item[{Parameters}] \leavevmode
\textbf{\texttt{layer\_id}} (\href{https://docs.python.org/2.7/library/functions.html\#int}{\emph{int}}) -- layer ID

\item[{Returns}] \leavevmode
virtual view

\item[{Return type}] \leavevmode
{\hyperref[orchest/virtualization_mgmt:escape.orchest.virtualization_mgmt.ESCAPEVirtualizer]{\emph{ESCAPEVirtualizer}}}

\end{description}\end{quote}

\end{fulllineitems}

\index{\_generate\_virtual\_view() (escape.orchest.virtualization\_mgmt.VirtualizerManager method)}

\begin{fulllineitems}
\phantomsection\label{orchest/virtualization_mgmt:escape.orchest.virtualization_mgmt.VirtualizerManager._generate_virtual_view}\pysiglinewithargsret{\bfcode{\_generate\_virtual\_view}}{\emph{layer\_id}}{}
Generate a missing {\hyperref[orchest/virtualization_mgmt:escape.orchest.virtualization_mgmt.ESCAPEVirtualizer]{\emph{\code{ESCAPEVirtualizer}}}} for other layer using global
view (DoV) and a given layer id
\begin{quote}\begin{description}
\item[{Parameters}] \leavevmode
\textbf{\texttt{layer\_id}} (\href{https://docs.python.org/2.7/library/functions.html\#int}{\emph{int}}) -- layer ID

\item[{Returns}] \leavevmode
generated Virtualizer derived from AbstractVirtualizer

\item[{Return type}] \leavevmode
{\hyperref[orchest/virtualization_mgmt:escape.orchest.virtualization_mgmt.ESCAPEVirtualizer]{\emph{ESCAPEVirtualizer}}}

\end{description}\end{quote}

\end{fulllineitems}


\end{fulllineitems}



\subparagraph{\emph{escape.adapt} package}
\label{adapt/adapt:module-escape.adapt}\label{adapt/adapt::doc}\label{adapt/adapt:escape-adapt-package}\index{escape.adapt (module)}
Sublayer for classes related to UNIFY's Controller Adaptation Sublayer (CAS)


\subparagraph{Submodules}
\label{adapt/adapt:submodules}

\subparagraph{\emph{adaptation.py} module}
\label{adapt/adaptation::doc}\label{adapt/adaptation:adaptation-py-module}
{\hyperref[adapt/adaptation:escape.adapt.adaptation.ControllerAdapter]{\emph{\code{ControllerAdapter}}}} implements the centralized functionality of
high-level adaptation and installation of {\hyperref[util/nffg:escape.util.nffg.NFFG]{\emph{\code{NFFG}}}}

{\hyperref[adapt/adaptation:escape.adapt.adaptation.DomainVirtualizer]{\emph{\code{DomainVirtualizer}}}} implement the standard virtualization/generalization
logic of the Resource Orchestration Sublayer

{\hyperref[adapt/adaptation:escape.adapt.adaptation.DomainResourceManager]{\emph{\code{DomainResourceManager}}}} stores and handles the global Virtualizer


\subparagraph{Module contents}
\label{adapt/adaptation:module-escape.adapt.adaptation}\label{adapt/adaptation:module-contents}\index{escape.adapt.adaptation (module)}
Contains classes relevant to the main adaptation function of the Controller
Adaptation Sublayer
\index{DomainConfigurator (class in escape.adapt.adaptation)}

\begin{fulllineitems}
\phantomsection\label{adapt/adaptation:escape.adapt.adaptation.DomainConfigurator}\pysiglinewithargsret{\strong{class }\code{escape.adapt.adaptation.}\bfcode{DomainConfigurator}}{\emph{ca}, \emph{lazy\_load=True}, \emph{remote=True}}{}
Bases: \href{https://docs.python.org/2.7/library/functions.html\#object}{\code{object}}

Initialize, configure and store Domain Manager objects

Use global config to create managers and adapters

Follows Componenet Configurator design pattern
\index{\_\_init\_\_() (escape.adapt.adaptation.DomainConfigurator method)}

\begin{fulllineitems}
\phantomsection\label{adapt/adaptation:escape.adapt.adaptation.DomainConfigurator.__init__}\pysiglinewithargsret{\bfcode{\_\_init\_\_}}{\emph{ca}, \emph{lazy\_load=True}, \emph{remote=True}}{}
For domain adapters the configurator checks the CONFIG first.

\begin{notice}{warning}{Warning:}
Adapter classes must be subclass of AbstractDomainAdapter
\end{notice}

\begin{notice}{note}{Note:}
Arbitrary domain adapters is searched in
{\hyperref[adapt/domain_adapters:module-escape.adapt.domain_adapters]{\emph{\code{escape.adapt.domain\_adapters}}}}
\end{notice}
\begin{quote}\begin{description}
\item[{Parameters}] \leavevmode\begin{itemize}
\item {} 
\textbf{\texttt{ca}} ({\hyperref[adapt/adaptation:escape.adapt.adaptation.ControllerAdapter]{\emph{\code{ControllerAdapter}}}}) -- ControllerAdapter instance

\item {} 
\textbf{\texttt{lazy\_load}} (\href{https://docs.python.org/2.7/library/functions.html\#bool}{\emph{bool}}) -- load adapters only at first reference (default: True)

\item {} 
\textbf{\texttt{remote}} (\href{https://docs.python.org/2.7/library/functions.html\#bool}{\emph{bool}}) -- use NETCONF RPCs or direct access (default: True)

\end{itemize}

\end{description}\end{quote}

\end{fulllineitems}

\index{get() (escape.adapt.adaptation.DomainConfigurator method)}

\begin{fulllineitems}
\phantomsection\label{adapt/adaptation:escape.adapt.adaptation.DomainConfigurator.get}\pysiglinewithargsret{\bfcode{get}}{\emph{domain\_name}}{}
Get Domain maganger with given name.
\begin{quote}\begin{description}
\item[{Parameters}] \leavevmode
\textbf{\texttt{domain\_name}} (\href{https://docs.python.org/2.7/library/functions.html\#str}{\emph{str}}) -- name of domain manager

\item[{Returns}] \leavevmode
None

\end{description}\end{quote}

\end{fulllineitems}

\index{start() (escape.adapt.adaptation.DomainConfigurator method)}

\begin{fulllineitems}
\phantomsection\label{adapt/adaptation:escape.adapt.adaptation.DomainConfigurator.start}\pysiglinewithargsret{\bfcode{start}}{\emph{domain\_name}}{}
Initialize and start a Domain manager.
\begin{quote}\begin{description}
\item[{Parameters}] \leavevmode
\textbf{\texttt{domain\_name}} (\href{https://docs.python.org/2.7/library/functions.html\#str}{\emph{str}}) -- name of domain manager

\item[{Returns}] \leavevmode
None

\end{description}\end{quote}

\end{fulllineitems}

\index{stop() (escape.adapt.adaptation.DomainConfigurator method)}

\begin{fulllineitems}
\phantomsection\label{adapt/adaptation:escape.adapt.adaptation.DomainConfigurator.stop}\pysiglinewithargsret{\bfcode{stop}}{\emph{domain\_name}}{}
Stop and derefer a Domain manager.
\begin{quote}\begin{description}
\item[{Parameters}] \leavevmode
\textbf{\texttt{domain\_name}} (\href{https://docs.python.org/2.7/library/functions.html\#str}{\emph{str}}) -- name of domain manager

\item[{Returns}] \leavevmode
None

\end{description}\end{quote}

\end{fulllineitems}

\index{components (escape.adapt.adaptation.DomainConfigurator attribute)}

\begin{fulllineitems}
\phantomsection\label{adapt/adaptation:escape.adapt.adaptation.DomainConfigurator.components}\pysigline{\bfcode{components}}
Return the dict of initiated Domain managers.
\begin{quote}\begin{description}
\item[{Returns}] \leavevmode
managers

\item[{Return type}] \leavevmode
\href{https://docs.python.org/2.7/library/stdtypes.html\#dict}{dict}

\end{description}\end{quote}

\end{fulllineitems}

\index{\_\_iter\_\_() (escape.adapt.adaptation.DomainConfigurator method)}

\begin{fulllineitems}
\phantomsection\label{adapt/adaptation:escape.adapt.adaptation.DomainConfigurator.__iter__}\pysiglinewithargsret{\bfcode{\_\_iter\_\_}}{}{}
Return with an iterator rely on initiated managers

\end{fulllineitems}

\index{load\_default\_mgrs() (escape.adapt.adaptation.DomainConfigurator method)}

\begin{fulllineitems}
\phantomsection\label{adapt/adaptation:escape.adapt.adaptation.DomainConfigurator.load_default_mgrs}\pysiglinewithargsret{\bfcode{load\_default\_mgrs}}{}{}
Init default adapters.

\end{fulllineitems}

\index{load\_internal\_mgr() (escape.adapt.adaptation.DomainConfigurator method)}

\begin{fulllineitems}
\phantomsection\label{adapt/adaptation:escape.adapt.adaptation.DomainConfigurator.load_internal_mgr}\pysiglinewithargsret{\bfcode{load\_internal\_mgr}}{\emph{remote=True}}{}
Init Domain Manager for internal domain.
\begin{quote}\begin{description}
\item[{Parameters}] \leavevmode
\textbf{\texttt{remote}} (\href{https://docs.python.org/2.7/library/functions.html\#bool}{\emph{bool}}) -- use NETCONF RPCs or direct access (default: True)

\item[{Returns}] \leavevmode
None

\end{description}\end{quote}

\end{fulllineitems}

\index{\_DomainConfigurator\_\_load\_component() (escape.adapt.adaptation.DomainConfigurator method)}

\begin{fulllineitems}
\phantomsection\label{adapt/adaptation:escape.adapt.adaptation.DomainConfigurator._DomainConfigurator__load_component}\pysiglinewithargsret{\bfcode{\_DomainConfigurator\_\_load\_component}}{\emph{component\_name}, \emph{from\_config=True}, \emph{**kwargs}}{}
Load given component from config.
\begin{quote}\begin{description}
\item[{Parameters}] \leavevmode\begin{itemize}
\item {} 
\textbf{\texttt{component\_name}} (\href{https://docs.python.org/2.7/library/functions.html\#str}{\emph{str}}) -- adapter's name

\item {} 
\textbf{\texttt{kwargs}} (\href{https://docs.python.org/2.7/library/stdtypes.html\#dict}{\emph{dict}}) -- adapter's initial parameters

\end{itemize}

\item[{Returns}] \leavevmode
initiated adapter

\item[{Return type}] \leavevmode
{\hyperref[util/adapter:escape.util.adapter.AbstractDomainAdapter]{\emph{\code{AbstractDomainAdapter}}}}

\end{description}\end{quote}

\end{fulllineitems}


\end{fulllineitems}

\index{ControllerAdapter (class in escape.adapt.adaptation)}

\begin{fulllineitems}
\phantomsection\label{adapt/adaptation:escape.adapt.adaptation.ControllerAdapter}\pysiglinewithargsret{\strong{class }\code{escape.adapt.adaptation.}\bfcode{ControllerAdapter}}{\emph{layer\_API}, \emph{with\_infr=False}}{}
Bases: \href{https://docs.python.org/2.7/library/functions.html\#object}{\code{object}}

Higher-level class for {\hyperref[util/nffg:escape.util.nffg.NFFG]{\emph{\code{NFFG}}}} adaptation
between multiple domains
\index{\_\_init\_\_() (escape.adapt.adaptation.ControllerAdapter method)}

\begin{fulllineitems}
\phantomsection\label{adapt/adaptation:escape.adapt.adaptation.ControllerAdapter.__init__}\pysiglinewithargsret{\bfcode{\_\_init\_\_}}{\emph{layer\_API}, \emph{with\_infr=False}}{}
Initialize Controller adapter

For domain adapters the ControllerAdapter checks the CONFIG first
If there is no adapter defined explicitly then initialize the default
Adapter class stored in \emph{\_defaults}

\begin{notice}{warning}{Warning:}
Adapter classes must be subclass of AbstractDomainAdapter
\end{notice}

\begin{notice}{note}{Note:}
Arbitrary domain adapters is searched in
{\hyperref[adapt/domain_adapters:module-escape.adapt.domain_adapters]{\emph{\code{escape.adapt.domain\_adapters}}}}
\end{notice}
\begin{quote}\begin{description}
\item[{Parameters}] \leavevmode\begin{itemize}
\item {} 
\textbf{\texttt{layer\_API}} ({\hyperref[adapt/cas_API:escape.adapt.cas_API.ControllerAdaptationAPI]{\emph{\code{ControllerAdaptationAPI}}}}) -- layer API instance

\item {} 
\textbf{\texttt{with\_infr}} (\href{https://docs.python.org/2.7/library/functions.html\#bool}{\emph{bool}}) -- using emulated infrastructure (default: False)

\end{itemize}

\end{description}\end{quote}

\end{fulllineitems}

\index{install\_nffg() (escape.adapt.adaptation.ControllerAdapter method)}

\begin{fulllineitems}
\phantomsection\label{adapt/adaptation:escape.adapt.adaptation.ControllerAdapter.install_nffg}\pysiglinewithargsret{\bfcode{install\_nffg}}{\emph{mapped\_nffg}}{}
Start NF-FG installation

Process given {\hyperref[util/nffg:escape.util.nffg.NFFG]{\emph{\code{NFFG}}}}, slice information based on domains an invoke
domain adapters to install domain specific parts
\begin{quote}\begin{description}
\item[{Parameters}] \leavevmode
\textbf{\texttt{mapped\_nffg}} ({\hyperref[util/nffg:escape.util.nffg.NFFG]{\emph{\emph{NFFG}}}}) -- mapped NF-FG instance which need to be installed

\item[{Returns}] \leavevmode
None or internal domain NFFG part

\end{description}\end{quote}

\end{fulllineitems}

\index{\_handle\_DomainChangedEvent() (escape.adapt.adaptation.ControllerAdapter method)}

\begin{fulllineitems}
\phantomsection\label{adapt/adaptation:escape.adapt.adaptation.ControllerAdapter._handle_DomainChangedEvent}\pysiglinewithargsret{\bfcode{\_handle\_DomainChangedEvent}}{\emph{event}}{}
Handle DomainChangedEvents, process changes and store relevant information
in DomainResourceManager

\end{fulllineitems}

\index{\_slice\_into\_domains() (escape.adapt.adaptation.ControllerAdapter method)}

\begin{fulllineitems}
\phantomsection\label{adapt/adaptation:escape.adapt.adaptation.ControllerAdapter._slice_into_domains}\pysiglinewithargsret{\bfcode{\_slice\_into\_domains}}{\emph{nffg}}{}
Slice given {\hyperref[util/nffg:escape.util.nffg.NFFG]{\emph{\code{NFFG}}}} into separate parts
\begin{quote}\begin{description}
\item[{Parameters}] \leavevmode
\textbf{\texttt{nffg}} ({\hyperref[util/nffg:escape.util.nffg.NFFG]{\emph{\emph{NFFG}}}}) -- mapped NFFG object

\item[{Returns}] \leavevmode
sliced parts

\item[{Return type}] \leavevmode
\href{https://docs.python.org/2.7/library/stdtypes.html\#dict}{dict}

\end{description}\end{quote}

\end{fulllineitems}


\end{fulllineitems}

\index{DomainVirtualizer (class in escape.adapt.adaptation)}

\begin{fulllineitems}
\phantomsection\label{adapt/adaptation:escape.adapt.adaptation.DomainVirtualizer}\pysiglinewithargsret{\strong{class }\code{escape.adapt.adaptation.}\bfcode{DomainVirtualizer}}{\emph{domainResManager}}{}
Bases: {\hyperref[orchest/virtualization_mgmt:escape.orchest.virtualization_mgmt.AbstractVirtualizer]{\emph{\code{escape.orchest.virtualization\_mgmt.AbstractVirtualizer}}}}

Specific Virtualizer class for global domain virtualization

Implement the same interface as {\hyperref[orchest/virtualization_mgmt:escape.orchest.virtualization_mgmt.AbstractVirtualizer]{\emph{\code{AbstractVirtualizer}}}}
\index{\_\_init\_\_() (escape.adapt.adaptation.DomainVirtualizer method)}

\begin{fulllineitems}
\phantomsection\label{adapt/adaptation:escape.adapt.adaptation.DomainVirtualizer.__init__}\pysiglinewithargsret{\bfcode{\_\_init\_\_}}{\emph{domainResManager}}{}
Init
\begin{quote}\begin{description}
\item[{Parameters}] \leavevmode
\textbf{\texttt{domainResManager}} ({\hyperref[adapt/adaptation:escape.adapt.adaptation.DomainResourceManager]{\emph{\emph{DomainResourceManager}}}}) -- domain resource manager

\item[{Returns}] \leavevmode
None

\end{description}\end{quote}

\end{fulllineitems}

\index{get\_resource\_info() (escape.adapt.adaptation.DomainVirtualizer method)}

\begin{fulllineitems}
\phantomsection\label{adapt/adaptation:escape.adapt.adaptation.DomainVirtualizer.get_resource_info}\pysiglinewithargsret{\bfcode{get\_resource\_info}}{}{}
Return the global resource info represented this class
\begin{quote}\begin{description}
\item[{Returns}] \leavevmode
global resource info

\item[{Return type}] \leavevmode
{\hyperref[util/nffg:escape.util.nffg.NFFG]{\emph{NFFG}}}

\end{description}\end{quote}

\end{fulllineitems}


\end{fulllineitems}

\index{DomainResourceManager (class in escape.adapt.adaptation)}

\begin{fulllineitems}
\phantomsection\label{adapt/adaptation:escape.adapt.adaptation.DomainResourceManager}\pysigline{\strong{class }\code{escape.adapt.adaptation.}\bfcode{DomainResourceManager}}
Bases: \href{https://docs.python.org/2.7/library/functions.html\#object}{\code{object}}

Handle and store global resources
\index{\_\_init\_\_() (escape.adapt.adaptation.DomainResourceManager method)}

\begin{fulllineitems}
\phantomsection\label{adapt/adaptation:escape.adapt.adaptation.DomainResourceManager.__init__}\pysiglinewithargsret{\bfcode{\_\_init\_\_}}{}{}
Init

\end{fulllineitems}

\index{dov (escape.adapt.adaptation.DomainResourceManager attribute)}

\begin{fulllineitems}
\phantomsection\label{adapt/adaptation:escape.adapt.adaptation.DomainResourceManager.dov}\pysigline{\bfcode{dov}}
Getter for {\hyperref[adapt/adaptation:escape.adapt.adaptation.DomainVirtualizer]{\emph{\code{DomainVirtualizer}}}}
\begin{quote}\begin{description}
\item[{Returns}] \leavevmode
Domain Virtualizer

\item[{Return type}] \leavevmode
{\hyperref[orchest/virtualization_mgmt:escape.orchest.virtualization_mgmt.ESCAPEVirtualizer]{\emph{ESCAPEVirtualizer}}}

\end{description}\end{quote}

\end{fulllineitems}

\index{update\_resource\_usage() (escape.adapt.adaptation.DomainResourceManager method)}

\begin{fulllineitems}
\phantomsection\label{adapt/adaptation:escape.adapt.adaptation.DomainResourceManager.update_resource_usage}\pysiglinewithargsret{\bfcode{update\_resource\_usage}}{\emph{data}}{}
Update global resource database with resource usage relevant to installed
components, routes, VNFs, etc.
\begin{quote}\begin{description}
\item[{Parameters}] \leavevmode
\textbf{\texttt{data}} (\href{https://docs.python.org/2.7/library/stdtypes.html\#dict}{\emph{dict}}) -- usage data

\item[{Returns}] \leavevmode
None

\end{description}\end{quote}

\end{fulllineitems}


\end{fulllineitems}



\subparagraph{\emph{cas\_API.py} module}
\label{adapt/cas_API:cas-api-py-module}\label{adapt/cas_API::doc}
{\hyperref[adapt/cas_API:escape.adapt.cas_API.GlobalResInfoEvent]{\emph{\code{GlobalResInfoEvent}}}} can send back global resource info requested from
upper layer

{\hyperref[adapt/cas_API:escape.adapt.cas_API.ControllerAdaptationAPI]{\emph{\code{ControllerAdaptationAPI}}}} represents the CAS layer and implement all
related functionality


\subparagraph{Module contents}
\label{adapt/cas_API:module-escape.adapt.cas_API}\label{adapt/cas_API:module-contents}\index{escape.adapt.cas\_API (module)}
Implements the platform and POX dependent logic for the Controller Adaptation
Sublayer
\index{GlobalResInfoEvent (class in escape.adapt.cas\_API)}

\begin{fulllineitems}
\phantomsection\label{adapt/cas_API:escape.adapt.cas_API.GlobalResInfoEvent}\pysiglinewithargsret{\strong{class }\code{escape.adapt.cas\_API.}\bfcode{GlobalResInfoEvent}}{\emph{resource\_info}}{}
Bases: \code{pox.lib.revent.revent.Event}

Event for sending back requested global resource info
\index{\_\_init\_\_() (escape.adapt.cas\_API.GlobalResInfoEvent method)}

\begin{fulllineitems}
\phantomsection\label{adapt/cas_API:escape.adapt.cas_API.GlobalResInfoEvent.__init__}\pysiglinewithargsret{\bfcode{\_\_init\_\_}}{\emph{resource\_info}}{}
Init
\begin{quote}\begin{description}
\item[{Parameters}] \leavevmode
\textbf{\texttt{resource\_info}} ({\hyperref[orchest/virtualization_mgmt:escape.orchest.virtualization_mgmt.ESCAPEVirtualizer]{\emph{\code{ESCAPEVirtualizer}}}}) -- resource info

\end{description}\end{quote}

\end{fulllineitems}


\end{fulllineitems}

\index{InstallationFinishedEvent (class in escape.adapt.cas\_API)}

\begin{fulllineitems}
\phantomsection\label{adapt/cas_API:escape.adapt.cas_API.InstallationFinishedEvent}\pysiglinewithargsret{\strong{class }\code{escape.adapt.cas\_API.}\bfcode{InstallationFinishedEvent}}{\emph{success}, \emph{error=None}}{}
Bases: \code{pox.lib.revent.revent.Event}

Event for signalling end of mapping process
\index{\_\_init\_\_() (escape.adapt.cas\_API.InstallationFinishedEvent method)}

\begin{fulllineitems}
\phantomsection\label{adapt/cas_API:escape.adapt.cas_API.InstallationFinishedEvent.__init__}\pysiglinewithargsret{\bfcode{\_\_init\_\_}}{\emph{success}, \emph{error=None}}{}
\end{fulllineitems}


\end{fulllineitems}

\index{DeployNFFGEvent (class in escape.adapt.cas\_API)}

\begin{fulllineitems}
\phantomsection\label{adapt/cas_API:escape.adapt.cas_API.DeployNFFGEvent}\pysiglinewithargsret{\strong{class }\code{escape.adapt.cas\_API.}\bfcode{DeployNFFGEvent}}{\emph{nffg\_part}}{}
Bases: \code{pox.lib.revent.revent.Event}

Event for passing mapped {\hyperref[util/nffg:escape.util.nffg.NFFG]{\emph{\code{NFFG}}}} to internally emulated network based on
Mininet for testing
\index{\_\_init\_\_() (escape.adapt.cas\_API.DeployNFFGEvent method)}

\begin{fulllineitems}
\phantomsection\label{adapt/cas_API:escape.adapt.cas_API.DeployNFFGEvent.__init__}\pysiglinewithargsret{\bfcode{\_\_init\_\_}}{\emph{nffg\_part}}{}
\end{fulllineitems}


\end{fulllineitems}

\index{ControllerAdaptationAPI (class in escape.adapt.cas\_API)}

\begin{fulllineitems}
\phantomsection\label{adapt/cas_API:escape.adapt.cas_API.ControllerAdaptationAPI}\pysiglinewithargsret{\strong{class }\code{escape.adapt.cas\_API.}\bfcode{ControllerAdaptationAPI}}{\emph{standalone=False}, \emph{**kwargs}}{}
Bases: {\hyperref[util/api:escape.util.api.AbstractAPI]{\emph{\code{escape.util.api.AbstractAPI}}}}

Entry point for Controller Adaptation Sublayer (CAS)

Maintain the contact with other UNIFY layers

Implement the Or - Ca reference point
\index{\_core\_name (escape.adapt.cas\_API.ControllerAdaptationAPI attribute)}

\begin{fulllineitems}
\phantomsection\label{adapt/cas_API:escape.adapt.cas_API.ControllerAdaptationAPI._core_name}\pysigline{\bfcode{\_core\_name}\strong{ = `adaptation'}}
\end{fulllineitems}

\index{\_\_init\_\_() (escape.adapt.cas\_API.ControllerAdaptationAPI method)}

\begin{fulllineitems}
\phantomsection\label{adapt/cas_API:escape.adapt.cas_API.ControllerAdaptationAPI.__init__}\pysiglinewithargsret{\bfcode{\_\_init\_\_}}{\emph{standalone=False}, \emph{**kwargs}}{}~

\strong{See also:}


{\hyperref[util/api:escape.util.api.AbstractAPI.__init__]{\emph{\code{AbstractAPI.\_\_init\_\_()}}}}



\end{fulllineitems}

\index{initialize() (escape.adapt.cas\_API.ControllerAdaptationAPI method)}

\begin{fulllineitems}
\phantomsection\label{adapt/cas_API:escape.adapt.cas_API.ControllerAdaptationAPI.initialize}\pysiglinewithargsret{\bfcode{initialize}}{}{}~

\strong{See also:}


{\hyperref[util/api:escape.util.api.AbstractAPI.initialize]{\emph{\code{AbstractAPI.initialze()}}}}



\end{fulllineitems}

\index{shutdown() (escape.adapt.cas\_API.ControllerAdaptationAPI method)}

\begin{fulllineitems}
\phantomsection\label{adapt/cas_API:escape.adapt.cas_API.ControllerAdaptationAPI.shutdown}\pysiglinewithargsret{\bfcode{shutdown}}{\emph{event}}{}~

\strong{See also:}


{\hyperref[util/api:escape.util.api.AbstractAPI.shutdown]{\emph{\code{AbstractAPI.shutdown()}}}}



\end{fulllineitems}

\index{\_handle\_InstallNFFGEvent() (escape.adapt.cas\_API.ControllerAdaptationAPI method)}

\begin{fulllineitems}
\phantomsection\label{adapt/cas_API:escape.adapt.cas_API.ControllerAdaptationAPI._handle_InstallNFFGEvent}\pysiglinewithargsret{\bfcode{\_handle\_InstallNFFGEvent}}{\emph{*args}, \emph{**kwargs}}{}
Install mapped NF-FG (UNIFY Or - Ca API)
\begin{quote}\begin{description}
\item[{Parameters}] \leavevmode
\textbf{\texttt{event}} ({\hyperref[orchest/ros_API:escape.orchest.ros_API.InstallNFFGEvent]{\emph{\code{InstallNFFGEvent}}}}) -- event object contains mapped NF-FG

\item[{Returns}] \leavevmode
None

\end{description}\end{quote}

\end{fulllineitems}

\index{\_handle\_GetGlobalResInfoEvent() (escape.adapt.cas\_API.ControllerAdaptationAPI method)}

\begin{fulllineitems}
\phantomsection\label{adapt/cas_API:escape.adapt.cas_API.ControllerAdaptationAPI._handle_GetGlobalResInfoEvent}\pysiglinewithargsret{\bfcode{\_handle\_GetGlobalResInfoEvent}}{\emph{event}}{}
Generate global resource info and send back to ROS
\begin{quote}\begin{description}
\item[{Parameters}] \leavevmode
\textbf{\texttt{event}} ({\hyperref[orchest/ros_API:escape.orchest.ros_API.GetGlobalResInfoEvent]{\emph{\code{GetGlobalResInfoEvent}}}}) -- event object

\item[{Returns}] \leavevmode
None

\end{description}\end{quote}

\end{fulllineitems}

\index{\_handle\_DeployEvent() (escape.adapt.cas\_API.ControllerAdaptationAPI method)}

\begin{fulllineitems}
\phantomsection\label{adapt/cas_API:escape.adapt.cas_API.ControllerAdaptationAPI._handle_DeployEvent}\pysiglinewithargsret{\bfcode{\_handle\_DeployEvent}}{\emph{event}}{}
Receive processed NF-FG from domain adapter(s) and forward to Infrastructure
\begin{quote}\begin{description}
\item[{Parameters}] \leavevmode
\textbf{\texttt{event}} ({\hyperref[adapt/cas_API:escape.adapt.cas_API.DeployNFFGEvent]{\emph{\code{DeployNFFGEvent}}}}) -- event object

\item[{Returns}] \leavevmode
None

\end{description}\end{quote}

\end{fulllineitems}

\index{\_handle\_DeploymentFinishedEvent() (escape.adapt.cas\_API.ControllerAdaptationAPI method)}

\begin{fulllineitems}
\phantomsection\label{adapt/cas_API:escape.adapt.cas_API.ControllerAdaptationAPI._handle_DeploymentFinishedEvent}\pysiglinewithargsret{\bfcode{\_handle\_DeploymentFinishedEvent}}{\emph{event}}{}
Receive successfull NF-FG deployment event and propagate upwards
\begin{quote}\begin{description}
\item[{Parameters}] \leavevmode
\textbf{\texttt{event}} ({\hyperref[infr/il_API:escape.infr.il_API.DeploymentFinishedEvent]{\emph{\code{DeploymentFinishedEvent}}}}) -- event object

\item[{Returns}] \leavevmode
None

\end{description}\end{quote}

\end{fulllineitems}


\end{fulllineitems}



\subparagraph{\emph{domain\_adapters.py} module}
\label{adapt/domain_adapters:domain-adapters-py-module}\label{adapt/domain_adapters::doc}
{\hyperref[adapt/domain_adapters:escape.adapt.domain_adapters.POXDomainAdapter]{\emph{\code{POXDomainAdapter}}}} implements POX related functionality.

{\hyperref[adapt/domain_adapters:escape.adapt.domain_adapters.MininetDomainAdapter]{\emph{\code{MininetDomainAdapter}}}} implements Mininet related functionality
transparently.

{\hyperref[adapt/domain_adapters:escape.adapt.domain_adapters.VNFStarterAdapter]{\emph{\code{VNFStarterAdapter}}}} is a wrapper class for vnf\_starter NETCONF module.

{\hyperref[adapt/domain_adapters:escape.adapt.domain_adapters.OpenStackRESTAdapter]{\emph{\code{OpenStackRESTAdapter}}}} is a wrapper class for OpenStack-related functions.

{\hyperref[adapt/domain_adapters:escape.adapt.domain_adapters.InternalDomainManager]{\emph{\code{InternalDomainManager}}}} represent the top class for interacting with
emulated infrastructure.

{\hyperref[adapt/domain_adapters:escape.adapt.domain_adapters.OpenStackDomainManager]{\emph{\code{OpenStackDomainManager}}}} implements OpenStack related functionality.

{\hyperref[adapt/domain_adapters:escape.adapt.domain_adapters.DockerDomainManager]{\emph{\code{DockerDomainManager}}}} implements Docker related functionality.


\subparagraph{Module contents}
\label{adapt/domain_adapters:module-escape.adapt.domain_adapters}\label{adapt/domain_adapters:module-contents}\index{escape.adapt.domain\_adapters (module)}
Contains Adapter classes which represent the connections between ESCAPEv2 and
other different domains
\index{POXDomainAdapter (class in escape.adapt.domain\_adapters)}

\begin{fulllineitems}
\phantomsection\label{adapt/domain_adapters:escape.adapt.domain_adapters.POXDomainAdapter}\pysiglinewithargsret{\strong{class }\code{escape.adapt.domain\_adapters.}\bfcode{POXDomainAdapter}}{\emph{name=None}, \emph{address=`127.0.0.1'}, \emph{port=6653}}{}
Bases: {\hyperref[util/adapter:escape.util.adapter.AbstractDomainAdapter]{\emph{\code{escape.util.adapter.AbstractDomainAdapter}}}}

Adapter class to handle communication with internal POX OpenFlow controller

Can be used to define a controller (based on POX) for other external domains
\index{name (escape.adapt.domain\_adapters.POXDomainAdapter attribute)}

\begin{fulllineitems}
\phantomsection\label{adapt/domain_adapters:escape.adapt.domain_adapters.POXDomainAdapter.name}\pysigline{\bfcode{name}\strong{ = `POX'}}
\end{fulllineitems}

\index{\_\_init\_\_() (escape.adapt.domain\_adapters.POXDomainAdapter method)}

\begin{fulllineitems}
\phantomsection\label{adapt/domain_adapters:escape.adapt.domain_adapters.POXDomainAdapter.__init__}\pysiglinewithargsret{\bfcode{\_\_init\_\_}}{\emph{name=None}, \emph{address=`127.0.0.1'}, \emph{port=6653}}{}
Initialize attributes, register specific connection Arbiter if needed and
set up listening of OpenFlow events.
\begin{quote}\begin{description}
\item[{Parameters}] \leavevmode\begin{itemize}
\item {} 
\textbf{\texttt{name}} (\href{https://docs.python.org/2.7/library/functions.html\#str}{\emph{str}}) -- name used to register component ito \code{pox.core}

\item {} 
\textbf{\texttt{address}} (\href{https://docs.python.org/2.7/library/functions.html\#str}{\emph{str}}) -- socket address (default: 127.0.0.1)

\item {} 
\textbf{\texttt{port}} (\href{https://docs.python.org/2.7/library/functions.html\#int}{\emph{int}}) -- socket port (default: 6653)

\end{itemize}

\end{description}\end{quote}

\end{fulllineitems}

\index{filter\_connections() (escape.adapt.domain\_adapters.POXDomainAdapter method)}

\begin{fulllineitems}
\phantomsection\label{adapt/domain_adapters:escape.adapt.domain_adapters.POXDomainAdapter.filter_connections}\pysiglinewithargsret{\bfcode{filter\_connections}}{\emph{event}}{}
Handle which connection should be handled by this Adapter class.

This adapter accept every OpenFlow connection by default.
\begin{quote}\begin{description}
\item[{Parameters}] \leavevmode
\textbf{\texttt{event}} (\code{pox.openflow.ConnectionUp}) -- POX internal ConnectionUp event (event.dpid, event.connection)

\item[{Returns}] \leavevmode
True os False obviously

\item[{Return type}] \leavevmode
\href{https://docs.python.org/2.7/library/functions.html\#bool}{bool}

\end{description}\end{quote}

\end{fulllineitems}

\index{\_handle\_ConnectionUp() (escape.adapt.domain\_adapters.POXDomainAdapter method)}

\begin{fulllineitems}
\phantomsection\label{adapt/domain_adapters:escape.adapt.domain_adapters.POXDomainAdapter._handle_ConnectionUp}\pysiglinewithargsret{\bfcode{\_handle\_ConnectionUp}}{\emph{event}}{}
Handle incoming OpenFlow connections

\end{fulllineitems}

\index{\_handle\_ConnectionDown() (escape.adapt.domain\_adapters.POXDomainAdapter method)}

\begin{fulllineitems}
\phantomsection\label{adapt/domain_adapters:escape.adapt.domain_adapters.POXDomainAdapter._handle_ConnectionDown}\pysiglinewithargsret{\bfcode{\_handle\_ConnectionDown}}{\emph{event}}{}
Handle disconnected device

\end{fulllineitems}

\index{install\_routes() (escape.adapt.domain\_adapters.POXDomainAdapter method)}

\begin{fulllineitems}
\phantomsection\label{adapt/domain_adapters:escape.adapt.domain_adapters.POXDomainAdapter.install_routes}\pysiglinewithargsret{\bfcode{install\_routes}}{\emph{routes}}{}
Install routes related to the managed domain. Translates the generic
format of the routes into OpenFlow flow rules.

Routes are computed by the ControllerAdapter's main adaptation algorithm
\begin{quote}\begin{description}
\item[{Parameters}] \leavevmode
\textbf{\texttt{routes}} ({\hyperref[util/nffg:escape.util.nffg.NFFG]{\emph{\code{NFFG}}}}) -- list of routes

\item[{Returns}] \leavevmode
None

\end{description}\end{quote}

\end{fulllineitems}


\end{fulllineitems}

\index{MininetDomainAdapter (class in escape.adapt.domain\_adapters)}

\begin{fulllineitems}
\phantomsection\label{adapt/domain_adapters:escape.adapt.domain_adapters.MininetDomainAdapter}\pysiglinewithargsret{\strong{class }\code{escape.adapt.domain\_adapters.}\bfcode{MininetDomainAdapter}}{\emph{mininet=None}}{}
Bases: {\hyperref[util/adapter:escape.util.adapter.AbstractDomainAdapter]{\emph{\code{escape.util.adapter.AbstractDomainAdapter}}}}, {\hyperref[util/adapter:escape.util.adapter.VNFStarterAPI]{\emph{\code{escape.util.adapter.VNFStarterAPI}}}}

Adapter class to handle communication with Mininet domain

Implement VNF managing API using direct access to the
\code{mininet.net.Mininet} object
\index{\_eventMixin\_events (escape.adapt.domain\_adapters.MininetDomainAdapter attribute)}

\begin{fulllineitems}
\phantomsection\label{adapt/domain_adapters:escape.adapt.domain_adapters.MininetDomainAdapter._eventMixin_events}\pysigline{\bfcode{\_eventMixin\_events}\strong{ = set({[}\textless{}class `escape.util.adapter.DeployEvent'\textgreater{}, \textless{}class `escape.util.adapter.DomainChangedEvent'\textgreater{}{]})}}
\end{fulllineitems}

\index{name (escape.adapt.domain\_adapters.MininetDomainAdapter attribute)}

\begin{fulllineitems}
\phantomsection\label{adapt/domain_adapters:escape.adapt.domain_adapters.MininetDomainAdapter.name}\pysigline{\bfcode{name}\strong{ = `MININET'}}
\end{fulllineitems}

\index{\_\_init\_\_() (escape.adapt.domain\_adapters.MininetDomainAdapter method)}

\begin{fulllineitems}
\phantomsection\label{adapt/domain_adapters:escape.adapt.domain_adapters.MininetDomainAdapter.__init__}\pysiglinewithargsret{\bfcode{\_\_init\_\_}}{\emph{mininet=None}}{}
Init
\begin{quote}\begin{description}
\item[{Parameters}] \leavevmode
\textbf{\texttt{mininet}} (\emph{:any{}`mininet.net.Mininet{}`}) -- set pre-defined network (optional)

\end{description}\end{quote}

\end{fulllineitems}

\index{initiate\_VNFs() (escape.adapt.domain\_adapters.MininetDomainAdapter method)}

\begin{fulllineitems}
\phantomsection\label{adapt/domain_adapters:escape.adapt.domain_adapters.MininetDomainAdapter.initiate_VNFs}\pysiglinewithargsret{\bfcode{initiate\_VNFs}}{\emph{nffg\_part}}{}
\end{fulllineitems}

\index{stopVNF() (escape.adapt.domain\_adapters.MininetDomainAdapter method)}

\begin{fulllineitems}
\phantomsection\label{adapt/domain_adapters:escape.adapt.domain_adapters.MininetDomainAdapter.stopVNF}\pysiglinewithargsret{\bfcode{stopVNF}}{\emph{vnf\_id}}{}
\end{fulllineitems}

\index{getVNFInfo() (escape.adapt.domain\_adapters.MininetDomainAdapter method)}

\begin{fulllineitems}
\phantomsection\label{adapt/domain_adapters:escape.adapt.domain_adapters.MininetDomainAdapter.getVNFInfo}\pysiglinewithargsret{\bfcode{getVNFInfo}}{\emph{vnf\_id=None}}{}
\end{fulllineitems}

\index{disconnectVNF() (escape.adapt.domain\_adapters.MininetDomainAdapter method)}

\begin{fulllineitems}
\phantomsection\label{adapt/domain_adapters:escape.adapt.domain_adapters.MininetDomainAdapter.disconnectVNF}\pysiglinewithargsret{\bfcode{disconnectVNF}}{\emph{vnf\_id}, \emph{vnf\_port}}{}
\end{fulllineitems}

\index{startVNF() (escape.adapt.domain\_adapters.MininetDomainAdapter method)}

\begin{fulllineitems}
\phantomsection\label{adapt/domain_adapters:escape.adapt.domain_adapters.MininetDomainAdapter.startVNF}\pysiglinewithargsret{\bfcode{startVNF}}{\emph{vnf\_id}}{}
\end{fulllineitems}

\index{connectVNF() (escape.adapt.domain\_adapters.MininetDomainAdapter method)}

\begin{fulllineitems}
\phantomsection\label{adapt/domain_adapters:escape.adapt.domain_adapters.MininetDomainAdapter.connectVNF}\pysiglinewithargsret{\bfcode{connectVNF}}{\emph{vnf\_id}, \emph{vnf\_port}, \emph{switch\_id}}{}
\end{fulllineitems}

\index{initiateVNF() (escape.adapt.domain\_adapters.MininetDomainAdapter method)}

\begin{fulllineitems}
\phantomsection\label{adapt/domain_adapters:escape.adapt.domain_adapters.MininetDomainAdapter.initiateVNF}\pysiglinewithargsret{\bfcode{initiateVNF}}{\emph{vnf\_type=None}, \emph{vnf\_description=None}, \emph{options=None}}{}
\end{fulllineitems}


\end{fulllineitems}

\index{VNFStarterAdapter (class in escape.adapt.domain\_adapters)}

\begin{fulllineitems}
\phantomsection\label{adapt/domain_adapters:escape.adapt.domain_adapters.VNFStarterAdapter}\pysiglinewithargsret{\strong{class }\code{escape.adapt.domain\_adapters.}\bfcode{VNFStarterAdapter}}{\emph{**kwargs}}{}
Bases: {\hyperref[util/netconf:escape.util.netconf.AbstractNETCONFAdapter]{\emph{\code{escape.util.netconf.AbstractNETCONFAdapter}}}}, {\hyperref[util/adapter:escape.util.adapter.AbstractDomainAdapter]{\emph{\code{escape.util.adapter.AbstractDomainAdapter}}}}, {\hyperref[util/adapter:escape.util.adapter.VNFStarterAPI]{\emph{\code{escape.util.adapter.VNFStarterAPI}}}}

This class is devoted to provide NETCONF specific functions for vnf\_starter
module. Documentation is transferred from vnf\_starter.yang

This class is devoted to start and stop CLICK-based VNFs that will be
connected to a mininet switch.

Follows the MixIn design patteran approach to support NETCONF functionality
\index{RPC\_NAMESPACE (escape.adapt.domain\_adapters.VNFStarterAdapter attribute)}

\begin{fulllineitems}
\phantomsection\label{adapt/domain_adapters:escape.adapt.domain_adapters.VNFStarterAdapter.RPC_NAMESPACE}\pysigline{\bfcode{RPC\_NAMESPACE}\strong{ = u'http://csikor.tmit.bme.hu/netconf/unify/vnf\_starter'}}
\end{fulllineitems}

\index{name (escape.adapt.domain\_adapters.VNFStarterAdapter attribute)}

\begin{fulllineitems}
\phantomsection\label{adapt/domain_adapters:escape.adapt.domain_adapters.VNFStarterAdapter.name}\pysigline{\bfcode{name}\strong{ = `VNFStarter'}}
\end{fulllineitems}

\index{\_\_init\_\_() (escape.adapt.domain\_adapters.VNFStarterAdapter method)}

\begin{fulllineitems}
\phantomsection\label{adapt/domain_adapters:escape.adapt.domain_adapters.VNFStarterAdapter.__init__}\pysiglinewithargsret{\bfcode{\_\_init\_\_}}{\emph{**kwargs}}{}
\end{fulllineitems}

\index{initiateVNF() (escape.adapt.domain\_adapters.VNFStarterAdapter method)}

\begin{fulllineitems}
\phantomsection\label{adapt/domain_adapters:escape.adapt.domain_adapters.VNFStarterAdapter.initiateVNF}\pysiglinewithargsret{\bfcode{initiateVNF}}{\emph{vnf\_type=None}, \emph{vnf\_description=None}, \emph{options=None}}{}
This RCP will start a VNF.
\begin{enumerate}
\setcounter{enumi}{-1}
\item {} 
initiate new VNF (initiate datastructure, generate unique ID)

\item {} 
set its arguments (control port, control ip, and VNF type/command)

\item {} 
returns the connection data, which from the vnf\_id is the most important

\end{enumerate}
\begin{quote}\begin{description}
\item[{Parameters}] \leavevmode\begin{itemize}
\item {} 
\textbf{\texttt{vnf\_type}} (\href{https://docs.python.org/2.7/library/functions.html\#str}{\emph{str}}) -- pre-defined VNF type (see in vnf\_starter/available\_vnfs)

\item {} 
\textbf{\texttt{vnf\_description}} (\href{https://docs.python.org/2.7/library/functions.html\#str}{\emph{str}}) -- Click description if there are no pre-defined type

\item {} 
\textbf{\texttt{options}} (\href{https://docs.python.org/2.7/library/collections.html\#collections.OrderedDict}{\emph{collections.OrderedDict}}) -- unlimited list of additional options as name-value pairs

\end{itemize}

\item[{Returns}] \leavevmode
RPC reply data

\item[{Raises}] \leavevmode
RPCError, OperationError, TransportError

\end{description}\end{quote}

\end{fulllineitems}

\index{connectVNF() (escape.adapt.domain\_adapters.VNFStarterAdapter method)}

\begin{fulllineitems}
\phantomsection\label{adapt/domain_adapters:escape.adapt.domain_adapters.VNFStarterAdapter.connectVNF}\pysiglinewithargsret{\bfcode{connectVNF}}{\emph{vnf\_id}, \emph{vnf\_port}, \emph{switch\_id}}{}
This RPC will practically start and connect the initiated VNF/CLICK to
the switch.
\begin{enumerate}
\setcounter{enumi}{-1}
\item {} 
create virtualEthernet pair(s)

\item {} 
connect either end of it (them) to the given switch(es)

\end{enumerate}

This RPC is also used for reconnecting a VNF. In this case, however,
if the input fields are not correctly set an error occurs
\begin{quote}\begin{description}
\item[{Parameters}] \leavevmode\begin{itemize}
\item {} 
\textbf{\texttt{vnf\_id}} (\href{https://docs.python.org/2.7/library/functions.html\#str}{\emph{str}}) -- VNF ID (mandatory)

\item {} 
\textbf{\texttt{vnf\_port}} (\href{https://docs.python.org/2.7/library/functions.html\#str}{\emph{str}}) -- VNF port (mandatory)

\item {} 
\textbf{\texttt{switch\_id}} (\href{https://docs.python.org/2.7/library/functions.html\#str}{\emph{str}}) -- switch ID (mandatory)

\end{itemize}

\item[{Returns}] \leavevmode
Returns the connected port(s) with the corresponding switch(es).

\item[{Raises}] \leavevmode
RPCError, OperationError, TransportError

\end{description}\end{quote}

\end{fulllineitems}

\index{disconnectVNF() (escape.adapt.domain\_adapters.VNFStarterAdapter method)}

\begin{fulllineitems}
\phantomsection\label{adapt/domain_adapters:escape.adapt.domain_adapters.VNFStarterAdapter.disconnectVNF}\pysiglinewithargsret{\bfcode{disconnectVNF}}{\emph{vnf\_id}, \emph{vnf\_port}}{}
This RPC will disconnect the VNF(s)/CLICK(s) from the switch(es).
\begin{enumerate}
\setcounter{enumi}{-1}
\item {} 
ip link set uny\_0 down

\item {} 
ip link set uny\_1 down

\item {} 
(if more ports) repeat 1. and 2. with the corresponding data

\end{enumerate}
\begin{quote}\begin{description}
\item[{Parameters}] \leavevmode\begin{itemize}
\item {} 
\textbf{\texttt{vnf\_id}} (\href{https://docs.python.org/2.7/library/functions.html\#str}{\emph{str}}) -- VNF ID (mandatory)

\item {} 
\textbf{\texttt{vnf\_port}} (\href{https://docs.python.org/2.7/library/functions.html\#str}{\emph{str}}) -- VNF port (mandatory)

\end{itemize}

\item[{Returns}] \leavevmode
reply data

\item[{Raises}] \leavevmode
RPCError, OperationError, TransportError

\end{description}\end{quote}

\end{fulllineitems}

\index{startVNF() (escape.adapt.domain\_adapters.VNFStarterAdapter method)}

\begin{fulllineitems}
\phantomsection\label{adapt/domain_adapters:escape.adapt.domain_adapters.VNFStarterAdapter.startVNF}\pysiglinewithargsret{\bfcode{startVNF}}{\emph{vnf\_id}}{}
This RPC will actually start the VNF/CLICK instance.
\begin{quote}\begin{description}
\item[{Parameters}] \leavevmode
\textbf{\texttt{vnf\_id}} (\href{https://docs.python.org/2.7/library/functions.html\#str}{\emph{str}}) -- VNF ID (mandatory)

\item[{Returns}] \leavevmode
reply data

\item[{Raises}] \leavevmode
RPCError, OperationError, TransportError

\end{description}\end{quote}

\end{fulllineitems}

\index{stopVNF() (escape.adapt.domain\_adapters.VNFStarterAdapter method)}

\begin{fulllineitems}
\phantomsection\label{adapt/domain_adapters:escape.adapt.domain_adapters.VNFStarterAdapter.stopVNF}\pysiglinewithargsret{\bfcode{stopVNF}}{\emph{vnf\_id}}{}
This RPC will gracefully shut down the VNF/CLICK instance.
\begin{enumerate}
\setcounter{enumi}{-1}
\item {} 
if disconnect() was not called before, we call it

\item {} 
delete virtual ethernet pairs

\item {} 
stop (kill) click

\item {} 
remove vnf's data from the data structure

\end{enumerate}
\begin{quote}\begin{description}
\item[{Parameters}] \leavevmode
\textbf{\texttt{vnf\_id}} (\href{https://docs.python.org/2.7/library/functions.html\#str}{\emph{str}}) -- VNF ID (mandatory)

\item[{Returns}] \leavevmode
reply data

\item[{Raises}] \leavevmode
RPCError, OperationError, TransportError

\end{description}\end{quote}

\end{fulllineitems}

\index{getVNFInfo() (escape.adapt.domain\_adapters.VNFStarterAdapter method)}

\begin{fulllineitems}
\phantomsection\label{adapt/domain_adapters:escape.adapt.domain_adapters.VNFStarterAdapter.getVNFInfo}\pysiglinewithargsret{\bfcode{getVNFInfo}}{\emph{vnf\_id=None}}{}
This RPC will send back all data of all VNFs that have been initiated by
this NETCONF Agent. If an input of vnf\_id is set, only that VNF's data
will be sent back. Most of the data this RPC replies is used for DEBUG,
however `status' is useful for indicating to upper layers whether a VNF
is UP\_AND\_RUNNING
\begin{quote}\begin{description}
\item[{Parameters}] \leavevmode
\textbf{\texttt{vnf\_id}} (\href{https://docs.python.org/2.7/library/functions.html\#str}{\emph{str}}) -- VNF ID

\item[{Returns}] \leavevmode
reply data

\item[{Raises}] \leavevmode
RPCError, OperationError, TransportError

\end{description}\end{quote}

\end{fulllineitems}


\end{fulllineitems}

\index{OpenStackRESTAdapter (class in escape.adapt.domain\_adapters)}

\begin{fulllineitems}
\phantomsection\label{adapt/domain_adapters:escape.adapt.domain_adapters.OpenStackRESTAdapter}\pysiglinewithargsret{\strong{class }\code{escape.adapt.domain\_adapters.}\bfcode{OpenStackRESTAdapter}}{\emph{url}}{}
Bases: {\hyperref[util/adapter:escape.util.adapter.AbstractRESTAdapter]{\emph{\code{escape.util.adapter.AbstractRESTAdapter}}}}, {\hyperref[util/adapter:escape.util.adapter.AbstractDomainAdapter]{\emph{\code{escape.util.adapter.AbstractDomainAdapter}}}}, {\hyperref[util/adapter:escape.util.adapter.OpenStackAPI]{\emph{\code{escape.util.adapter.OpenStackAPI}}}}
\index{\_\_init\_\_() (escape.adapt.domain\_adapters.OpenStackRESTAdapter method)}

\begin{fulllineitems}
\phantomsection\label{adapt/domain_adapters:escape.adapt.domain_adapters.OpenStackRESTAdapter.__init__}\pysiglinewithargsret{\bfcode{\_\_init\_\_}}{\emph{url}}{}
Init
\begin{quote}\begin{description}
\item[{Parameters}] \leavevmode
\textbf{\texttt{url}} (\href{https://docs.python.org/2.7/library/functions.html\#str}{\emph{str}}) -- OpenStack RESTful API URL

\end{description}\end{quote}

\end{fulllineitems}


\end{fulllineitems}

\index{InternalDomainManager (class in escape.adapt.domain\_adapters)}

\begin{fulllineitems}
\phantomsection\label{adapt/domain_adapters:escape.adapt.domain_adapters.InternalDomainManager}\pysiglinewithargsret{\strong{class }\code{escape.adapt.domain\_adapters.}\bfcode{InternalDomainManager}}{\emph{controller=None}, \emph{network=None}, \emph{remote=None}}{}
Bases: {\hyperref[util/adapter:escape.util.adapter.AbstractDomainManager]{\emph{\code{escape.util.adapter.AbstractDomainManager}}}}

Manager class to handle communication with internally emulated network

\begin{notice}{note}{Note:}
Uses {\hyperref[adapt/domain_adapters:escape.adapt.domain_adapters.MininetDomainAdapter]{\emph{\code{MininetDomainAdapter}}}} for managing the emulated network and
{\hyperref[adapt/domain_adapters:escape.adapt.domain_adapters.POXDomainAdapter]{\emph{\code{POXDomainAdapter}}}} for controlling the network
\end{notice}
\index{name (escape.adapt.domain\_adapters.InternalDomainManager attribute)}

\begin{fulllineitems}
\phantomsection\label{adapt/domain_adapters:escape.adapt.domain_adapters.InternalDomainManager.name}\pysigline{\bfcode{name}\strong{ = `INTERNAL'}}
\end{fulllineitems}

\index{\_\_init\_\_() (escape.adapt.domain\_adapters.InternalDomainManager method)}

\begin{fulllineitems}
\phantomsection\label{adapt/domain_adapters:escape.adapt.domain_adapters.InternalDomainManager.__init__}\pysiglinewithargsret{\bfcode{\_\_init\_\_}}{\emph{controller=None}, \emph{network=None}, \emph{remote=None}}{}
Init

\end{fulllineitems}

\index{controller\_name (escape.adapt.domain\_adapters.InternalDomainManager attribute)}

\begin{fulllineitems}
\phantomsection\label{adapt/domain_adapters:escape.adapt.domain_adapters.InternalDomainManager.controller_name}\pysigline{\bfcode{controller\_name}}
\end{fulllineitems}

\index{install\_nffg() (escape.adapt.domain\_adapters.InternalDomainManager method)}

\begin{fulllineitems}
\phantomsection\label{adapt/domain_adapters:escape.adapt.domain_adapters.InternalDomainManager.install_nffg}\pysiglinewithargsret{\bfcode{install\_nffg}}{\emph{nffg\_part}}{}
Install an {\hyperref[util/nffg:escape.util.nffg.NFFG]{\emph{\code{NFFG}}}} related to the internal domain

Split given {\hyperref[util/nffg:escape.util.nffg.NFFG]{\emph{\code{NFFG}}}} to a set of NFs need to be initiated and a set of
routes/connections between the NFs
\begin{quote}\begin{description}
\item[{Parameters}] \leavevmode
\textbf{\texttt{nffg\_part}} ({\hyperref[util/nffg:escape.util.nffg.NFFG]{\emph{\code{NFFG}}}}) -- NF-FG need to be deployed

\item[{Returns}] \leavevmode
None

\end{description}\end{quote}

\end{fulllineitems}


\end{fulllineitems}

\index{OpenStackDomainManager (class in escape.adapt.domain\_adapters)}

\begin{fulllineitems}
\phantomsection\label{adapt/domain_adapters:escape.adapt.domain_adapters.OpenStackDomainManager}\pysiglinewithargsret{\strong{class }\code{escape.adapt.domain\_adapters.}\bfcode{OpenStackDomainManager}}{\emph{url}}{}
Bases: {\hyperref[util/adapter:escape.util.adapter.AbstractDomainManager]{\emph{\code{escape.util.adapter.AbstractDomainManager}}}}

Adapter class to handle communication with OpenStack domain

\begin{notice}{warning}{Warning:}
Not implemented yet!
\end{notice}
\index{name (escape.adapt.domain\_adapters.OpenStackDomainManager attribute)}

\begin{fulllineitems}
\phantomsection\label{adapt/domain_adapters:escape.adapt.domain_adapters.OpenStackDomainManager.name}\pysigline{\bfcode{name}\strong{ = `OPENSTACK'}}
\end{fulllineitems}

\index{\_\_init\_\_() (escape.adapt.domain\_adapters.OpenStackDomainManager method)}

\begin{fulllineitems}
\phantomsection\label{adapt/domain_adapters:escape.adapt.domain_adapters.OpenStackDomainManager.__init__}\pysiglinewithargsret{\bfcode{\_\_init\_\_}}{\emph{url}}{}
Init
\begin{quote}\begin{description}
\item[{Parameters}] \leavevmode
\textbf{\texttt{url}} (\href{https://docs.python.org/2.7/library/functions.html\#str}{\emph{str}}) -- OpenStack RESTful API URL

\end{description}\end{quote}

\end{fulllineitems}

\index{install\_nffg() (escape.adapt.domain\_adapters.OpenStackDomainManager method)}

\begin{fulllineitems}
\phantomsection\label{adapt/domain_adapters:escape.adapt.domain_adapters.OpenStackDomainManager.install_nffg}\pysiglinewithargsret{\bfcode{install\_nffg}}{\emph{nffg\_part}}{}
\end{fulllineitems}


\end{fulllineitems}

\index{DockerDomainManager (class in escape.adapt.domain\_adapters)}

\begin{fulllineitems}
\phantomsection\label{adapt/domain_adapters:escape.adapt.domain_adapters.DockerDomainManager}\pysigline{\strong{class }\code{escape.adapt.domain\_adapters.}\bfcode{DockerDomainManager}}
Bases: {\hyperref[util/adapter:escape.util.adapter.AbstractDomainManager]{\emph{\code{escape.util.adapter.AbstractDomainManager}}}}

Adapter class to handle communication component in a Docker domain

\begin{notice}{warning}{Warning:}
Not implemented yet!
\end{notice}
\index{name (escape.adapt.domain\_adapters.DockerDomainManager attribute)}

\begin{fulllineitems}
\phantomsection\label{adapt/domain_adapters:escape.adapt.domain_adapters.DockerDomainManager.name}\pysigline{\bfcode{name}\strong{ = `DOCKER'}}
\end{fulllineitems}

\index{\_\_init\_\_() (escape.adapt.domain\_adapters.DockerDomainManager method)}

\begin{fulllineitems}
\phantomsection\label{adapt/domain_adapters:escape.adapt.domain_adapters.DockerDomainManager.__init__}\pysiglinewithargsret{\bfcode{\_\_init\_\_}}{}{}
Init

\end{fulllineitems}

\index{install\_nffg() (escape.adapt.domain\_adapters.DockerDomainManager method)}

\begin{fulllineitems}
\phantomsection\label{adapt/domain_adapters:escape.adapt.domain_adapters.DockerDomainManager.install_nffg}\pysiglinewithargsret{\bfcode{install\_nffg}}{\emph{nffg\_part}}{}
\end{fulllineitems}


\end{fulllineitems}



\subparagraph{\emph{escape.infr} package}
\label{infr/infr:module-escape.infr}\label{infr/infr::doc}\label{infr/infr:escape-infr-package}\index{escape.infr (module)}
Sublayer for classes related to UNIFY's Infrastructure Layer (IL)


\subparagraph{Submodules}
\label{infr/infr:submodules}

\subparagraph{\emph{il\_API.py} module}
\label{infr/il_API::doc}\label{infr/il_API:il-api-py-module}
{\hyperref[infr/il_API:escape.infr.il_API.InfrastructureLayerAPI]{\emph{\code{InfrastructureLayerAPI}}}} represents the IL layer and implement all
related functionality


\subparagraph{Module contents}
\label{infr/il_API:module-contents}\label{infr/il_API:module-escape.infr.il_API}\index{escape.infr.il\_API (module)}
Emulate UNIFY's Infrastructure Layer for testing purposes based on Mininet
\index{DeploymentFinishedEvent (class in escape.infr.il\_API)}

\begin{fulllineitems}
\phantomsection\label{infr/il_API:escape.infr.il_API.DeploymentFinishedEvent}\pysiglinewithargsret{\strong{class }\code{escape.infr.il\_API.}\bfcode{DeploymentFinishedEvent}}{\emph{success}, \emph{error=None}}{}
Bases: \code{pox.lib.revent.revent.Event}

Event for signaling NF-FG deployment
\index{\_\_init\_\_() (escape.infr.il\_API.DeploymentFinishedEvent method)}

\begin{fulllineitems}
\phantomsection\label{infr/il_API:escape.infr.il_API.DeploymentFinishedEvent.__init__}\pysiglinewithargsret{\bfcode{\_\_init\_\_}}{\emph{success}, \emph{error=None}}{}
\end{fulllineitems}


\end{fulllineitems}

\index{InfrastructureLayerAPI (class in escape.infr.il\_API)}

\begin{fulllineitems}
\phantomsection\label{infr/il_API:escape.infr.il_API.InfrastructureLayerAPI}\pysiglinewithargsret{\strong{class }\code{escape.infr.il\_API.}\bfcode{InfrastructureLayerAPI}}{\emph{standalone=False}, \emph{**kwargs}}{}
Bases: {\hyperref[util/api:escape.util.api.AbstractAPI]{\emph{\code{escape.util.api.AbstractAPI}}}}

Entry point for Infrastructure Layer (IL)

Maintain the contact with other UNIFY layers

Implement a specific part of the Co - Rm reference point
\index{\_core\_name (escape.infr.il\_API.InfrastructureLayerAPI attribute)}

\begin{fulllineitems}
\phantomsection\label{infr/il_API:escape.infr.il_API.InfrastructureLayerAPI._core_name}\pysigline{\bfcode{\_core\_name}\strong{ = `infrastructure'}}
\end{fulllineitems}

\index{\_eventMixin\_events (escape.infr.il\_API.InfrastructureLayerAPI attribute)}

\begin{fulllineitems}
\phantomsection\label{infr/il_API:escape.infr.il_API.InfrastructureLayerAPI._eventMixin_events}\pysigline{\bfcode{\_eventMixin\_events}\strong{ = set({[}\textless{}class `escape.infr.il\_API.DeploymentFinishedEvent'\textgreater{}{]})}}
\end{fulllineitems}

\index{\_\_init\_\_() (escape.infr.il\_API.InfrastructureLayerAPI method)}

\begin{fulllineitems}
\phantomsection\label{infr/il_API:escape.infr.il_API.InfrastructureLayerAPI.__init__}\pysiglinewithargsret{\bfcode{\_\_init\_\_}}{\emph{standalone=False}, \emph{**kwargs}}{}~

\strong{See also:}


{\hyperref[util/api:escape.util.api.AbstractAPI.__init__]{\emph{\code{AbstractAPI.\_\_init\_\_()}}}}



\end{fulllineitems}

\index{initialize() (escape.infr.il\_API.InfrastructureLayerAPI method)}

\begin{fulllineitems}
\phantomsection\label{infr/il_API:escape.infr.il_API.InfrastructureLayerAPI.initialize}\pysiglinewithargsret{\bfcode{initialize}}{}{}~

\strong{See also:}


{\hyperref[util/api:escape.util.api.AbstractAPI.initialize]{\emph{\code{AbstractAPI.initialize()}}}}



\end{fulllineitems}

\index{shutdown() (escape.infr.il\_API.InfrastructureLayerAPI method)}

\begin{fulllineitems}
\phantomsection\label{infr/il_API:escape.infr.il_API.InfrastructureLayerAPI.shutdown}\pysiglinewithargsret{\bfcode{shutdown}}{\emph{event}}{}~

\strong{See also:}


{\hyperref[util/api:escape.util.api.AbstractAPI.shutdown]{\emph{\code{AbstractAPI.shutdown()}}}}



\end{fulllineitems}

\index{\_handle\_ComponentRegistered() (escape.infr.il\_API.InfrastructureLayerAPI method)}

\begin{fulllineitems}
\phantomsection\label{infr/il_API:escape.infr.il_API.InfrastructureLayerAPI._handle_ComponentRegistered}\pysiglinewithargsret{\bfcode{\_handle\_ComponentRegistered}}{\emph{event}}{}
Wait for controller (internal POX module)
\begin{quote}\begin{description}
\item[{Parameters}] \leavevmode
\textbf{\texttt{event}} (\code{ComponentRegistered}) -- registered component event

\item[{Returns}] \leavevmode
None

\end{description}\end{quote}

\end{fulllineitems}

\index{\_handle\_DeployNFFGEvent() (escape.infr.il\_API.InfrastructureLayerAPI method)}

\begin{fulllineitems}
\phantomsection\label{infr/il_API:escape.infr.il_API.InfrastructureLayerAPI._handle_DeployNFFGEvent}\pysiglinewithargsret{\bfcode{\_handle\_DeployNFFGEvent}}{\emph{*args}, \emph{**kwargs}}{}
Install mapped NFFG part into the emulated network

:param event:event object
:return: {\hyperref[adapt/cas_API:escape.adapt.cas_API.DeployNFFGEvent]{\emph{\code{DeployNFFGEvent}}}}

\end{fulllineitems}

\index{install\_route() (escape.infr.il\_API.InfrastructureLayerAPI method)}

\begin{fulllineitems}
\phantomsection\label{infr/il_API:escape.infr.il_API.InfrastructureLayerAPI.install_route}\pysiglinewithargsret{\bfcode{install\_route}}{}{}
\end{fulllineitems}


\end{fulllineitems}



\subparagraph{\emph{topology.py} module}
\label{infr/topology:topology-py-module}\label{infr/topology::doc}
{\hyperref[infr/topology:escape.infr.topology.InternalControllerProxy]{\emph{\code{InternalControllerProxy}}}} represents the connection between the internal
controller and the emulated network

{\hyperref[infr/topology:escape.infr.topology.AbstractTopology]{\emph{\code{AbstractTopology}}}} represents the emulated topology for the high-level API


\subparagraph{Module contents}
\label{infr/topology:module-contents}\label{infr/topology:module-escape.infr.topology}\index{escape.infr.topology (module)}
Wrapper module for handling emulated test topology based on Mininet
\index{AbstractTopology (class in escape.infr.topology)}

\begin{fulllineitems}
\phantomsection\label{infr/topology:escape.infr.topology.AbstractTopology}\pysiglinewithargsret{\strong{class }\code{escape.infr.topology.}\bfcode{AbstractTopology}}{\emph{hopts=None}, \emph{sopts=None}, \emph{lopts=None}, \emph{eopts=None}}{}
Bases: \code{mininet.topo.Topo}

Abstract class for representing emulated topology.

Have the functions to build a ESCAPE-specific topology.

Can be used to define reusable topology similar to Mininet's high-level API.
Reusable, convenient and pre-defined way to define a topology, but less
flexible and powerful.
\index{default\_host\_opts (escape.infr.topology.AbstractTopology attribute)}

\begin{fulllineitems}
\phantomsection\label{infr/topology:escape.infr.topology.AbstractTopology.default_host_opts}\pysigline{\bfcode{default\_host\_opts}\strong{ = None}}
\end{fulllineitems}

\index{default\_switch\_opts (escape.infr.topology.AbstractTopology attribute)}

\begin{fulllineitems}
\phantomsection\label{infr/topology:escape.infr.topology.AbstractTopology.default_switch_opts}\pysigline{\bfcode{default\_switch\_opts}\strong{ = None}}
\end{fulllineitems}

\index{default\_link\_opts (escape.infr.topology.AbstractTopology attribute)}

\begin{fulllineitems}
\phantomsection\label{infr/topology:escape.infr.topology.AbstractTopology.default_link_opts}\pysigline{\bfcode{default\_link\_opts}\strong{ = None}}
\end{fulllineitems}

\index{default\_EE\_opts (escape.infr.topology.AbstractTopology attribute)}

\begin{fulllineitems}
\phantomsection\label{infr/topology:escape.infr.topology.AbstractTopology.default_EE_opts}\pysigline{\bfcode{default\_EE\_opts}\strong{ = None}}
\end{fulllineitems}

\index{\_\_init\_\_() (escape.infr.topology.AbstractTopology method)}

\begin{fulllineitems}
\phantomsection\label{infr/topology:escape.infr.topology.AbstractTopology.__init__}\pysiglinewithargsret{\bfcode{\_\_init\_\_}}{\emph{hopts=None}, \emph{sopts=None}, \emph{lopts=None}, \emph{eopts=None}}{}
\end{fulllineitems}


\end{fulllineitems}

\index{BackupTopology (class in escape.infr.topology)}

\begin{fulllineitems}
\phantomsection\label{infr/topology:escape.infr.topology.BackupTopology}\pysigline{\strong{class }\code{escape.infr.topology.}\bfcode{BackupTopology}}
Bases: {\hyperref[infr/topology:escape.infr.topology.AbstractTopology]{\emph{\code{escape.infr.topology.AbstractTopology}}}}

Topology class for testing purposes and serve as a fallback topology
\index{\_\_init\_\_() (escape.infr.topology.BackupTopology method)}

\begin{fulllineitems}
\phantomsection\label{infr/topology:escape.infr.topology.BackupTopology.__init__}\pysiglinewithargsret{\bfcode{\_\_init\_\_}}{}{}
Init and build test topology

\end{fulllineitems}


\end{fulllineitems}

\index{InternalControllerProxy (class in escape.infr.topology)}

\begin{fulllineitems}
\phantomsection\label{infr/topology:escape.infr.topology.InternalControllerProxy}\pysiglinewithargsret{\strong{class }\code{escape.infr.topology.}\bfcode{InternalControllerProxy}}{\emph{name='InternalPOXController'}, \emph{ip=`127.0.0.1'}, \emph{port=6653}, \emph{**kwargs}}{}
Bases: \code{mininet.node.RemoteController}

Controller class for emulated Mininet network. Making connection with
internal controller initiated by POXDomainAdapter.
\index{\_\_init\_\_() (escape.infr.topology.InternalControllerProxy method)}

\begin{fulllineitems}
\phantomsection\label{infr/topology:escape.infr.topology.InternalControllerProxy.__init__}\pysiglinewithargsret{\bfcode{\_\_init\_\_}}{\emph{name='InternalPOXController'}, \emph{ip=`127.0.0.1'}, \emph{port=6653}, \emph{**kwargs}}{}
Init
\begin{quote}\begin{description}
\item[{Parameters}] \leavevmode\begin{itemize}
\item {} 
\textbf{\texttt{name}} (\href{https://docs.python.org/2.7/library/functions.html\#str}{\emph{str}}) -- name of the controller (default: InternalPOXController)

\item {} 
\textbf{\texttt{ip}} (\href{https://docs.python.org/2.7/library/functions.html\#str}{\emph{str}}) -- IP address (default: 127.0.0.1)

\item {} 
\textbf{\texttt{port}} (\href{https://docs.python.org/2.7/library/functions.html\#int}{\emph{int}}) -- port number (default 6633)

\end{itemize}

\end{description}\end{quote}

\end{fulllineitems}

\index{checkListening() (escape.infr.topology.InternalControllerProxy method)}

\begin{fulllineitems}
\phantomsection\label{infr/topology:escape.infr.topology.InternalControllerProxy.checkListening}\pysiglinewithargsret{\bfcode{checkListening}}{}{}
Check the controller port is open

\end{fulllineitems}


\end{fulllineitems}

\index{ESCAPENetworkBridge (class in escape.infr.topology)}

\begin{fulllineitems}
\phantomsection\label{infr/topology:escape.infr.topology.ESCAPENetworkBridge}\pysiglinewithargsret{\strong{class }\code{escape.infr.topology.}\bfcode{ESCAPENetworkBridge}}{\emph{network=None}}{}
Bases: \href{https://docs.python.org/2.7/library/functions.html\#object}{\code{object}}

Internal class for representing the emulated topology.

Represents a container class for network elements such as switches, nodes,
execution environments, links etc. Contains network management functions
similar to Mininet's mid-level API extended with ESCAPEv2 related capabilities

Separate the interface using internally from original Mininet object to
implement loose coupling and avoid changes caused by Mininet API changes
e.g. 2.1.0 -\textgreater{} 2.2.0

Follows Bridge design pattern.
\index{\_\_init\_\_() (escape.infr.topology.ESCAPENetworkBridge method)}

\begin{fulllineitems}
\phantomsection\label{infr/topology:escape.infr.topology.ESCAPENetworkBridge.__init__}\pysiglinewithargsret{\bfcode{\_\_init\_\_}}{\emph{network=None}}{}
Initialize Mininet implementation with proper attributes.

\end{fulllineitems}

\index{network (escape.infr.topology.ESCAPENetworkBridge attribute)}

\begin{fulllineitems}
\phantomsection\label{infr/topology:escape.infr.topology.ESCAPENetworkBridge.network}\pysigline{\bfcode{network}}
Internal network representation
\begin{quote}\begin{description}
\item[{Returns}] \leavevmode
network representation

\item[{Return type}] \leavevmode
\code{mininet.net.Mininet}

\end{description}\end{quote}

\end{fulllineitems}

\index{start\_network() (escape.infr.topology.ESCAPENetworkBridge method)}

\begin{fulllineitems}
\phantomsection\label{infr/topology:escape.infr.topology.ESCAPENetworkBridge.start_network}\pysiglinewithargsret{\bfcode{start\_network}}{}{}
Start network

\end{fulllineitems}

\index{stop\_network() (escape.infr.topology.ESCAPENetworkBridge method)}

\begin{fulllineitems}
\phantomsection\label{infr/topology:escape.infr.topology.ESCAPENetworkBridge.stop_network}\pysiglinewithargsret{\bfcode{stop\_network}}{}{}
Stop network

\end{fulllineitems}

\index{cleanup() (escape.infr.topology.ESCAPENetworkBridge method)}

\begin{fulllineitems}
\phantomsection\label{infr/topology:escape.infr.topology.ESCAPENetworkBridge.cleanup}\pysiglinewithargsret{\bfcode{cleanup}}{}{}
Clean up junk which might be left over from old runs.
\begin{description}
\item[{..seealso::}] \leavevmode
\code{mininet.clean.cleanup()}

\end{description}

\end{fulllineitems}


\end{fulllineitems}

\index{TopologyBuilderException}

\begin{fulllineitems}
\phantomsection\label{infr/topology:escape.infr.topology.TopologyBuilderException}\pysigline{\strong{exception }\code{escape.infr.topology.}\bfcode{TopologyBuilderException}}
Bases: \href{https://docs.python.org/2.7/library/exceptions.html\#exceptions.Exception}{\code{exceptions.Exception}}

Exception class for topology errors.

\end{fulllineitems}

\index{ESCAPENetworkBuilder (class in escape.infr.topology)}

\begin{fulllineitems}
\phantomsection\label{infr/topology:escape.infr.topology.ESCAPENetworkBuilder}\pysiglinewithargsret{\strong{class }\code{escape.infr.topology.}\bfcode{ESCAPENetworkBuilder}}{\emph{net=None}, \emph{opts=None}, \emph{run\_dry=True}}{}
Bases: \href{https://docs.python.org/2.7/library/functions.html\#object}{\code{object}}

Builder class for topology.

Update the network object based on the parameters if it's given or create
an empty instance.

Always return with an ESCAPENetworkBridge instance which offer a generic
interface for created :any::\emph{Mininet} object and hide implementation's nature.

Follows Builder design pattern.
\index{default\_opts (escape.infr.topology.ESCAPENetworkBuilder attribute)}

\begin{fulllineitems}
\phantomsection\label{infr/topology:escape.infr.topology.ESCAPENetworkBuilder.default_opts}\pysigline{\bfcode{default\_opts}\strong{ = \{`listenPort': None, `autoSetMacs': True, `inNamespace': False, `autoStaticArp': True, `controller': \textless{}class `escape.infr.topology.InternalControllerProxy'\textgreater{}, `build': False\}}}
\end{fulllineitems}

\index{topology\_config\_name (escape.infr.topology.ESCAPENetworkBuilder attribute)}

\begin{fulllineitems}
\phantomsection\label{infr/topology:escape.infr.topology.ESCAPENetworkBuilder.topology_config_name}\pysigline{\bfcode{topology\_config\_name}\strong{ = `FALLBACK-TOPO'}}
\end{fulllineitems}

\index{\_\_init\_\_() (escape.infr.topology.ESCAPENetworkBuilder method)}

\begin{fulllineitems}
\phantomsection\label{infr/topology:escape.infr.topology.ESCAPENetworkBuilder.__init__}\pysiglinewithargsret{\bfcode{\_\_init\_\_}}{\emph{net=None}, \emph{opts=None}, \emph{run\_dry=True}}{}
Initialize NetworkBuilder.

If the topology definition is not found, an exception will be raised or
an empty :any::\emph{Mininet} topology will be created if \code{run\_dry} is set.
\begin{quote}\begin{description}
\item[{Parameters}] \leavevmode\begin{itemize}
\item {} 
\textbf{\texttt{net}} (:any::\emph{Mininet}) -- update given Mininet object instead of creating a new one

\item {} 
\textbf{\texttt{opts}} (\href{https://docs.python.org/2.7/library/stdtypes.html\#dict}{\emph{dict}}) -- update default options with the given opts

\item {} 
\textbf{\texttt{run\_dry}} (\href{https://docs.python.org/2.7/library/functions.html\#bool}{\emph{bool}}) -- do not raise an Exception and return with bare Mininet obj.

\end{itemize}

\end{description}\end{quote}

\end{fulllineitems}

\index{\_ESCAPENetworkBuilder\_\_init\_from\_AbstractTopology() (escape.infr.topology.ESCAPENetworkBuilder method)}

\begin{fulllineitems}
\phantomsection\label{infr/topology:escape.infr.topology.ESCAPENetworkBuilder._ESCAPENetworkBuilder__init_from_AbstractTopology}\pysiglinewithargsret{\bfcode{\_ESCAPENetworkBuilder\_\_init\_from\_AbstractTopology}}{\emph{topo\_class}}{}
Build topology from pre-defined Topology class.
\begin{quote}\begin{description}
\item[{Parameters}] \leavevmode
\textbf{\texttt{topo}} ({\hyperref[infr/topology:escape.infr.topology.AbstractTopology]{\emph{\code{AbstractTopology}}}}) -- topology

\item[{Returns}] \leavevmode
None

\end{description}\end{quote}

\end{fulllineitems}

\index{\_ESCAPENetworkBuilder\_\_init\_from\_CONFIG() (escape.infr.topology.ESCAPENetworkBuilder method)}

\begin{fulllineitems}
\phantomsection\label{infr/topology:escape.infr.topology.ESCAPENetworkBuilder._ESCAPENetworkBuilder__init_from_CONFIG}\pysiglinewithargsret{\bfcode{\_ESCAPENetworkBuilder\_\_init\_from\_CONFIG}}{}{}
Build a pre-defined topology stored in CONFIG.
\begin{quote}\begin{description}
\item[{Returns}] \leavevmode
None

\end{description}\end{quote}

\end{fulllineitems}

\index{\_ESCAPENetworkBuilder\_\_init\_from\_NFFG() (escape.infr.topology.ESCAPENetworkBuilder method)}

\begin{fulllineitems}
\phantomsection\label{infr/topology:escape.infr.topology.ESCAPENetworkBuilder._ESCAPENetworkBuilder__init_from_NFFG}\pysiglinewithargsret{\bfcode{\_ESCAPENetworkBuilder\_\_init\_from\_NFFG}}{\emph{net}, \emph{nffg}}{}
Initialize topology from {\hyperref[util/nffg:escape.util.nffg.NFFG]{\emph{\code{NFFG}}}}
\begin{quote}\begin{description}
\item[{Parameters}] \leavevmode
\textbf{\texttt{nffg}} ({\hyperref[util/nffg:escape.util.nffg.NFFG]{\emph{\code{NFFG}}}}) -- topology

\item[{Returns}] \leavevmode
None

\end{description}\end{quote}

\end{fulllineitems}

\index{\_ESCAPENetworkBuilder\_\_init\_from\_dict() (escape.infr.topology.ESCAPENetworkBuilder method)}

\begin{fulllineitems}
\phantomsection\label{infr/topology:escape.infr.topology.ESCAPENetworkBuilder._ESCAPENetworkBuilder__init_from_dict}\pysiglinewithargsret{\bfcode{\_ESCAPENetworkBuilder\_\_init\_from\_dict}}{\emph{dict}}{}
Initialize topology from a dictionary.

Keywords for network elements: controllers, ee, saps, switches, links

Option keywords: netopts
\begin{quote}\begin{description}
\item[{Parameters}] \leavevmode
\textbf{\texttt{dict}} ({\hyperref[util/nffg:escape.util.nffg.NFFG]{\emph{\code{NFFG}}}}) -- topology

\item[{Returns}] \leavevmode
None

\end{description}\end{quote}

\end{fulllineitems}

\index{\_ESCAPENetworkBuilder\_\_init\_from\_file() (escape.infr.topology.ESCAPENetworkBuilder method)}

\begin{fulllineitems}
\phantomsection\label{infr/topology:escape.infr.topology.ESCAPENetworkBuilder._ESCAPENetworkBuilder__init_from_file}\pysiglinewithargsret{\bfcode{\_ESCAPENetworkBuilder\_\_init\_from\_file}}{\emph{path='escape.topo'}}{}
Build a pre-defined topology stored in a file in JSON format.
\begin{quote}\begin{description}
\item[{Parameters}] \leavevmode
\textbf{\texttt{path}} (\href{https://docs.python.org/2.7/library/functions.html\#str}{\emph{str}}) -- file path

\item[{Returns}] \leavevmode
None

\end{description}\end{quote}

\end{fulllineitems}

\index{build() (escape.infr.topology.ESCAPENetworkBuilder method)}

\begin{fulllineitems}
\phantomsection\label{infr/topology:escape.infr.topology.ESCAPENetworkBuilder.build}\pysiglinewithargsret{\bfcode{build}}{\emph{topology=None}}{}
Initialize network
\begin{quote}\begin{description}
\item[{Parameters}] \leavevmode
\textbf{\texttt{topology}} ({\hyperref[util/nffg:escape.util.nffg.NFFG]{\emph{\code{NFFG}}}} or \href{https://docs.python.org/2.7/reference/expressions.html\#dict}{\code{Dictionary displays}} or {\hyperref[infr/topology:escape.infr.topology.AbstractTopology]{\emph{\code{AbstractTopology}}}}) -- topology representation

\item[{Returns}] \leavevmode
None

\end{description}\end{quote}

\end{fulllineitems}


\end{fulllineitems}



\subparagraph{\emph{escape.util} package}
\label{util/util:module-escape.util}\label{util/util::doc}\label{util/util:escape-util-package}\index{escape.util (module)}
Additional functions, classes, components


\subparagraph{Submodules}
\label{util/util:submodules}

\subparagraph{\emph{adapter.py} module}
\label{util/adapter::doc}\label{util/adapter:adapter-py-module}
{\hyperref[util/adapter:escape.util.adapter.DomainChangedEvent]{\emph{\code{DomainChangedEvent}}}} signals changes for {\hyperref[adapt/adaptation:escape.adapt.adaptation.ControllerAdapter]{\emph{\code{ControllerAdapter}}}} in
an unified way.

{\hyperref[util/adapter:escape.util.adapter.AbstractDomainManager]{\emph{\code{AbstractDomainManager}}}} contains general logic for top domain managers

{\hyperref[util/adapter:escape.util.adapter.AbstractDomainAdapter]{\emph{\code{AbstractDomainAdapter}}}} contains general logic for actual Adapters.

{\hyperref[util/adapter:escape.util.adapter.VNFStarterAPI]{\emph{\code{VNFStarterAPI}}}} defines the interface for VNF management based on
VNFStarter YANG description.

{\hyperref[util/adapter:escape.util.adapter.OpenStackAPI]{\emph{\code{OpenStackAPI}}}} defines the interface for communication with OpenStack
domain.

Requirements:

\begin{Verbatim}[commandchars=\\\{\}]
sudo pip install requests
\end{Verbatim}

{\hyperref[util/adapter:escape.util.adapter.AbstractRESTAdapter]{\emph{\code{AbstractRESTAdapter}}}} contains the general functions for communication
through an HTTP/RESTful API


\subparagraph{Module contents}
\label{util/adapter:module-contents}\label{util/adapter:module-escape.util.adapter}\index{escape.util.adapter (module)}
Implement the supporting classes for doamin adapters
\index{DomainChangedEvent (class in escape.util.adapter)}

\begin{fulllineitems}
\phantomsection\label{util/adapter:escape.util.adapter.DomainChangedEvent}\pysiglinewithargsret{\strong{class }\code{escape.util.adapter.}\bfcode{DomainChangedEvent}}{\emph{domain}, \emph{cause}, \emph{data=None}}{}
Bases: \code{pox.lib.revent.revent.Event}

Event class for signaling all kind of change(s) in specific domain

This event's purpose is to hide the domain specific operations and give a
general and unified way to signal domain changes to ControllerAdapter in
order to handle all the changes in the same function/algorithm
\index{type (escape.util.adapter.DomainChangedEvent attribute)}

\begin{fulllineitems}
\phantomsection\label{util/adapter:escape.util.adapter.DomainChangedEvent.type}\pysigline{\bfcode{type}}
alias of \code{enum}

\end{fulllineitems}

\index{\_\_init\_\_() (escape.util.adapter.DomainChangedEvent method)}

\begin{fulllineitems}
\phantomsection\label{util/adapter:escape.util.adapter.DomainChangedEvent.__init__}\pysiglinewithargsret{\bfcode{\_\_init\_\_}}{\emph{domain}, \emph{cause}, \emph{data=None}}{}
Init event object
\begin{quote}\begin{description}
\item[{Parameters}] \leavevmode\begin{itemize}
\item {} 
\textbf{\texttt{domain}} (\href{https://docs.python.org/2.7/library/functions.html\#str}{\emph{str}}) -- domain name. Should be {\hyperref[util/adapter:escape.util.adapter.AbstractDomainAdapter.name]{\emph{\code{AbstractDomainAdapter.name}}}}

\item {} 
\textbf{\texttt{cause}} (\href{https://docs.python.org/2.7/library/functions.html\#str}{\emph{str}}) -- type of the domain change: {\hyperref[util/adapter:escape.util.adapter.DomainChangedEvent.type]{\emph{\code{DomainChangedEvent.type}}}}

\item {} 
\textbf{\texttt{data}} (\href{https://docs.python.org/2.7/library/functions.html\#object}{\emph{object}}) -- data connected to the change (optional)

\end{itemize}

\item[{Returns}] \leavevmode
None

\end{description}\end{quote}

\end{fulllineitems}


\end{fulllineitems}

\index{DeployEvent (class in escape.util.adapter)}

\begin{fulllineitems}
\phantomsection\label{util/adapter:escape.util.adapter.DeployEvent}\pysiglinewithargsret{\strong{class }\code{escape.util.adapter.}\bfcode{DeployEvent}}{\emph{nffg\_part}}{}
Bases: \code{pox.lib.revent.revent.Event}

Event class for signaling NF-FG deployment to infrastructure layer API

Used by DirectMininetAdapter for internal NF-FG deployment
\index{\_\_init\_\_() (escape.util.adapter.DeployEvent method)}

\begin{fulllineitems}
\phantomsection\label{util/adapter:escape.util.adapter.DeployEvent.__init__}\pysiglinewithargsret{\bfcode{\_\_init\_\_}}{\emph{nffg\_part}}{}
\end{fulllineitems}


\end{fulllineitems}

\index{AbstractDomainManager (class in escape.util.adapter)}

\begin{fulllineitems}
\phantomsection\label{util/adapter:escape.util.adapter.AbstractDomainManager}\pysigline{\strong{class }\code{escape.util.adapter.}\bfcode{AbstractDomainManager}}
Bases: \code{pox.lib.revent.revent.EventMixin}

Abstract class for different domain managers

Domain managers is top level classes to handle and manage domains
transparently

Follows the MixIn design pattern approach to support general manager
functionality for topmost ControllerAdapter class

Follows the Component Configurator design pattern as base component class
\index{init() (escape.util.adapter.AbstractDomainManager method)}

\begin{fulllineitems}
\phantomsection\label{util/adapter:escape.util.adapter.AbstractDomainManager.init}\pysiglinewithargsret{\bfcode{init}}{}{}
Abstract function for component initialization

\end{fulllineitems}

\index{run() (escape.util.adapter.AbstractDomainManager method)}

\begin{fulllineitems}
\phantomsection\label{util/adapter:escape.util.adapter.AbstractDomainManager.run}\pysiglinewithargsret{\bfcode{run}}{}{}
Abstract function for starting component

\end{fulllineitems}

\index{finit() (escape.util.adapter.AbstractDomainManager method)}

\begin{fulllineitems}
\phantomsection\label{util/adapter:escape.util.adapter.AbstractDomainManager.finit}\pysiglinewithargsret{\bfcode{finit}}{}{}
Abstract function for starting component

\end{fulllineitems}

\index{suspend() (escape.util.adapter.AbstractDomainManager method)}

\begin{fulllineitems}
\phantomsection\label{util/adapter:escape.util.adapter.AbstractDomainManager.suspend}\pysiglinewithargsret{\bfcode{suspend}}{}{}
Abstract class for suspending a running component

\end{fulllineitems}

\index{resume() (escape.util.adapter.AbstractDomainManager method)}

\begin{fulllineitems}
\phantomsection\label{util/adapter:escape.util.adapter.AbstractDomainManager.resume}\pysiglinewithargsret{\bfcode{resume}}{}{}
Abstract function for resuming a suspended component

\end{fulllineitems}

\index{info() (escape.util.adapter.AbstractDomainManager method)}

\begin{fulllineitems}
\phantomsection\label{util/adapter:escape.util.adapter.AbstractDomainManager.info}\pysiglinewithargsret{\bfcode{info}}{}{}
Abstract function for requesting information about the component

\end{fulllineitems}

\index{install\_nffg() (escape.util.adapter.AbstractDomainManager method)}

\begin{fulllineitems}
\phantomsection\label{util/adapter:escape.util.adapter.AbstractDomainManager.install_nffg}\pysiglinewithargsret{\bfcode{install\_nffg}}{\emph{nffg\_part}}{}
Install an {\hyperref[util/nffg:escape.util.nffg.NFFG]{\emph{\code{NFFG}}}} related to the specific domain
\begin{quote}\begin{description}
\item[{Parameters}] \leavevmode
\textbf{\texttt{nffg\_part}} ({\hyperref[util/nffg:escape.util.nffg.NFFG]{\emph{\code{NFFG}}}}) -- NF-FG need to be deployed

\item[{Returns}] \leavevmode
None

\end{description}\end{quote}

\end{fulllineitems}


\end{fulllineitems}

\index{AbstractDomainAdapter (class in escape.util.adapter)}

\begin{fulllineitems}
\phantomsection\label{util/adapter:escape.util.adapter.AbstractDomainAdapter}\pysigline{\strong{class }\code{escape.util.adapter.}\bfcode{AbstractDomainAdapter}}
Bases: \code{pox.lib.revent.revent.EventMixin}

Abstract class for different domain adapters.

Domain adapters can handle domains as a whole or well-separated parts of a
domain e.g. control part of an SDN network, infrastructure containers or
other entities through a specific protocol (NETCONF, HTTP/REST).

Follows the Adapter design pattern (Adaptor base class).

Follows the MixIn design patteran approach to support general adapter
functionality for manager classes mostly.
\index{\_eventMixin\_events (escape.util.adapter.AbstractDomainAdapter attribute)}

\begin{fulllineitems}
\phantomsection\label{util/adapter:escape.util.adapter.AbstractDomainAdapter._eventMixin_events}\pysigline{\bfcode{\_eventMixin\_events}\strong{ = set({[}\textless{}class `escape.util.adapter.DomainChangedEvent'\textgreater{}{]})}}
\end{fulllineitems}

\index{name (escape.util.adapter.AbstractDomainAdapter attribute)}

\begin{fulllineitems}
\phantomsection\label{util/adapter:escape.util.adapter.AbstractDomainAdapter.name}\pysigline{\bfcode{name}\strong{ = None}}
\end{fulllineitems}

\index{\_\_init\_\_() (escape.util.adapter.AbstractDomainAdapter method)}

\begin{fulllineitems}
\phantomsection\label{util/adapter:escape.util.adapter.AbstractDomainAdapter.__init__}\pysiglinewithargsret{\bfcode{\_\_init\_\_}}{}{}
Init

\end{fulllineitems}

\index{start\_polling() (escape.util.adapter.AbstractDomainAdapter method)}

\begin{fulllineitems}
\phantomsection\label{util/adapter:escape.util.adapter.AbstractDomainAdapter.start_polling}\pysiglinewithargsret{\bfcode{start\_polling}}{\emph{wait=1}}{}
Initialize and start a Timer co-op task for polling.
\begin{quote}\begin{description}
\item[{Parameters}] \leavevmode
\textbf{\texttt{wait}} (\href{https://docs.python.org/2.7/library/functions.html\#int}{\emph{int}}) -- polling period (default: 1)

\end{description}\end{quote}

\end{fulllineitems}

\index{stop\_polling() (escape.util.adapter.AbstractDomainAdapter method)}

\begin{fulllineitems}
\phantomsection\label{util/adapter:escape.util.adapter.AbstractDomainAdapter.stop_polling}\pysiglinewithargsret{\bfcode{stop\_polling}}{}{}
Stop timer.

\end{fulllineitems}

\index{poll() (escape.util.adapter.AbstractDomainAdapter method)}

\begin{fulllineitems}
\phantomsection\label{util/adapter:escape.util.adapter.AbstractDomainAdapter.poll}\pysiglinewithargsret{\bfcode{poll}}{}{}
Template fuction to poll domain state. Called by a Timer co-op multitask.
If the function return with False the timer will be cancelled.

\end{fulllineitems}


\end{fulllineitems}

\index{VNFStarterAPI (class in escape.util.adapter)}

\begin{fulllineitems}
\phantomsection\label{util/adapter:escape.util.adapter.VNFStarterAPI}\pysigline{\strong{class }\code{escape.util.adapter.}\bfcode{VNFStarterAPI}}
Bases: \href{https://docs.python.org/2.7/library/functions.html\#object}{\code{object}}

Define interface for managing VNFs.


\strong{See also:}


\code{vnf\_starter.yang}



Follows the MixIn design pattern approach to support VNFStarter functionality.
\index{\_\_init\_\_() (escape.util.adapter.VNFStarterAPI method)}

\begin{fulllineitems}
\phantomsection\label{util/adapter:escape.util.adapter.VNFStarterAPI.__init__}\pysiglinewithargsret{\bfcode{\_\_init\_\_}}{}{}
\end{fulllineitems}

\index{initiateVNF() (escape.util.adapter.VNFStarterAPI method)}

\begin{fulllineitems}
\phantomsection\label{util/adapter:escape.util.adapter.VNFStarterAPI.initiateVNF}\pysiglinewithargsret{\bfcode{initiateVNF}}{\emph{vnf\_type=None}, \emph{vnf\_description=None}, \emph{options=None}}{}
Initiate a VNF.
\begin{quote}\begin{description}
\item[{Parameters}] \leavevmode\begin{itemize}
\item {} 
\textbf{\texttt{vnf\_type}} (\href{https://docs.python.org/2.7/library/functions.html\#str}{\emph{str}}) -- pre-defined VNF type (see in vnf\_starter/available\_vnfs)

\item {} 
\textbf{\texttt{vnf\_description}} (\href{https://docs.python.org/2.7/library/functions.html\#str}{\emph{str}}) -- Click description if there are no pre-defined type

\item {} 
\textbf{\texttt{options}} (\href{https://docs.python.org/2.7/library/collections.html\#collections.OrderedDict}{\emph{collections.OrderedDict}}) -- unlimited list of additional options as name-value pairs

\end{itemize}

\end{description}\end{quote}

\end{fulllineitems}

\index{connectVNF() (escape.util.adapter.VNFStarterAPI method)}

\begin{fulllineitems}
\phantomsection\label{util/adapter:escape.util.adapter.VNFStarterAPI.connectVNF}\pysiglinewithargsret{\bfcode{connectVNF}}{\emph{vnf\_id}, \emph{vnf\_port}, \emph{switch\_id}}{}
Connect a VNF to a switch.
\begin{quote}\begin{description}
\item[{Parameters}] \leavevmode\begin{itemize}
\item {} 
\textbf{\texttt{vnf\_id}} (\href{https://docs.python.org/2.7/library/functions.html\#str}{\emph{str}}) -- VNF ID (mandatory)

\item {} 
\textbf{\texttt{vnf\_port}} (\href{https://docs.python.org/2.7/library/functions.html\#str}{\emph{str}}) -- VNF port (mandatory)

\item {} 
\textbf{\texttt{switch\_id}} (\href{https://docs.python.org/2.7/library/functions.html\#str}{\emph{str}}) -- switch ID (mandatory)

\end{itemize}

\item[{Returns}] \leavevmode
Returns the connected port(s) with the corresponding switch(es).

\end{description}\end{quote}

\end{fulllineitems}

\index{disconnectVNF() (escape.util.adapter.VNFStarterAPI method)}

\begin{fulllineitems}
\phantomsection\label{util/adapter:escape.util.adapter.VNFStarterAPI.disconnectVNF}\pysiglinewithargsret{\bfcode{disconnectVNF}}{\emph{vnf\_id}, \emph{vnf\_port}}{}
Disconnect VNF from a switch.
\begin{quote}\begin{description}
\item[{Parameters}] \leavevmode\begin{itemize}
\item {} 
\textbf{\texttt{vnf\_id}} (\href{https://docs.python.org/2.7/library/functions.html\#str}{\emph{str}}) -- VNF ID (mandatory)

\item {} 
\textbf{\texttt{vnf\_port}} (\href{https://docs.python.org/2.7/library/functions.html\#str}{\emph{str}}) -- VNF port (mandatory)

\end{itemize}

\item[{Returns}] \leavevmode
reply data

\end{description}\end{quote}

\end{fulllineitems}

\index{startVNF() (escape.util.adapter.VNFStarterAPI method)}

\begin{fulllineitems}
\phantomsection\label{util/adapter:escape.util.adapter.VNFStarterAPI.startVNF}\pysiglinewithargsret{\bfcode{startVNF}}{\emph{vnf\_id}}{}
Start VNF.
\begin{quote}\begin{description}
\item[{Parameters}] \leavevmode
\textbf{\texttt{vnf\_id}} (\href{https://docs.python.org/2.7/library/functions.html\#str}{\emph{str}}) -- VNF ID (mandatory)

\item[{Returns}] \leavevmode
reply data

\end{description}\end{quote}

\end{fulllineitems}

\index{stopVNF() (escape.util.adapter.VNFStarterAPI method)}

\begin{fulllineitems}
\phantomsection\label{util/adapter:escape.util.adapter.VNFStarterAPI.stopVNF}\pysiglinewithargsret{\bfcode{stopVNF}}{\emph{vnf\_id}}{}
Stop VNF.
\begin{quote}\begin{description}
\item[{Parameters}] \leavevmode
\textbf{\texttt{vnf\_id}} (\href{https://docs.python.org/2.7/library/functions.html\#str}{\emph{str}}) -- VNF ID (mandatory)

\item[{Returns}] \leavevmode
reply data

\end{description}\end{quote}

\end{fulllineitems}

\index{getVNFInfo() (escape.util.adapter.VNFStarterAPI method)}

\begin{fulllineitems}
\phantomsection\label{util/adapter:escape.util.adapter.VNFStarterAPI.getVNFInfo}\pysiglinewithargsret{\bfcode{getVNFInfo}}{\emph{vnf\_id=None}}{}
\end{fulllineitems}


\end{fulllineitems}

\index{OpenStackAPI (class in escape.util.adapter)}

\begin{fulllineitems}
\phantomsection\label{util/adapter:escape.util.adapter.OpenStackAPI}\pysigline{\strong{class }\code{escape.util.adapter.}\bfcode{OpenStackAPI}}
Bases: \href{https://docs.python.org/2.7/library/functions.html\#object}{\code{object}}

Define interface for managing OpenStack domain.

\begin{notice}{note}{Note:}
Based on separated REST API which need to be discussed!
\end{notice}

Follows the MixIn design pattern approach to support OpenStack functionality.

\end{fulllineitems}

\index{AbstractRESTAdapter (class in escape.util.adapter)}

\begin{fulllineitems}
\phantomsection\label{util/adapter:escape.util.adapter.AbstractRESTAdapter}\pysiglinewithargsret{\strong{class }\code{escape.util.adapter.}\bfcode{AbstractRESTAdapter}}{\emph{base\_url}, \emph{auth=None}}{}
Bases: \code{requests.sessions.Session}

Abstract class for various adapters rely on a RESTful API.

Contains basic functions for managing connections.

Inhereted from \href{http://docs.python-requests.org/en/latest/api/\#requests.Session}{\code{requests.Session}}. Provided features: coockie
persistence, connection-pooling and configuration.
\begin{description}
\item[{Implements Context Manager Python protocol::}] \leavevmode
\begin{Verbatim}[commandchars=\\\{\}]
\PYG{g+gp}{\PYGZgt{}\PYGZgt{}\PYGZgt{} }\PYG{k}{with} \PYG{n}{AbstractRESTAdapter} \PYG{k}{as} \PYG{n}{a}\PYG{p}{:}
\PYG{g+gp}{\PYGZgt{}\PYGZgt{}\PYGZgt{} }  \PYG{n}{a}\PYG{o}{.}\PYG{o}{\PYGZlt{}}\PYG{n}{method}\PYG{o}{\PYGZgt{}}\PYG{p}{(}\PYG{p}{)}
\end{Verbatim}

\end{description}


\strong{See also:}


\href{http://docs.python-requests.org/en/latest/api/\#requests.Session}{http://docs.python-requests.org/en/latest/api/\#requests.Session}



Follows Adapter design pattern.
\index{custom\_headers (escape.util.adapter.AbstractRESTAdapter attribute)}

\begin{fulllineitems}
\phantomsection\label{util/adapter:escape.util.adapter.AbstractRESTAdapter.custom_headers}\pysigline{\bfcode{custom\_headers}\strong{ = \{`user-agent': `ESCAPE/2.0.0'\}}}
\end{fulllineitems}

\index{\_\_init\_\_() (escape.util.adapter.AbstractRESTAdapter method)}

\begin{fulllineitems}
\phantomsection\label{util/adapter:escape.util.adapter.AbstractRESTAdapter.__init__}\pysiglinewithargsret{\bfcode{\_\_init\_\_}}{\emph{base\_url}, \emph{auth=None}}{}
\end{fulllineitems}

\index{\_send\_request() (escape.util.adapter.AbstractRESTAdapter method)}

\begin{fulllineitems}
\phantomsection\label{util/adapter:escape.util.adapter.AbstractRESTAdapter._send_request}\pysiglinewithargsret{\bfcode{\_send\_request}}{\emph{method}, \emph{url=None}, \emph{body=None}, \emph{**kwargs}}{}
Prepare the request and send it. If valid URL is given that value will be
used else it will be append to the end of the \code{base\_url}. If \code{url} is
not given only the \code{base\_url} will be used.
\begin{quote}\begin{description}
\item[{Parameters}] \leavevmode\begin{itemize}
\item {} 
\textbf{\texttt{method}} (\href{https://docs.python.org/2.7/library/functions.html\#str}{\emph{str}}) -- HTTP method

\item {} 
\textbf{\texttt{url}} (\href{https://docs.python.org/2.7/library/functions.html\#str}{\emph{str}}) -- valid URL or relevent part follows \code{self.base\_url}

\item {} 
\textbf{\texttt{body}} ({\hyperref[util/nffg:escape.util.nffg.NFFG]{\emph{\code{NFFG}}}} or dict or bytes or str) -- request body

\item {} 
\textbf{\texttt{kwargs}} -- additional params. See \href{http://docs.python-requests.org/en/latest/api/\#requests.Session.request}{\code{requests.Session.request}}

\end{itemize}

\item[{Returns}] \leavevmode
response text as JSON

\item[{Return type}] \leavevmode
\href{https://docs.python.org/2.7/library/functions.html\#str}{str}

\item[{Raises}] \leavevmode\begin{itemize}
\item {} 
\textbf{\texttt{HTTPError}} -- if responde code is between 400 and 600

\item {} 
\textbf{\texttt{ConnectionError}} -- connection error

\item {} 
\textbf{\texttt{Timeout}} -- many error occured when request timed out

\end{itemize}

\end{description}\end{quote}

\end{fulllineitems}


\end{fulllineitems}



\subparagraph{\emph{api.py} module}
\label{util/api:api-py-module}\label{util/api::doc}
{\hyperref[util/api:escape.util.api.AbstractAPI]{\emph{\code{AbstractAPI}}}} contains the register mechanism into the POX core for
layer APIs, the event handling/registering logic and defines the general
functions for initialization and finalization steps

{\hyperref[util/api:escape.util.api.RESTServer]{\emph{\code{RESTServer}}}} is a general HTTP server which parse HTTP request and
forward to explicitly given request handler

{\hyperref[util/api:escape.util.api.AbstractRequestHandler]{\emph{\code{AbstractRequestHandler}}}} is a base class for concrete request handling.
It implements the general URL and request body parsing functions


\subparagraph{Module contents}
\label{util/api:module-contents}\label{util/api:module-escape.util.api}\index{escape.util.api (module)}
Contains abstract classes for concrete layer API modules
\index{AbstractAPI (class in escape.util.api)}

\begin{fulllineitems}
\phantomsection\label{util/api:escape.util.api.AbstractAPI}\pysiglinewithargsret{\strong{class }\code{escape.util.api.}\bfcode{AbstractAPI}}{\emph{standalone=False}, \emph{**kwargs}}{}
Bases: \code{pox.lib.revent.revent.EventMixin}

Abstract class for UNIFY's API

Contain common functions

Follows Facade design pattern -\textgreater{} simplified enty/exit point ot the layers
\index{\_core\_name (escape.util.api.AbstractAPI attribute)}

\begin{fulllineitems}
\phantomsection\label{util/api:escape.util.api.AbstractAPI._core_name}\pysigline{\bfcode{\_core\_name}\strong{ = `AbstractAPI'}}
\end{fulllineitems}

\index{dependencies (escape.util.api.AbstractAPI attribute)}

\begin{fulllineitems}
\phantomsection\label{util/api:escape.util.api.AbstractAPI.dependencies}\pysigline{\bfcode{dependencies}\strong{ = ()}}
\end{fulllineitems}

\index{\_\_init\_\_() (escape.util.api.AbstractAPI method)}

\begin{fulllineitems}
\phantomsection\label{util/api:escape.util.api.AbstractAPI.__init__}\pysiglinewithargsret{\bfcode{\_\_init\_\_}}{\emph{standalone=False}, \emph{**kwargs}}{}
Abstract class constructor

Handle core registration along with {\hyperref[util/api:escape.util.api.AbstractAPI._all_dependencies_met]{\emph{\code{\_all\_dependencies\_met()}}}}

Set given parameters (standalone parameter is mandatory) automatically as:

\begin{Verbatim}[commandchars=\\\{\}]
self.\PYGZus{}\PYGZlt{}param\PYGZus{}name\PYGZgt{} = \PYGZlt{}param\PYGZus{}value\PYGZgt{}
\end{Verbatim}

Base constructor functions have to be called as the last step in derived
classes. Same situation with {\hyperref[util/api:escape.util.api.AbstractAPI._all_dependencies_met]{\emph{\code{\_all\_dependencies\_met()}}}} respectively.
Must not override these function, just use {\hyperref[util/api:escape.util.api.AbstractAPI.initialize]{\emph{\code{initialize()}}}} for
init steps. Actual API classes must only call \href{https://docs.python.org/2.7/library/functions.html\#super}{\code{super()}} in their
constructor with the form:

\begin{Verbatim}[commandchars=\\\{\}]
super(\PYGZlt{}API Class name\PYGZgt{}, self).\PYGZus{}\PYGZus{}init\PYGZus{}\PYGZus{}(standalone=standalone, **kwargs)
\end{Verbatim}

\begin{notice}{warning}{Warning:}
Do not use prefixes in the name of event handlers, because of automatic
dependency discovery considers that as a dependent component and this
situation cause a dead lock (component will be waiting to each other to
set up)
\end{notice}
\begin{quote}\begin{description}
\item[{Parameters}] \leavevmode
\textbf{\texttt{standalone}} (\href{https://docs.python.org/2.7/library/functions.html\#bool}{\emph{bool}}) -- started in standalone mode or not

\end{description}\end{quote}

\end{fulllineitems}

\index{\_all\_dependencies\_met() (escape.util.api.AbstractAPI method)}

\begin{fulllineitems}
\phantomsection\label{util/api:escape.util.api.AbstractAPI._all_dependencies_met}\pysiglinewithargsret{\bfcode{\_all\_dependencies\_met}}{}{}
Called when every component on which depends are initialized on POX core

Contain dependency relevant initialization.
\begin{quote}\begin{description}
\item[{Returns}] \leavevmode
None

\end{description}\end{quote}

\end{fulllineitems}

\index{initialize() (escape.util.api.AbstractAPI method)}

\begin{fulllineitems}
\phantomsection\label{util/api:escape.util.api.AbstractAPI.initialize}\pysiglinewithargsret{\bfcode{initialize}}{}{}
Init function for child API classes to simplify dynamic initialization

Called when every component on which depends are initialized and registered
in POX core

This function should be overwritten by child classes.
\begin{quote}\begin{description}
\item[{Returns}] \leavevmode
None

\end{description}\end{quote}

\end{fulllineitems}

\index{shutdown() (escape.util.api.AbstractAPI method)}

\begin{fulllineitems}
\phantomsection\label{util/api:escape.util.api.AbstractAPI.shutdown}\pysiglinewithargsret{\bfcode{shutdown}}{\emph{event}}{}
Finalization, deallocation, etc. of actual component

Should be overwritten by child classes
\begin{quote}\begin{description}
\item[{Parameters}] \leavevmode
\textbf{\texttt{event}} (\emph{GoingDownEvent}) -- shutdown event raised by POX core

\item[{Returns}] \leavevmode
None

\end{description}\end{quote}

\end{fulllineitems}

\index{\_read\_json\_from\_file() (escape.util.api.AbstractAPI static method)}

\begin{fulllineitems}
\phantomsection\label{util/api:escape.util.api.AbstractAPI._read_json_from_file}\pysiglinewithargsret{\strong{static }\bfcode{\_read\_json\_from\_file}}{\emph{graph\_file}}{}
Read the given file and return a string formatted as JSON
\begin{quote}\begin{description}
\item[{Parameters}] \leavevmode
\textbf{\texttt{graph\_file}} (\href{https://docs.python.org/2.7/library/functions.html\#str}{\emph{str}}) -- file path

\item[{Returns}] \leavevmode
JSON data

\item[{Return type}] \leavevmode
\href{https://docs.python.org/2.7/library/functions.html\#str}{str}

\end{description}\end{quote}

\end{fulllineitems}

\index{\_\_str\_\_() (escape.util.api.AbstractAPI method)}

\begin{fulllineitems}
\phantomsection\label{util/api:escape.util.api.AbstractAPI.__str__}\pysiglinewithargsret{\bfcode{\_\_str\_\_}}{}{}
Print class type and non-private attributes with their types for debugging
\begin{quote}\begin{description}
\item[{Returns}] \leavevmode
specific string

\item[{Return type}] \leavevmode
\href{https://docs.python.org/2.7/library/functions.html\#str}{str}

\end{description}\end{quote}

\end{fulllineitems}


\end{fulllineitems}

\index{RESTServer (class in escape.util.api)}

\begin{fulllineitems}
\phantomsection\label{util/api:escape.util.api.RESTServer}\pysiglinewithargsret{\strong{class }\code{escape.util.api.}\bfcode{RESTServer}}{\emph{RequestHandlerClass}, \emph{address=`127.0.0.1'}, \emph{port=8008}}{}
Bases: \href{https://docs.python.org/2.7/library/basehttpserver.html\#BaseHTTPServer.HTTPServer}{\code{BaseHTTPServer.HTTPServer}}, \code{SocketServer.ThreadingMixIn}

Base HTTP server for RESTful API

Initiate an \code{HTTPServer} and run it in different thread
\index{\_\_init\_\_() (escape.util.api.RESTServer method)}

\begin{fulllineitems}
\phantomsection\label{util/api:escape.util.api.RESTServer.__init__}\pysiglinewithargsret{\bfcode{\_\_init\_\_}}{\emph{RequestHandlerClass}, \emph{address=`127.0.0.1'}, \emph{port=8008}}{}
Set up an \href{https://docs.python.org/2.7/library/basehttpserver.html\#BaseHTTPServer.HTTPServer}{\code{BaseHTTPServer.HTTPServer}} in a different
thread
\begin{quote}\begin{description}
\item[{Parameters}] \leavevmode\begin{itemize}
\item {} 
\textbf{\texttt{RequestHandlerClass}} ({\hyperref[util/api:escape.util.api.AbstractRequestHandler]{\emph{\emph{AbstractRequestHandler}}}}) -- Class of a handler which handles HTTP request

\item {} 
\textbf{\texttt{address}} (\href{https://docs.python.org/2.7/library/functions.html\#str}{\emph{str}}) -- Used IP address

\item {} 
\textbf{\texttt{port}} (\href{https://docs.python.org/2.7/library/functions.html\#int}{\emph{int}}) -- Used port number

\end{itemize}

\end{description}\end{quote}

\end{fulllineitems}

\index{start() (escape.util.api.RESTServer method)}

\begin{fulllineitems}
\phantomsection\label{util/api:escape.util.api.RESTServer.start}\pysiglinewithargsret{\bfcode{start}}{}{}
Start RESTServer thread

\end{fulllineitems}

\index{stop() (escape.util.api.RESTServer method)}

\begin{fulllineitems}
\phantomsection\label{util/api:escape.util.api.RESTServer.stop}\pysiglinewithargsret{\bfcode{stop}}{}{}
Stop RESTServer thread

\end{fulllineitems}

\index{run() (escape.util.api.RESTServer method)}

\begin{fulllineitems}
\phantomsection\label{util/api:escape.util.api.RESTServer.run}\pysiglinewithargsret{\bfcode{run}}{}{}
Handle one request at a time until shutdown.

\end{fulllineitems}


\end{fulllineitems}

\index{RESTError}

\begin{fulllineitems}
\phantomsection\label{util/api:escape.util.api.RESTError}\pysiglinewithargsret{\strong{exception }\code{escape.util.api.}\bfcode{RESTError}}{\emph{msg=None}, \emph{code=0}}{}
Bases: \href{https://docs.python.org/2.7/library/exceptions.html\#exceptions.Exception}{\code{exceptions.Exception}}

Exception class for REST errors
\index{\_\_init\_\_() (escape.util.api.RESTError method)}

\begin{fulllineitems}
\phantomsection\label{util/api:escape.util.api.RESTError.__init__}\pysiglinewithargsret{\bfcode{\_\_init\_\_}}{\emph{msg=None}, \emph{code=0}}{}
\end{fulllineitems}

\index{msg (escape.util.api.RESTError attribute)}

\begin{fulllineitems}
\phantomsection\label{util/api:escape.util.api.RESTError.msg}\pysigline{\bfcode{msg}}
\end{fulllineitems}

\index{code (escape.util.api.RESTError attribute)}

\begin{fulllineitems}
\phantomsection\label{util/api:escape.util.api.RESTError.code}\pysigline{\bfcode{code}}
\end{fulllineitems}

\index{\_\_str\_\_() (escape.util.api.RESTError method)}

\begin{fulllineitems}
\phantomsection\label{util/api:escape.util.api.RESTError.__str__}\pysiglinewithargsret{\bfcode{\_\_str\_\_}}{}{}
\end{fulllineitems}


\end{fulllineitems}

\index{AbstractRequestHandler (class in escape.util.api)}

\begin{fulllineitems}
\phantomsection\label{util/api:escape.util.api.AbstractRequestHandler}\pysiglinewithargsret{\strong{class }\code{escape.util.api.}\bfcode{AbstractRequestHandler}}{\emph{request}, \emph{client\_address}, \emph{server}}{}
Bases: \href{https://docs.python.org/2.7/library/basehttpserver.html\#BaseHTTPServer.BaseHTTPRequestHandler}{\code{BaseHTTPServer.BaseHTTPRequestHandler}}

Minimalistic RESTful API for Layer APIs

Handle /escape/* URLs.

Method calling permissions represented in escape\_intf dictionary

\begin{notice}{warning}{Warning:}
This class is out of the context of the recoco's co-operative thread
context! While you don't need to worry much about synchronization between
recoco tasks, you do need to think about synchronization between recoco task
and normal threads. Synchronisation is needed to take care manually - use
relevant helper function of core object: \code{callLater()}/
\code{raiseLater()} or use {\hyperref[util/misc:escape.util.misc.schedule_as_coop_task]{\emph{\code{schedule\_as\_coop\_task()}}}} decorator defined in
{\hyperref[util/misc:module-escape.util.misc]{\emph{\code{escape.util.misc}}}} on the called function
\end{notice}
\index{server\_version (escape.util.api.AbstractRequestHandler attribute)}

\begin{fulllineitems}
\phantomsection\label{util/api:escape.util.api.AbstractRequestHandler.server_version}\pysigline{\bfcode{server\_version}\strong{ = `ESCAPE/2.0.0'}}
\end{fulllineitems}

\index{static\_prefix (escape.util.api.AbstractRequestHandler attribute)}

\begin{fulllineitems}
\phantomsection\label{util/api:escape.util.api.AbstractRequestHandler.static_prefix}\pysigline{\bfcode{static\_prefix}\strong{ = `escape'}}
\end{fulllineitems}

\index{request\_perm (escape.util.api.AbstractRequestHandler attribute)}

\begin{fulllineitems}
\phantomsection\label{util/api:escape.util.api.AbstractRequestHandler.request_perm}\pysigline{\bfcode{request\_perm}\strong{ = \{`PUT': (), `POST': (), `DELETE': (), `GET': ()\}}}
\end{fulllineitems}

\index{bounded\_layer (escape.util.api.AbstractRequestHandler attribute)}

\begin{fulllineitems}
\phantomsection\label{util/api:escape.util.api.AbstractRequestHandler.bounded_layer}\pysigline{\bfcode{bounded\_layer}\strong{ = None}}
\end{fulllineitems}

\index{log (escape.util.api.AbstractRequestHandler attribute)}

\begin{fulllineitems}
\phantomsection\label{util/api:escape.util.api.AbstractRequestHandler.log}\pysigline{\bfcode{log}\strong{ = \textless{}logging.Logger object\textgreater{}}}
\end{fulllineitems}

\index{do\_GET() (escape.util.api.AbstractRequestHandler method)}

\begin{fulllineitems}
\phantomsection\label{util/api:escape.util.api.AbstractRequestHandler.do_GET}\pysiglinewithargsret{\bfcode{do\_GET}}{}{}
Get information about an entity. R for CRUD convention.

\end{fulllineitems}

\index{do\_POST() (escape.util.api.AbstractRequestHandler method)}

\begin{fulllineitems}
\phantomsection\label{util/api:escape.util.api.AbstractRequestHandler.do_POST}\pysiglinewithargsret{\bfcode{do\_POST}}{}{}
Create an entity. C for CRUD convention.

\end{fulllineitems}

\index{do\_PUT() (escape.util.api.AbstractRequestHandler method)}

\begin{fulllineitems}
\phantomsection\label{util/api:escape.util.api.AbstractRequestHandler.do_PUT}\pysiglinewithargsret{\bfcode{do\_PUT}}{}{}
Update an entity. U for CRUD convention.

\end{fulllineitems}

\index{do\_DELETE() (escape.util.api.AbstractRequestHandler method)}

\begin{fulllineitems}
\phantomsection\label{util/api:escape.util.api.AbstractRequestHandler.do_DELETE}\pysiglinewithargsret{\bfcode{do\_DELETE}}{}{}
Delete an entity. D for CRUD convention.

\end{fulllineitems}

\index{do\_OPTIONS() (escape.util.api.AbstractRequestHandler method)}

\begin{fulllineitems}
\phantomsection\label{util/api:escape.util.api.AbstractRequestHandler.do_OPTIONS}\pysiglinewithargsret{\bfcode{do\_OPTIONS}}{}{}
Handling unsupported HTTP verbs
\begin{quote}\begin{description}
\item[{Returns}] \leavevmode
None

\end{description}\end{quote}

\end{fulllineitems}

\index{do\_HEAD() (escape.util.api.AbstractRequestHandler method)}

\begin{fulllineitems}
\phantomsection\label{util/api:escape.util.api.AbstractRequestHandler.do_HEAD}\pysiglinewithargsret{\bfcode{do\_HEAD}}{}{}
Handling unsupported HTTP verbs
\begin{quote}\begin{description}
\item[{Returns}] \leavevmode
None

\end{description}\end{quote}

\end{fulllineitems}

\index{do\_TRACE() (escape.util.api.AbstractRequestHandler method)}

\begin{fulllineitems}
\phantomsection\label{util/api:escape.util.api.AbstractRequestHandler.do_TRACE}\pysiglinewithargsret{\bfcode{do\_TRACE}}{}{}
Handling unsupported HTTP verbs
\begin{quote}\begin{description}
\item[{Returns}] \leavevmode
None

\end{description}\end{quote}

\end{fulllineitems}

\index{do\_CONNECT() (escape.util.api.AbstractRequestHandler method)}

\begin{fulllineitems}
\phantomsection\label{util/api:escape.util.api.AbstractRequestHandler.do_CONNECT}\pysiglinewithargsret{\bfcode{do\_CONNECT}}{}{}
Handling unsupported HTTP verbs
\begin{quote}\begin{description}
\item[{Returns}] \leavevmode
None

\end{description}\end{quote}

\end{fulllineitems}

\index{\_process\_url() (escape.util.api.AbstractRequestHandler method)}

\begin{fulllineitems}
\phantomsection\label{util/api:escape.util.api.AbstractRequestHandler._process_url}\pysiglinewithargsret{\bfcode{\_process\_url}}{}{}
Split HTTP path and call the carved function if it is defined in this class
and in request\_perm
\begin{quote}\begin{description}
\item[{Returns}] \leavevmode
None

\end{description}\end{quote}

\end{fulllineitems}

\index{\_parse\_json\_body() (escape.util.api.AbstractRequestHandler method)}

\begin{fulllineitems}
\phantomsection\label{util/api:escape.util.api.AbstractRequestHandler._parse_json_body}\pysiglinewithargsret{\bfcode{\_parse\_json\_body}}{}{}
Parse HTTP request body in JSON format

\begin{notice}{note}{Note:}
Call only once by HTTP request
\end{notice}

\begin{notice}{note}{Note:}
Parsed JSON object is Unicode
\end{notice}

GET, DELETE messages don't have body - return empty dict by default
\begin{quote}\begin{description}
\item[{Returns}] \leavevmode
request body in JSON format

\item[{Return type}] \leavevmode
\href{https://docs.python.org/2.7/library/stdtypes.html\#dict}{dict}

\end{description}\end{quote}

\end{fulllineitems}

\index{send\_REST\_headers() (escape.util.api.AbstractRequestHandler method)}

\begin{fulllineitems}
\phantomsection\label{util/api:escape.util.api.AbstractRequestHandler.send_REST_headers}\pysiglinewithargsret{\bfcode{send\_REST\_headers}}{}{}
Set the allowed REST verbs as an HTTP header (Allow)
\begin{quote}\begin{description}
\item[{Returns}] \leavevmode
None

\end{description}\end{quote}

\end{fulllineitems}

\index{send\_acknowledge() (escape.util.api.AbstractRequestHandler method)}

\begin{fulllineitems}
\phantomsection\label{util/api:escape.util.api.AbstractRequestHandler.send_acknowledge}\pysiglinewithargsret{\bfcode{send\_acknowledge}}{\emph{msg='\{``result'': ``Accepted''\}'}}{}
Send back acknowlede message
\begin{quote}\begin{description}
\item[{Parameters}] \leavevmode\begin{itemize}
\item {} 
\textbf{\texttt{msg}} -- response body

\item {} 
\textbf{\texttt{msg}} -- dict

\end{itemize}

\item[{Returns}] \leavevmode
None

\end{description}\end{quote}

\end{fulllineitems}

\index{\_send\_json\_response() (escape.util.api.AbstractRequestHandler method)}

\begin{fulllineitems}
\phantomsection\label{util/api:escape.util.api.AbstractRequestHandler._send_json_response}\pysiglinewithargsret{\bfcode{\_send\_json\_response}}{\emph{data}, \emph{encoding='utf-8'}}{}
Send requested data in JSON format
\begin{quote}\begin{description}
\item[{Parameters}] \leavevmode\begin{itemize}
\item {} 
\textbf{\texttt{data}} (\href{https://docs.python.org/2.7/library/stdtypes.html\#dict}{\emph{dict}}) -- data in JSON format

\item {} 
\textbf{\texttt{encoding}} (\href{https://docs.python.org/2.7/library/functions.html\#str}{\emph{str}}) -- Set data encoding (optional)

\end{itemize}

\item[{Returns}] \leavevmode
None

\end{description}\end{quote}

\end{fulllineitems}

\index{error\_content\_type (escape.util.api.AbstractRequestHandler attribute)}

\begin{fulllineitems}
\phantomsection\label{util/api:escape.util.api.AbstractRequestHandler.error_content_type}\pysigline{\bfcode{error\_content\_type}\strong{ = `text/json'}}
\end{fulllineitems}

\index{send\_error() (escape.util.api.AbstractRequestHandler method)}

\begin{fulllineitems}
\phantomsection\label{util/api:escape.util.api.AbstractRequestHandler.send_error}\pysiglinewithargsret{\bfcode{send\_error}}{\emph{code}, \emph{message=None}}{}
Override original function to send back allowed HTTP verbs and set format
to JSON

\end{fulllineitems}

\index{log\_error() (escape.util.api.AbstractRequestHandler method)}

\begin{fulllineitems}
\phantomsection\label{util/api:escape.util.api.AbstractRequestHandler.log_error}\pysiglinewithargsret{\bfcode{log\_error}}{\emph{mformat}, \emph{*args}}{}
Overwritten to use POX logging mechanism

\end{fulllineitems}

\index{log\_message() (escape.util.api.AbstractRequestHandler method)}

\begin{fulllineitems}
\phantomsection\label{util/api:escape.util.api.AbstractRequestHandler.log_message}\pysiglinewithargsret{\bfcode{log\_message}}{\emph{mformat}, \emph{*args}}{}
Disable logging of incoming messages

\end{fulllineitems}

\index{log\_full\_message() (escape.util.api.AbstractRequestHandler method)}

\begin{fulllineitems}
\phantomsection\label{util/api:escape.util.api.AbstractRequestHandler.log_full_message}\pysiglinewithargsret{\bfcode{log\_full\_message}}{\emph{mformat}, \emph{*args}}{}
Overwritten to use POX logging mechanism

\end{fulllineitems}

\index{\_proceed\_API\_call() (escape.util.api.AbstractRequestHandler method)}

\begin{fulllineitems}
\phantomsection\label{util/api:escape.util.api.AbstractRequestHandler._proceed_API_call}\pysiglinewithargsret{\bfcode{\_proceed\_API\_call}}{\emph{function}, \emph{*args}, \emph{**kwargs}}{}
Fail-safe method to call API function

The cooperative micro-task context is handled by actual APIs

Should call this with params, not directly the function of actual API
\begin{quote}\begin{description}
\item[{Parameters}] \leavevmode\begin{itemize}
\item {} 
\textbf{\texttt{function}} (\href{https://docs.python.org/2.7/library/functions.html\#str}{\emph{str}}) -- function name

\item {} 
\textbf{\texttt{args}} (\href{https://docs.python.org/2.7/library/functions.html\#tuple}{\emph{tuple}}) -- Optional params

\item {} 
\textbf{\texttt{kwargs}} (\href{https://docs.python.org/2.7/library/stdtypes.html\#dict}{\emph{dict}}) -- Optional named params

\end{itemize}

\item[{Returns}] \leavevmode
None

\end{description}\end{quote}

\end{fulllineitems}


\end{fulllineitems}



\subparagraph{\emph{mapping.py} module}
\label{util/mapping:mapping-py-module}\label{util/mapping::doc}
{\hyperref[util/mapping:escape.util.mapping.AbstractMapper]{\emph{\code{AbstractMapper}}}} is an abstract class for orchestration method which
should implement mapping preparations and invoke actual mapping algorithm

{\hyperref[util/mapping:escape.util.mapping.AbstractMappingStrategy]{\emph{\code{AbstractMappingStrategy}}}} is an abstract class for containing entirely
the mapping algorithm as a class method


\subparagraph{Module contents}
\label{util/mapping:module-escape.util.mapping}\label{util/mapping:module-contents}\index{escape.util.mapping (module)}
Contains abstract classes for NFFG mapping
\index{AbstractMappingStrategy (class in escape.util.mapping)}

\begin{fulllineitems}
\phantomsection\label{util/mapping:escape.util.mapping.AbstractMappingStrategy}\pysigline{\strong{class }\code{escape.util.mapping.}\bfcode{AbstractMappingStrategy}}
Bases: \href{https://docs.python.org/2.7/library/functions.html\#object}{\code{object}}

Abstract class for the mapping strategies

Follows the Strategy design pattern
\index{\_\_init\_\_() (escape.util.mapping.AbstractMappingStrategy method)}

\begin{fulllineitems}
\phantomsection\label{util/mapping:escape.util.mapping.AbstractMappingStrategy.__init__}\pysiglinewithargsret{\bfcode{\_\_init\_\_}}{}{}
Init

\end{fulllineitems}

\index{map() (escape.util.mapping.AbstractMappingStrategy class method)}

\begin{fulllineitems}
\phantomsection\label{util/mapping:escape.util.mapping.AbstractMappingStrategy.map}\pysiglinewithargsret{\strong{classmethod }\bfcode{map}}{\emph{graph}, \emph{resource}}{}
Abstract function for mapping algorithm

\begin{notice}{warning}{Warning:}
Derived class have to override this function
\end{notice}
\begin{quote}\begin{description}
\item[{Parameters}] \leavevmode\begin{itemize}
\item {} 
\textbf{\texttt{graph}} ({\hyperref[util/nffg:escape.util.nffg.NFFG]{\emph{\emph{NFFG}}}}) -- Input graph which need to be mapped

\item {} 
\textbf{\texttt{resource}} ({\hyperref[util/nffg:escape.util.nffg.NFFG]{\emph{\emph{NFFG}}}}) -- resource info

\end{itemize}

\item[{Raise}] \leavevmode
NotImplementedError

\item[{Returns}] \leavevmode
mapped graph

\item[{Return type}] \leavevmode
{\hyperref[util/nffg:escape.util.nffg.NFFG]{\emph{NFFG}}}

\end{description}\end{quote}

\end{fulllineitems}


\end{fulllineitems}

\index{AbstractMapper (class in escape.util.mapping)}

\begin{fulllineitems}
\phantomsection\label{util/mapping:escape.util.mapping.AbstractMapper}\pysiglinewithargsret{\strong{class }\code{escape.util.mapping.}\bfcode{AbstractMapper}}{\emph{layer\_name}, \emph{strategy=None}, \emph{threaded=None}}{}
Bases: \code{pox.lib.revent.revent.EventMixin}

Abstract class for graph mapping function

Inherited from :class{}`EventMixin{}` to implement internal event-based
communication

Contain common functions and initialization
\index{\_defaults (escape.util.mapping.AbstractMapper attribute)}

\begin{fulllineitems}
\phantomsection\label{util/mapping:escape.util.mapping.AbstractMapper._defaults}\pysigline{\bfcode{\_defaults}\strong{ = \{`orchestration': `ESCAPEMappingStrategy', `service': `DefaultServiceMappingStrategy'\}}}
\end{fulllineitems}

\index{\_\_init\_\_() (escape.util.mapping.AbstractMapper method)}

\begin{fulllineitems}
\phantomsection\label{util/mapping:escape.util.mapping.AbstractMapper.__init__}\pysiglinewithargsret{\bfcode{\_\_init\_\_}}{\emph{layer\_name}, \emph{strategy=None}, \emph{threaded=None}}{}
Initialize Mapper class

Set given strategy class and threaded value or check in \emph{CONFIG}

If no valid value is found for arguments set the default params defined
in \emph{\_default}

\begin{notice}{warning}{Warning:}
Strategy classes must be a subclass of AbstractMappingStrategy
\end{notice}

\begin{notice}{note}{Note:}
SAS strategy is searched in {\hyperref[service/sas_mapping:module-escape.service.sas_mapping]{\emph{\code{escape.service.sas\_mapping}}}}
\end{notice}

\begin{notice}{note}{Note:}
ROS strategy is searched in {\hyperref[orchest/ros_mapping:module-escape.orchest.ros_mapping]{\emph{\code{escape.orchest.ros\_mapping}}}}
\end{notice}
\begin{quote}\begin{description}
\item[{Parameters}] \leavevmode\begin{itemize}
\item {} 
\textbf{\texttt{layer\_name}} (\href{https://docs.python.org/2.7/library/functions.html\#str}{\emph{str}}) -- name of the layer which initialize this class. This
value is used to search the layer configuration in \emph{CONFIG}

\item {} 
\textbf{\texttt{strategy}} ({\hyperref[util/mapping:escape.util.mapping.AbstractMappingStrategy]{\emph{\emph{AbstractMappingStrategy}}}}) -- strategy class (optional)

\item {} 
\textbf{\texttt{threaded}} (\href{https://docs.python.org/2.7/library/functions.html\#bool}{\emph{bool}}) -- run mapping algorithm in separate Python thread instead
of in the coop microtask environment (optional)

\end{itemize}

\item[{Returns}] \leavevmode
None

\end{description}\end{quote}

\end{fulllineitems}

\index{orchestrate() (escape.util.mapping.AbstractMapper method)}

\begin{fulllineitems}
\phantomsection\label{util/mapping:escape.util.mapping.AbstractMapper.orchestrate}\pysiglinewithargsret{\bfcode{orchestrate}}{\emph{input\_graph}, \emph{resource\_view}}{}
Abstract function for wrapping optional steps connected to orchestration

Implemented function call the mapping algorithm

\begin{notice}{warning}{Warning:}
Derived class have to override this function
\end{notice}
\begin{quote}\begin{description}
\item[{Parameters}] \leavevmode\begin{itemize}
\item {} 
\textbf{\texttt{input\_graph}} ({\hyperref[util/nffg:escape.util.nffg.NFFG]{\emph{\emph{NFFG}}}}) -- graph representation which need to be mapped

\item {} 
\textbf{\texttt{resource\_view}} ({\hyperref[orchest/virtualization_mgmt:escape.orchest.virtualization_mgmt.AbstractVirtualizer]{\emph{\emph{AbstractVirtualizer}}}}) -- resource information

\end{itemize}

\item[{Raise}] \leavevmode
NotImplementedError

\item[{Returns}] \leavevmode
mapped graph

\item[{Return type}] \leavevmode
{\hyperref[util/nffg:escape.util.nffg.NFFG]{\emph{NFFG}}}

\end{description}\end{quote}

\end{fulllineitems}

\index{\_start\_mapping() (escape.util.mapping.AbstractMapper method)}

\begin{fulllineitems}
\phantomsection\label{util/mapping:escape.util.mapping.AbstractMapper._start_mapping}\pysiglinewithargsret{\bfcode{\_start\_mapping}}{\emph{graph}, \emph{resource}}{}
Run mapping algorithm in a separate Python thread
\begin{quote}\begin{description}
\item[{Parameters}] \leavevmode\begin{itemize}
\item {} 
\textbf{\texttt{graph}} ({\hyperref[util/nffg:escape.util.nffg.NFFG]{\emph{\emph{NFFG}}}}) -- Network Function Forwarding Graph

\item {} 
\textbf{\texttt{resource}} ({\hyperref[util/nffg:escape.util.nffg.NFFG]{\emph{\emph{NFFG}}}}) -- global resource

\end{itemize}

\item[{Returns}] \leavevmode
None

\end{description}\end{quote}

\end{fulllineitems}

\index{\_mapping\_finished() (escape.util.mapping.AbstractMapper method)}

\begin{fulllineitems}
\phantomsection\label{util/mapping:escape.util.mapping.AbstractMapper._mapping_finished}\pysiglinewithargsret{\bfcode{\_mapping\_finished}}{\emph{nffg}}{}
Called from a separate thread when the mapping process is finished

\begin{notice}{warning}{Warning:}
Derived class have to override this function
\end{notice}
\begin{quote}\begin{description}
\item[{Parameters}] \leavevmode
\textbf{\texttt{nffg}} ({\hyperref[util/nffg:escape.util.nffg.NFFG]{\emph{\emph{NFFG}}}}) -- generated NF-FG

\item[{Returns}] \leavevmode
None

\end{description}\end{quote}

\end{fulllineitems}


\end{fulllineitems}



\subparagraph{\emph{misc.py} module}
\label{util/misc:misc-py-module}\label{util/misc::doc}
{\hyperref[util/misc:escape.util.misc.schedule_as_coop_task]{\emph{\code{schedule\_as\_coop\_task()}}}} helps invoking a function in POX's cooperative
microtask environment

{\hyperref[util/misc:escape.util.misc.call_as_coop_task]{\emph{\code{call\_as\_coop\_task()}}}} hides POC core functionality and schedule a
function in the coop microtask environment directly

{\hyperref[util/misc:escape.util.misc.SimpleStandaloneHelper]{\emph{\code{SimpleStandaloneHelper}}}} is a helper class for mimic a minimal layer
API as a dependency for other layer APIs to handles events

{\hyperref[util/misc:escape.util.misc.enum]{\emph{\code{enum()}}}} is a helper function to generate Pythonic enumeration

{\hyperref[util/misc:escape.util.misc.ESCAPEConfig]{\emph{\code{ESCAPEConfig}}}} is a wrapper class for config


\subparagraph{Module contents}
\label{util/misc:module-contents}\label{util/misc:module-escape.util.misc}\index{escape.util.misc (module)}
Contains miscellaneous helper functions
\index{schedule\_as\_coop\_task() (in module escape.util.misc)}

\begin{fulllineitems}
\phantomsection\label{util/misc:escape.util.misc.schedule_as_coop_task}\pysiglinewithargsret{\code{escape.util.misc.}\bfcode{schedule\_as\_coop\_task}}{\emph{func}}{}
Decorator functions for running functions in an asynchronous way as a
microtask in recoco's cooperative multitasking context (in which POX was
written)
\begin{quote}\begin{description}
\item[{Parameters}] \leavevmode
\textbf{\texttt{func}} (\emph{func}) -- decorated function

\item[{Returns}] \leavevmode
decorator function

\item[{Return type}] \leavevmode
func

\end{description}\end{quote}

\end{fulllineitems}

\index{call\_as\_coop\_task() (in module escape.util.misc)}

\begin{fulllineitems}
\phantomsection\label{util/misc:escape.util.misc.call_as_coop_task}\pysiglinewithargsret{\code{escape.util.misc.}\bfcode{call\_as\_coop\_task}}{\emph{func}, \emph{*args}, \emph{**kwargs}}{}
Schedule a coop microtask and run the given function with parameters in it

Use POX core logic directly
\begin{quote}\begin{description}
\item[{Parameters}] \leavevmode\begin{itemize}
\item {} 
\textbf{\texttt{func}} (\emph{func}) -- function need to run

\item {} 
\textbf{\texttt{args}} (\href{https://docs.python.org/2.7/library/functions.html\#tuple}{\emph{tuple}}) -- nameless arguments

\item {} 
\textbf{\texttt{kwargs}} (\href{https://docs.python.org/2.7/library/stdtypes.html\#dict}{\emph{dict}}) -- named arguments

\end{itemize}

\item[{Returns}] \leavevmode
None

\end{description}\end{quote}

\end{fulllineitems}

\index{enum() (in module escape.util.misc)}

\begin{fulllineitems}
\phantomsection\label{util/misc:escape.util.misc.enum}\pysiglinewithargsret{\code{escape.util.misc.}\bfcode{enum}}{\emph{*sequential}, \emph{**named}}{}
Helper function to define enumeration. E.g.:

\begin{Verbatim}[commandchars=\\\{\}]
\PYG{g+gp}{\PYGZgt{}\PYGZgt{}\PYGZgt{} }\PYG{n}{Numbers} \PYG{o}{=} \PYG{n}{enum}\PYG{p}{(}\PYG{n}{ONE}\PYG{o}{=}\PYG{l+m+mi}{1}\PYG{p}{,} \PYG{n}{TWO}\PYG{o}{=}\PYG{l+m+mi}{2}\PYG{p}{,} \PYG{n}{THREE}\PYG{o}{=}\PYG{l+s}{\PYGZsq{}}\PYG{l+s}{three}\PYG{l+s}{\PYGZsq{}}\PYG{p}{)}
\PYG{g+gp}{\PYGZgt{}\PYGZgt{}\PYGZgt{} }\PYG{n}{Numbers} \PYG{o}{=} \PYG{n}{enum}\PYG{p}{(}\PYG{l+s}{\PYGZsq{}}\PYG{l+s}{ZERO}\PYG{l+s}{\PYGZsq{}}\PYG{p}{,} \PYG{l+s}{\PYGZsq{}}\PYG{l+s}{ONE}\PYG{l+s}{\PYGZsq{}}\PYG{p}{,} \PYG{l+s}{\PYGZsq{}}\PYG{l+s}{TWO}\PYG{l+s}{\PYGZsq{}}\PYG{p}{)}
\PYG{g+gp}{\PYGZgt{}\PYGZgt{}\PYGZgt{} }\PYG{n}{Numbers}\PYG{o}{.}\PYG{n}{ONE}
\PYG{g+go}{1}
\PYG{g+gp}{\PYGZgt{}\PYGZgt{}\PYGZgt{} }\PYG{n}{Numbers}\PYG{o}{.}\PYG{n}{reversed}\PYG{p}{[}\PYG{l+m+mi}{2}\PYG{p}{]}
\PYG{g+go}{\PYGZsq{}TWO\PYGZsq{}}
\end{Verbatim}
\begin{quote}\begin{description}
\item[{Parameters}] \leavevmode\begin{itemize}
\item {} 
\textbf{\texttt{sequential}} (\href{https://docs.python.org/2.7/library/functions.html\#list}{\emph{list}}) -- support automatic enumeration

\item {} 
\textbf{\texttt{named}} (\href{https://docs.python.org/2.7/library/stdtypes.html\#dict}{\emph{dict}}) -- support definition with unique keys

\end{itemize}

\item[{Returns}] \leavevmode
Enum object

\item[{Return type}] \leavevmode
\href{https://docs.python.org/2.7/library/stdtypes.html\#dict}{dict}

\end{description}\end{quote}

\end{fulllineitems}

\index{quit\_with\_error() (in module escape.util.misc)}

\begin{fulllineitems}
\phantomsection\label{util/misc:escape.util.misc.quit_with_error}\pysiglinewithargsret{\code{escape.util.misc.}\bfcode{quit\_with\_error}}{\emph{msg=None}, \emph{logger='core'}}{}
Helper function for quitting in case of an error
\begin{quote}\begin{description}
\item[{Parameters}] \leavevmode\begin{itemize}
\item {} 
\textbf{\texttt{msg}} (\href{https://docs.python.org/2.7/library/functions.html\#str}{\emph{str}}) -- error message (optional)

\item {} 
\textbf{\texttt{logger}} (\href{https://docs.python.org/2.7/library/functions.html\#str}{\emph{str}}) -- logger name (default: core)

\end{itemize}

\item[{Returns}] \leavevmode
None

\end{description}\end{quote}

\end{fulllineitems}

\index{SimpleStandaloneHelper (class in escape.util.misc)}

\begin{fulllineitems}
\phantomsection\label{util/misc:escape.util.misc.SimpleStandaloneHelper}\pysiglinewithargsret{\strong{class }\code{escape.util.misc.}\bfcode{SimpleStandaloneHelper}}{\emph{container}, \emph{cover\_name}}{}
Bases: \href{https://docs.python.org/2.7/library/functions.html\#object}{\code{object}}

Helper class for layer APIs to catch events and handle these in separate
handler functions
\index{\_\_init\_\_() (escape.util.misc.SimpleStandaloneHelper method)}

\begin{fulllineitems}
\phantomsection\label{util/misc:escape.util.misc.SimpleStandaloneHelper.__init__}\pysiglinewithargsret{\bfcode{\_\_init\_\_}}{\emph{container}, \emph{cover\_name}}{}
Init
\begin{quote}\begin{description}
\item[{Parameters}] \leavevmode\begin{itemize}
\item {} 
\textbf{\texttt{container}} -- Container class reference

\item {} 
\textbf{\texttt{cover\_name}} (\href{https://docs.python.org/2.7/library/functions.html\#str}{\emph{str}}) -- Container's name for logging

\end{itemize}

\item[{Type}] \leavevmode
EventMixin

\end{description}\end{quote}

\end{fulllineitems}

\index{\_register\_listeners() (escape.util.misc.SimpleStandaloneHelper method)}

\begin{fulllineitems}
\phantomsection\label{util/misc:escape.util.misc.SimpleStandaloneHelper._register_listeners}\pysiglinewithargsret{\bfcode{\_register\_listeners}}{}{}
Register event listeners

If a listener is explicitly defined in the class use this function
otherwise use the common logger function
\begin{quote}\begin{description}
\item[{Returns}] \leavevmode
None

\end{description}\end{quote}

\end{fulllineitems}

\index{\_log\_event() (escape.util.misc.SimpleStandaloneHelper method)}

\begin{fulllineitems}
\phantomsection\label{util/misc:escape.util.misc.SimpleStandaloneHelper._log_event}\pysiglinewithargsret{\bfcode{\_log\_event}}{\emph{event}}{}
Log given event
\begin{quote}\begin{description}
\item[{Parameters}] \leavevmode
\textbf{\texttt{event}} (\emph{Event}) -- Event object which need to be logged

\item[{Returns}] \leavevmode
None

\end{description}\end{quote}

\end{fulllineitems}


\end{fulllineitems}

\index{Singleton (class in escape.util.misc)}

\begin{fulllineitems}
\phantomsection\label{util/misc:escape.util.misc.Singleton}\pysigline{\strong{class }\code{escape.util.misc.}\bfcode{Singleton}}
Bases: \href{https://docs.python.org/2.7/library/functions.html\#type}{\code{type}}

Metaclass for classes need to be created only once.

Realize Singleton design pattern in a pythonic way.
\index{\_instances (escape.util.misc.Singleton attribute)}

\begin{fulllineitems}
\phantomsection\label{util/misc:escape.util.misc.Singleton._instances}\pysigline{\bfcode{\_instances}\strong{ = \{\textless{}class `escape.util.misc.ESCAPEConfig'\textgreater{}: \textless{}escape.util.misc.ESCAPEConfig object at 0x7f807e775d50\textgreater{}\}}}
\end{fulllineitems}

\index{\_\_call\_\_() (escape.util.misc.Singleton method)}

\begin{fulllineitems}
\phantomsection\label{util/misc:escape.util.misc.Singleton.__call__}\pysiglinewithargsret{\bfcode{\_\_call\_\_}}{\emph{*args}, \emph{**kwargs}}{}
\end{fulllineitems}


\end{fulllineitems}

\index{ESCAPEConfig (class in escape.util.misc)}

\begin{fulllineitems}
\phantomsection\label{util/misc:escape.util.misc.ESCAPEConfig}\pysiglinewithargsret{\strong{class }\code{escape.util.misc.}\bfcode{ESCAPEConfig}}{\emph{default=None}}{}
Bases: \href{https://docs.python.org/2.7/library/functions.html\#object}{\code{object}}

Wrapper class for configuration to hide specialies with respect to storing,
loading, parging and getting special data.

Contains functions for config handling and manipulation.

Should be instantiated once!
\index{\_\_metaclass\_\_ (escape.util.misc.ESCAPEConfig attribute)}

\begin{fulllineitems}
\phantomsection\label{util/misc:escape.util.misc.ESCAPEConfig.__metaclass__}\pysigline{\bfcode{\_\_metaclass\_\_}}
alias of {\hyperref[util/misc:escape.util.misc.Singleton]{\emph{\code{Singleton}}}}

\end{fulllineitems}

\index{LAYERS (escape.util.misc.ESCAPEConfig attribute)}

\begin{fulllineitems}
\phantomsection\label{util/misc:escape.util.misc.ESCAPEConfig.LAYERS}\pysigline{\bfcode{LAYERS}\strong{ = (`service', `orchestration', `adaptation', `infrastructure')}}
\end{fulllineitems}

\index{\_\_init\_\_() (escape.util.misc.ESCAPEConfig method)}

\begin{fulllineitems}
\phantomsection\label{util/misc:escape.util.misc.ESCAPEConfig.__init__}\pysiglinewithargsret{\bfcode{\_\_init\_\_}}{\emph{default=None}}{}
Init configuration from given data or an empty dict
\begin{quote}\begin{description}
\item[{Parameters}] \leavevmode
\textbf{\texttt{default}} (\href{https://docs.python.org/2.7/library/stdtypes.html\#dict}{\emph{dict}}) -- default configuration

\end{description}\end{quote}

\end{fulllineitems}

\index{add\_cfg() (escape.util.misc.ESCAPEConfig method)}

\begin{fulllineitems}
\phantomsection\label{util/misc:escape.util.misc.ESCAPEConfig.add_cfg}\pysiglinewithargsret{\bfcode{add\_cfg}}{\emph{cfg}}{}
Override configuration
\begin{quote}\begin{description}
\item[{Parameters}] \leavevmode
\textbf{\texttt{cfg}} (\href{https://docs.python.org/2.7/library/stdtypes.html\#dict}{\emph{dict}}) -- new configuration

\item[{Returns}] \leavevmode
None

\end{description}\end{quote}

\end{fulllineitems}

\index{load\_config() (escape.util.misc.ESCAPEConfig method)}

\begin{fulllineitems}
\phantomsection\label{util/misc:escape.util.misc.ESCAPEConfig.load_config}\pysiglinewithargsret{\bfcode{load\_config}}{\emph{config='escape.config'}}{}
Load static configuration from file if it exist or leave the default intact.

\begin{notice}{note}{Note:}
The CONFIG is updated per data under the Layer entries. This means that
the
minimal amount of data have to given is the hole sequence or collection
unter the appropriate key. E.g. the hole data under the `STRATEGY' key in
`orchestration' layer.
\end{notice}
\begin{quote}\begin{description}
\item[{Parameters}] \leavevmode
\textbf{\texttt{config}} (\href{https://docs.python.org/2.7/library/functions.html\#str}{\emph{str}}) -- config file name relative to pox.py (optional)

\item[{Returns}] \leavevmode
self

\item[{Return type}] \leavevmode
{\hyperref[util/misc:escape.util.misc.ESCAPEConfig]{\emph{\code{ESCAPEConfig}}}}

\end{description}\end{quote}

\end{fulllineitems}

\index{dump() (escape.util.misc.ESCAPEConfig method)}

\begin{fulllineitems}
\phantomsection\label{util/misc:escape.util.misc.ESCAPEConfig.dump}\pysiglinewithargsret{\bfcode{dump}}{}{}
Return with the entire configuration in JSON.
\begin{quote}\begin{description}
\item[{Returns}] \leavevmode
config

\item[{Return type}] \leavevmode
\href{https://docs.python.org/2.7/library/functions.html\#str}{str}

\end{description}\end{quote}

\end{fulllineitems}

\index{is\_loaded() (escape.util.misc.ESCAPEConfig method)}

\begin{fulllineitems}
\phantomsection\label{util/misc:escape.util.misc.ESCAPEConfig.is_loaded}\pysiglinewithargsret{\bfcode{is\_loaded}}{\emph{layer}}{}
Return the value given UNIFY's layer is loaded or not.
\begin{quote}\begin{description}
\item[{Parameters}] \leavevmode
\textbf{\texttt{layer}} (\href{https://docs.python.org/2.7/library/functions.html\#str}{\emph{str}}) -- layer name

\item[{Returns}] \leavevmode
layer condition

\item[{Return type}] \leavevmode
\href{https://docs.python.org/2.7/library/functions.html\#bool}{bool}

\end{description}\end{quote}

\end{fulllineitems}

\index{set\_loaded() (escape.util.misc.ESCAPEConfig method)}

\begin{fulllineitems}
\phantomsection\label{util/misc:escape.util.misc.ESCAPEConfig.set_loaded}\pysiglinewithargsret{\bfcode{set\_loaded}}{\emph{layer}}{}
Set the given layer LOADED value.
\begin{quote}\begin{description}
\item[{Parameters}] \leavevmode
\textbf{\texttt{layer}} (\href{https://docs.python.org/2.7/library/functions.html\#str}{\emph{str}}) -- layer name

\item[{Returns}] \leavevmode
None

\end{description}\end{quote}

\end{fulllineitems}

\index{\_\_getitem\_\_() (escape.util.misc.ESCAPEConfig method)}

\begin{fulllineitems}
\phantomsection\label{util/misc:escape.util.misc.ESCAPEConfig.__getitem__}\pysiglinewithargsret{\bfcode{\_\_getitem\_\_}}{\emph{item}}{}
Can be used the config as a dictionary: CONFIG{[}...{]}
\begin{quote}\begin{description}
\item[{Parameters}] \leavevmode
\textbf{\texttt{item}} (\href{https://docs.python.org/2.7/library/functions.html\#str}{\emph{str}}) -- layer key

\item[{Returns}] \leavevmode
layer config

\item[{Return type}] \leavevmode
\href{https://docs.python.org/2.7/library/stdtypes.html\#dict}{dict}

\end{description}\end{quote}

\end{fulllineitems}

\index{\_\_setitem\_\_() (escape.util.misc.ESCAPEConfig method)}

\begin{fulllineitems}
\phantomsection\label{util/misc:escape.util.misc.ESCAPEConfig.__setitem__}\pysiglinewithargsret{\bfcode{\_\_setitem\_\_}}{\emph{key}, \emph{value}}{}
Disable explicit layer config modification.

\end{fulllineitems}

\index{\_\_delitem\_\_() (escape.util.misc.ESCAPEConfig method)}

\begin{fulllineitems}
\phantomsection\label{util/misc:escape.util.misc.ESCAPEConfig.__delitem__}\pysiglinewithargsret{\bfcode{\_\_delitem\_\_}}{\emph{key}}{}
Disable explicit layer config deletion.

\end{fulllineitems}

\index{get\_strategy() (escape.util.misc.ESCAPEConfig method)}

\begin{fulllineitems}
\phantomsection\label{util/misc:escape.util.misc.ESCAPEConfig.get_strategy}\pysiglinewithargsret{\bfcode{get\_strategy}}{\emph{layer}}{}
Return with the Strategy class of the given layer.
\begin{quote}\begin{description}
\item[{Parameters}] \leavevmode
\textbf{\texttt{layer}} (\href{https://docs.python.org/2.7/library/functions.html\#str}{\emph{str}}) -- layer name

\item[{Returns}] \leavevmode
Strategy class

\item[{Return type}] \leavevmode
{\hyperref[util/mapping:escape.util.mapping.AbstractMappingStrategy]{\emph{\code{AbstractMappingStrategy}}}}

\end{description}\end{quote}

\end{fulllineitems}

\index{get\_threaded() (escape.util.misc.ESCAPEConfig method)}

\begin{fulllineitems}
\phantomsection\label{util/misc:escape.util.misc.ESCAPEConfig.get_threaded}\pysiglinewithargsret{\bfcode{get\_threaded}}{\emph{layer}}{}
Return with the value if the mapping strategy is needed to run in
separated thread or not. If value is not defined: return False.
\begin{quote}\begin{description}
\item[{Parameters}] \leavevmode
\textbf{\texttt{layer}} (\href{https://docs.python.org/2.7/library/functions.html\#str}{\emph{str}}) -- layer name

\item[{Returns}] \leavevmode
threading value

\item[{Return type}] \leavevmode
\href{https://docs.python.org/2.7/library/functions.html\#bool}{bool}

\end{description}\end{quote}

\end{fulllineitems}

\index{get\_domain\_component() (escape.util.misc.ESCAPEConfig method)}

\begin{fulllineitems}
\phantomsection\label{util/misc:escape.util.misc.ESCAPEConfig.get_domain_component}\pysiglinewithargsret{\bfcode{get\_domain\_component}}{\emph{component}}{}
Return with the class of the adaptation component.
\begin{quote}\begin{description}
\item[{Parameters}] \leavevmode
\textbf{\texttt{component}} (\href{https://docs.python.org/2.7/library/functions.html\#str}{\emph{str}}) -- component name

\item[{Returns}] \leavevmode
component class

\end{description}\end{quote}

\end{fulllineitems}

\index{get\_default\_mgrs() (escape.util.misc.ESCAPEConfig method)}

\begin{fulllineitems}
\phantomsection\label{util/misc:escape.util.misc.ESCAPEConfig.get_default_mgrs}\pysiglinewithargsret{\bfcode{get\_default\_mgrs}}{}{}
Return the default DomainManagers for initialization on start.
\begin{quote}\begin{description}
\item[{Returns}] \leavevmode
list of {\hyperref[util/adapter:escape.util.adapter.AbstractDomainManager]{\emph{\code{AbstractDomainManager}}}}

\item[{Return type}] \leavevmode
\href{https://docs.python.org/2.7/library/functions.html\#list}{list}

\end{description}\end{quote}

\end{fulllineitems}

\index{get\_fallback\_topology() (escape.util.misc.ESCAPEConfig method)}

\begin{fulllineitems}
\phantomsection\label{util/misc:escape.util.misc.ESCAPEConfig.get_fallback_topology}\pysiglinewithargsret{\bfcode{get\_fallback\_topology}}{\emph{topo\_name}}{}
Return the fallback topology class.
\begin{quote}\begin{description}
\item[{Parameters}] \leavevmode
\textbf{\texttt{topo\_name}} (\href{https://docs.python.org/2.7/library/functions.html\#str}{\emph{str}}) -- name of the topo in CONFIG

\item[{Returns}] \leavevmode
fallback topo class

\item[{Return type}] \leavevmode
:any::\emph{AbstractTopology}

\end{description}\end{quote}

\end{fulllineitems}

\index{get\_clean\_after\_shutdown() (escape.util.misc.ESCAPEConfig method)}

\begin{fulllineitems}
\phantomsection\label{util/misc:escape.util.misc.ESCAPEConfig.get_clean_after_shutdown}\pysiglinewithargsret{\bfcode{get\_clean\_after\_shutdown}}{}{}
Return with the value if a cleaning process need to be done or not.
\begin{quote}\begin{description}
\item[{Returns}] \leavevmode
cleanup (default: False)

\item[{Return type}] \leavevmode
\href{https://docs.python.org/2.7/library/functions.html\#bool}{bool}

\end{description}\end{quote}

\end{fulllineitems}


\end{fulllineitems}



\subparagraph{\emph{netconf.py} module}
\label{util/netconf:netconf-py-module}\label{util/netconf::doc}
Requirements:

\begin{Verbatim}[commandchars=\\\{\}]
sudo apt\PYGZhy{}get install python\PYGZhy{}setuptools python\PYGZhy{}paramiko python\PYGZhy{}lxml \PYGZbs{}
python\PYGZhy{}libxml2 python\PYGZhy{}libxslt1 libxml2 libxslt1\PYGZhy{}dev

sudo pip install ncclient
\end{Verbatim}

{\hyperref[util/netconf:escape.util.netconf.AbstractNETCONFAdapter]{\emph{\code{AbstractNETCONFAdapter}}}} contains the main function for communication
over NETCONF such as managing SSH channel, handling configuration, assemble
RPC request and parse RPC reply


\subparagraph{Module contents}
\label{util/netconf:module-contents}\label{util/netconf:module-escape.util.netconf}\index{escape.util.netconf (module)}
Implement the supporting classes for communication over NETCONF
\index{AbstractNETCONFAdapter (class in escape.util.netconf)}

\begin{fulllineitems}
\phantomsection\label{util/netconf:escape.util.netconf.AbstractNETCONFAdapter}\pysiglinewithargsret{\strong{class }\code{escape.util.netconf.}\bfcode{AbstractNETCONFAdapter}}{\emph{server}, \emph{port}, \emph{username}, \emph{password}, \emph{timeout=30}, \emph{debug=False}}{}
Bases: \href{https://docs.python.org/2.7/library/functions.html\#object}{\code{object}}

Abstract class for various Adapters rely on NETCONF protocol (\index{RFC!RFC 4741}\href{https://tools.ietf.org/html/rfc4741.html}{\textbf{RFC 4741}})

Contains basic functions for managing connection and invoking RPC calls.
Configuration management can be handled by the external
\href{http://ncclient.readthedocs.org/en/latest/manager.html\#ncclient.manager.Manager}{\code{ncclient.manager.Manager}} class exposed by the manager property

Follows the Adapter design pattern
\index{NETCONF\_NAMESPACE (escape.util.netconf.AbstractNETCONFAdapter attribute)}

\begin{fulllineitems}
\phantomsection\label{util/netconf:escape.util.netconf.AbstractNETCONFAdapter.NETCONF_NAMESPACE}\pysigline{\bfcode{NETCONF\_NAMESPACE}\strong{ = `urn:ietf:params:xml:ns:netconf:base:1.0'}}
\end{fulllineitems}

\index{RPC\_NAMESPACE (escape.util.netconf.AbstractNETCONFAdapter attribute)}

\begin{fulllineitems}
\phantomsection\label{util/netconf:escape.util.netconf.AbstractNETCONFAdapter.RPC_NAMESPACE}\pysigline{\bfcode{RPC\_NAMESPACE}\strong{ = None}}
\end{fulllineitems}

\index{\_\_init\_\_() (escape.util.netconf.AbstractNETCONFAdapter method)}

\begin{fulllineitems}
\phantomsection\label{util/netconf:escape.util.netconf.AbstractNETCONFAdapter.__init__}\pysiglinewithargsret{\bfcode{\_\_init\_\_}}{\emph{server}, \emph{port}, \emph{username}, \emph{password}, \emph{timeout=30}, \emph{debug=False}}{}
Initialize connection parameters
\begin{quote}\begin{description}
\item[{Parameters}] \leavevmode\begin{itemize}
\item {} 
\textbf{\texttt{server}} (\href{https://docs.python.org/2.7/library/functions.html\#str}{\emph{str}}) -- server address

\item {} 
\textbf{\texttt{port}} (\href{https://docs.python.org/2.7/library/functions.html\#int}{\emph{int}}) -- port number

\item {} 
\textbf{\texttt{username}} (\href{https://docs.python.org/2.7/library/functions.html\#str}{\emph{str}}) -- username

\item {} 
\textbf{\texttt{password}} (\href{https://docs.python.org/2.7/library/functions.html\#str}{\emph{str}}) -- password

\item {} 
\textbf{\texttt{timeout}} (\href{https://docs.python.org/2.7/library/functions.html\#int}{\emph{int}}) -- connection timeout (default=30)

\item {} 
\textbf{\texttt{debug}} (\href{https://docs.python.org/2.7/library/functions.html\#bool}{\emph{bool}}) -- print DEBUG infos, RPC messages ect. (default: False)

\end{itemize}

\item[{Returns}] \leavevmode
None

\end{description}\end{quote}

\end{fulllineitems}

\index{connected (escape.util.netconf.AbstractNETCONFAdapter attribute)}

\begin{fulllineitems}
\phantomsection\label{util/netconf:escape.util.netconf.AbstractNETCONFAdapter.connected}\pysigline{\bfcode{connected}}~\begin{quote}\begin{description}
\item[{Returns}] \leavevmode
Return connection state

\item[{Return type}] \leavevmode
\href{https://docs.python.org/2.7/library/functions.html\#bool}{bool}

\end{description}\end{quote}

\end{fulllineitems}

\index{connection\_data (escape.util.netconf.AbstractNETCONFAdapter attribute)}

\begin{fulllineitems}
\phantomsection\label{util/netconf:escape.util.netconf.AbstractNETCONFAdapter.connection_data}\pysigline{\bfcode{connection\_data}}~\begin{quote}\begin{description}
\item[{Returns}] \leavevmode
Return connection data in (server, port, username) tuples

\item[{Return type}] \leavevmode
\href{https://docs.python.org/2.7/library/functions.html\#tuple}{tuple}

\end{description}\end{quote}

\end{fulllineitems}

\index{manager (escape.util.netconf.AbstractNETCONFAdapter attribute)}

\begin{fulllineitems}
\phantomsection\label{util/netconf:escape.util.netconf.AbstractNETCONFAdapter.manager}\pysigline{\bfcode{manager}}~\begin{quote}\begin{description}
\item[{Returns}] \leavevmode
Return the connection manager (wrapper for NETCONF commands)

\item[{Return type}] \leavevmode
\href{http://ncclient.readthedocs.org/en/latest/manager.html\#ncclient.manager.Manager}{\code{ncclient.manager.Manager}}

\end{description}\end{quote}

\end{fulllineitems}

\index{connect() (escape.util.netconf.AbstractNETCONFAdapter method)}

\begin{fulllineitems}
\phantomsection\label{util/netconf:escape.util.netconf.AbstractNETCONFAdapter.connect}\pysiglinewithargsret{\bfcode{connect}}{}{}
This function will connect to the netconf server.
\begin{quote}\begin{description}
\item[{Returns}] \leavevmode
Also returns the NETCONF connection manager

\item[{Return type}] \leavevmode
\href{http://ncclient.readthedocs.org/en/latest/manager.html\#ncclient.manager.Manager}{\code{ncclient.manager.Manager}}

\end{description}\end{quote}

\end{fulllineitems}

\index{disconnect() (escape.util.netconf.AbstractNETCONFAdapter method)}

\begin{fulllineitems}
\phantomsection\label{util/netconf:escape.util.netconf.AbstractNETCONFAdapter.disconnect}\pysiglinewithargsret{\bfcode{disconnect}}{}{}
This function will close the connection.
\begin{quote}\begin{description}
\item[{Returns}] \leavevmode
None

\end{description}\end{quote}

\end{fulllineitems}

\index{get\_config() (escape.util.netconf.AbstractNETCONFAdapter method)}

\begin{fulllineitems}
\phantomsection\label{util/netconf:escape.util.netconf.AbstractNETCONFAdapter.get_config}\pysiglinewithargsret{\bfcode{get\_config}}{\emph{source='running'}, \emph{to\_file=False}}{}
This function will download the configuration of the NETCONF agent in an
XML format. If source is None then the running config will be downloaded.
Other configurations are netconf specific (\index{RFC!RFC 6241}\href{https://tools.ietf.org/html/rfc6241.html}{\textbf{RFC 6241}}) - running,
candidate, startup
\begin{quote}\begin{description}
\item[{Parameters}] \leavevmode\begin{itemize}
\item {} 
\textbf{\texttt{source}} (\href{https://docs.python.org/2.7/library/functions.html\#str}{\emph{str}}) -- NETCONF specific configuration source (defalut: running)

\item {} 
\textbf{\texttt{to\_file}} (\href{https://docs.python.org/2.7/library/functions.html\#bool}{\emph{bool}}) -- save config to file

\end{itemize}

\item[{Returns}] \leavevmode
None

\end{description}\end{quote}

\end{fulllineitems}

\index{get() (escape.util.netconf.AbstractNETCONFAdapter method)}

\begin{fulllineitems}
\phantomsection\label{util/netconf:escape.util.netconf.AbstractNETCONFAdapter.get}\pysiglinewithargsret{\bfcode{get}}{\emph{expr='/'}}{}
This process works as yangcli's GET function. A lot of information can be
got from the running NETCONF agent. If an xpath-based expression is also
set, the results can be filtered. The results are not printed out in a
file, it's only printed to stdout
\begin{quote}\begin{description}
\item[{Parameters}] \leavevmode
\textbf{\texttt{expr}} (\href{https://docs.python.org/2.7/library/functions.html\#str}{\emph{str}}) -- xpath-based expression

\item[{Returns}] \leavevmode
result in XML

\item[{Return type}] \leavevmode
\href{https://docs.python.org/2.7/library/functions.html\#str}{str}

\end{description}\end{quote}

\end{fulllineitems}

\index{\_create\_rpc\_request() (escape.util.netconf.AbstractNETCONFAdapter method)}

\begin{fulllineitems}
\phantomsection\label{util/netconf:escape.util.netconf.AbstractNETCONFAdapter._create_rpc_request}\pysiglinewithargsret{\bfcode{\_create\_rpc\_request}}{\emph{rpc\_name}, \emph{**params}}{}
This function is devoted to create a raw RPC request message in XML format.
Any further additional rpc-input can be passed towards, if netconf agent
has this input list, called `options'. Switches is used for connectVNF
rpc in order to set the switches where the vnf should be connected.
\begin{quote}\begin{description}
\item[{Parameters}] \leavevmode\begin{itemize}
\item {} 
\textbf{\texttt{rpc\_name}} (\href{https://docs.python.org/2.7/library/functions.html\#str}{\emph{str}}) -- rpc name

\item {} 
\textbf{\texttt{options}} (\href{https://docs.python.org/2.7/library/stdtypes.html\#dict}{\emph{dict}}) -- additional RPC input in the specific \textless{}options\textgreater{} tag

\item {} 
\textbf{\texttt{switches}} (\href{https://docs.python.org/2.7/library/functions.html\#list}{\emph{list}}) -- set the switches where the vnf should be connected

\item {} 
\textbf{\texttt{params}} (\href{https://docs.python.org/2.7/library/stdtypes.html\#dict}{\emph{dict}}) -- input params for the RPC using param's name as XML tag name

\end{itemize}

\item[{Returns}] \leavevmode
raw RPC message in XML format (lxml library)

\item[{Return type}] \leavevmode
\code{lxml.etree.ElementTree}

\end{description}\end{quote}

\end{fulllineitems}

\index{\_parse\_rpc\_response() (escape.util.netconf.AbstractNETCONFAdapter method)}

\begin{fulllineitems}
\phantomsection\label{util/netconf:escape.util.netconf.AbstractNETCONFAdapter._parse_rpc_response}\pysiglinewithargsret{\bfcode{\_parse\_rpc\_response}}{\emph{data=None}}{}
Parse raw XML response and return params in dictionary. If data is given
it is parsed instead of the lasr response and the result will not be saved.
\begin{quote}\begin{description}
\item[{Parameters}] \leavevmode
\textbf{\texttt{data}} (\code{lxml.etree.ElementTree}) -- raw data (uses last reply by default)

\item[{Returns}] \leavevmode
return parsed params

\item[{Return type}] \leavevmode
\href{https://docs.python.org/2.7/library/stdtypes.html\#dict}{dict}

\end{description}\end{quote}

\end{fulllineitems}

\index{\_invoke\_rpc() (escape.util.netconf.AbstractNETCONFAdapter method)}

\begin{fulllineitems}
\phantomsection\label{util/netconf:escape.util.netconf.AbstractNETCONFAdapter._invoke_rpc}\pysiglinewithargsret{\bfcode{\_invoke\_rpc}}{\emph{request\_data}}{}
This function is devoted to call an RPC, and parses the rpc-reply message
(if needed) and returns every important parts of it in as a dictionary.
Any further additional rpc-input can be passed towards, if netconf agent
has this input list, called `options'. Switches is used for connectVNF
rpc in order to set the switches where the vnf should be connected.
\begin{quote}\begin{description}
\item[{Parameters}] \leavevmode
\textbf{\texttt{request\_data}} (\href{https://docs.python.org/2.7/library/stdtypes.html\#dict}{\emph{dict}}) -- data for RPC request body

\item[{Returns}] \leavevmode
raw RPC response

\item[{Return type}] \leavevmode
\code{lxml.etree.ElementTree}

\end{description}\end{quote}

\end{fulllineitems}

\index{call\_RPC() (escape.util.netconf.AbstractNETCONFAdapter method)}

\begin{fulllineitems}
\phantomsection\label{util/netconf:escape.util.netconf.AbstractNETCONFAdapter.call_RPC}\pysiglinewithargsret{\bfcode{call\_RPC}}{\emph{rpc\_name}, \emph{no\_rpc\_error=False}, \emph{**params}}{}
Call an RPC given by rpc\_name. If \emph{no\_rpc\_error} is set returns with a
dict instead of raising \code{RPCError}
\begin{quote}\begin{description}
\item[{Parameters}] \leavevmode\begin{itemize}
\item {} 
\textbf{\texttt{rpc\_name}} (\href{https://docs.python.org/2.7/library/functions.html\#str}{\emph{str}}) -- RPC name

\item {} 
\textbf{\texttt{no\_rpc\_error}} (\href{https://docs.python.org/2.7/library/functions.html\#bool}{\emph{bool}}) -- return with dict (RPC error) instead of exception

\end{itemize}

\item[{Returns}] \leavevmode
RPC reply

\item[{Return type}] \leavevmode
\href{https://docs.python.org/2.7/library/stdtypes.html\#dict}{dict}

\end{description}\end{quote}

\end{fulllineitems}

\index{\_\_enter\_\_() (escape.util.netconf.AbstractNETCONFAdapter method)}

\begin{fulllineitems}
\phantomsection\label{util/netconf:escape.util.netconf.AbstractNETCONFAdapter.__enter__}\pysiglinewithargsret{\bfcode{\_\_enter\_\_}}{}{}
Context manager setup action.

Usage:

\begin{Verbatim}[commandchars=\\\{\}]
\PYG{k}{with} \PYG{n}{AbstractNETCONFAdapter}\PYG{p}{(}\PYG{p}{)} \PYG{k}{as} \PYG{n}{adapter}\PYG{p}{:}
  \PYG{o}{.}\PYG{o}{.}\PYG{o}{.}
\end{Verbatim}

\end{fulllineitems}

\index{\_\_exit\_\_() (escape.util.netconf.AbstractNETCONFAdapter method)}

\begin{fulllineitems}
\phantomsection\label{util/netconf:escape.util.netconf.AbstractNETCONFAdapter.__exit__}\pysiglinewithargsret{\bfcode{\_\_exit\_\_}}{\emph{exc\_type}, \emph{exc\_val}, \emph{exc\_tb}}{}
Context manager cleanup action

\end{fulllineitems}

\index{\_AbstractNETCONFAdapter\_\_parse\_rpc\_params() (escape.util.netconf.AbstractNETCONFAdapter method)}

\begin{fulllineitems}
\phantomsection\label{util/netconf:escape.util.netconf.AbstractNETCONFAdapter._AbstractNETCONFAdapter__parse_rpc_params}\pysiglinewithargsret{\bfcode{\_AbstractNETCONFAdapter\_\_parse\_rpc\_params}}{\emph{rpc\_request}, \emph{params}}{}
Parse given keyword arguments and generate RPC body in proper XML format.
The key value is used as the XML tag name. If the value is another
dictionary the XML structure follows the hierarchy. The param values can
be only simple types and dictionary for simplicity. Convertation example:

\begin{Verbatim}[commandchars=\\\{\}]
\PYGZob{}
  \PYGZsq{}vnf\PYGZus{}type\PYGZsq{}: \PYGZsq{}headerDecompressor\PYGZsq{},
  \PYGZsq{}options\PYGZsq{}: \PYGZob{}
              \PYGZsq{}name\PYGZsq{}: \PYGZsq{}ip\PYGZsq{},
              \PYGZsq{}value\PYGZsq{}: 127.0.0.1
              \PYGZcb{}
\PYGZcb{}
\end{Verbatim}

will be generated into:

\begin{Verbatim}[commandchars=\\\{\}]
\PYGZlt{}rpc\PYGZhy{}call\PYGZhy{}name\PYGZgt{}
  \PYGZlt{}vnf\PYGZus{}type\PYGZgt{}headerDecompressor\PYGZlt{}/vnf\PYGZus{}type\PYGZgt{}
  \PYGZlt{}options\PYGZgt{}
    \PYGZlt{}name\PYGZgt{}ip\PYGZlt{}/name\PYGZgt{}
    \PYGZlt{}value\PYGZgt{}127.0.0.1\PYGZlt{}/value\PYGZgt{}
  \PYGZlt{}/options\PYGZgt{}
\PYGZlt{}/rpc\PYGZhy{}call\PYGZhy{}name\PYGZgt{}
\end{Verbatim}
\begin{quote}\begin{description}
\item[{Parameters}] \leavevmode\begin{itemize}
\item {} 
\textbf{\texttt{rpc\_request}} (\code{lxml.etree.ElementTree}) -- empty RPC request

\item {} 
\textbf{\texttt{params}} (\href{https://docs.python.org/2.7/library/stdtypes.html\#dict}{\emph{dict}}) -- RPC call argument given in a dictionary

\end{itemize}

\item[{Returns}] \leavevmode
parsed params in XML format (lxml library)

\item[{Return type}] \leavevmode
\code{lxml.etree.ElementTree}

\end{description}\end{quote}

\end{fulllineitems}

\index{\_AbstractNETCONFAdapter\_\_parse\_xml\_response() (escape.util.netconf.AbstractNETCONFAdapter method)}

\begin{fulllineitems}
\phantomsection\label{util/netconf:escape.util.netconf.AbstractNETCONFAdapter._AbstractNETCONFAdapter__parse_xml_response}\pysiglinewithargsret{\bfcode{\_AbstractNETCONFAdapter\_\_parse\_xml\_response}}{\emph{element}, \emph{namespace=None}}{}
This is an inner function, which is devoted to automatically analyze the
rpc-reply message and iterate through all the xml elements until the last
child is found, and then create a dictionary. Return a dict with the
parsed data. If the reply is OK the returned dict contains an \emph{rcp-reply}
element with value \emph{OK}.
\begin{quote}\begin{description}
\item[{Parameters}] \leavevmode\begin{itemize}
\item {} 
\textbf{\texttt{element}} (\code{lxml.etree.ElementTree}) -- XML element

\item {} 
\textbf{\texttt{namespace}} -- namespace

\end{itemize}

\item[{Type}] \leavevmode
str

\item[{Returns}] \leavevmode
parsed XML data

\item[{Return type}] \leavevmode
\href{https://docs.python.org/2.7/library/stdtypes.html\#dict}{dict}

\end{description}\end{quote}

\end{fulllineitems}

\index{\_AbstractNETCONFAdapter\_\_remove\_namespace() (escape.util.netconf.AbstractNETCONFAdapter method)}

\begin{fulllineitems}
\phantomsection\label{util/netconf:escape.util.netconf.AbstractNETCONFAdapter._AbstractNETCONFAdapter__remove_namespace}\pysiglinewithargsret{\bfcode{\_AbstractNETCONFAdapter\_\_remove\_namespace}}{\emph{xml\_element}, \emph{namespace=None}}{}
Own function to remove the ncclient's namespace prefix, because it causes
``definition not found error'' if OWN modules and RPCs are being used
\begin{quote}\begin{description}
\item[{Parameters}] \leavevmode\begin{itemize}
\item {} 
\textbf{\texttt{xml\_element}} (\code{lxml.etree.ElementTree}) -- XML element

\item {} 
\textbf{\texttt{namespace}} (\code{lxml.etree.ElementTree}) -- namespace

\end{itemize}

\item[{Returns}] \leavevmode
cleaned XML elemenet

\item[{Return type}] \leavevmode
\code{lxml.etree.ElementTree}

\end{description}\end{quote}

\end{fulllineitems}


\end{fulllineitems}



\subparagraph{\emph{nffg.py} module}
\label{util/nffg::doc}\label{util/nffg:nffg-py-module}
{\hyperref[util/nffg:escape.util.nffg.NFFG]{\emph{\code{NFFG}}}} represents the internal format of NF-FG, SG and RG


\subparagraph{Module contents}
\label{util/nffg:module-contents}\label{util/nffg:module-escape.util.nffg}\index{escape.util.nffg (module)}
Abstract class and implementation for basic operations with a single
NF-FG, such as building, parsing, processing NF-FG, helper functions,
etc.
\index{AbstractNFFG (class in escape.util.nffg)}

\begin{fulllineitems}
\phantomsection\label{util/nffg:escape.util.nffg.AbstractNFFG}\pysiglinewithargsret{\strong{class }\code{escape.util.nffg.}\bfcode{AbstractNFFG}}{\emph{id}, \emph{name=None}, \emph{version=`1.0'}, \emph{json=None}, \emph{file=None}}{}
Bases: \href{https://docs.python.org/2.7/library/functions.html\#object}{\code{object}}

Abstract class for managing single NF-FG data structure

The NF-FG data model is described in YANG.  This class provides the
interfaces with the high level data manipulation functions.
\index{\_\_metaclass\_\_ (escape.util.nffg.AbstractNFFG attribute)}

\begin{fulllineitems}
\phantomsection\label{util/nffg:escape.util.nffg.AbstractNFFG.__metaclass__}\pysigline{\bfcode{\_\_metaclass\_\_}}
alias of \code{ABCMeta}

\end{fulllineitems}

\index{\_\_init\_\_() (escape.util.nffg.AbstractNFFG method)}

\begin{fulllineitems}
\phantomsection\label{util/nffg:escape.util.nffg.AbstractNFFG.__init__}\pysiglinewithargsret{\bfcode{\_\_init\_\_}}{\emph{id}, \emph{name=None}, \emph{version=`1.0'}, \emph{json=None}, \emph{file=None}}{}
Init

\end{fulllineitems}

\index{add\_nf() (escape.util.nffg.AbstractNFFG method)}

\begin{fulllineitems}
\phantomsection\label{util/nffg:escape.util.nffg.AbstractNFFG.add_nf}\pysiglinewithargsret{\bfcode{add\_nf}}{\emph{node\_nf}}{}
Add a single NF node to the NF-FG

\end{fulllineitems}

\index{add\_sap() (escape.util.nffg.AbstractNFFG method)}

\begin{fulllineitems}
\phantomsection\label{util/nffg:escape.util.nffg.AbstractNFFG.add_sap}\pysiglinewithargsret{\bfcode{add\_sap}}{\emph{node\_sap}}{}
Add a single SAP node to the NF-FG

\end{fulllineitems}

\index{add\_infra() (escape.util.nffg.AbstractNFFG method)}

\begin{fulllineitems}
\phantomsection\label{util/nffg:escape.util.nffg.AbstractNFFG.add_infra}\pysiglinewithargsret{\bfcode{add\_infra}}{\emph{node\_infra}}{}
Add a single infrastructure node to the NF-FG

\end{fulllineitems}

\index{add\_edge() (escape.util.nffg.AbstractNFFG method)}

\begin{fulllineitems}
\phantomsection\label{util/nffg:escape.util.nffg.AbstractNFFG.add_edge}\pysiglinewithargsret{\bfcode{add\_edge}}{\emph{src}, \emph{dst}, \emph{params=None}}{}
Add an edge to the NF-FG
\begin{quote}\begin{description}
\item[{Parameters}] \leavevmode\begin{itemize}
\item {} 
\textbf{\texttt{src}} (\emph{(Node, Port) inherited Node classes: NodeNF, NodeSAP, NodeInfra}) -- source (node, port) of the edge

\item {} 
\textbf{\texttt{dst}} (\emph{(Node, Port) inherited Node classes: NodeNF, NodeSAP, NodeInfra}) -- destination (node, port) of the edge

\item {} 
\textbf{\texttt{params}} (\emph{ResOfEdge or Flowrule}) -- attribute of the edge depending on the type

\end{itemize}

\item[{Returns}] \leavevmode
None

\end{description}\end{quote}

\end{fulllineitems}

\index{add\_link() (escape.util.nffg.AbstractNFFG method)}

\begin{fulllineitems}
\phantomsection\label{util/nffg:escape.util.nffg.AbstractNFFG.add_link}\pysiglinewithargsret{\bfcode{add\_link}}{\emph{edge\_link}}{}
Add a static or dynamic infrastructure link to the NF-FG

\end{fulllineitems}

\index{add\_sglink() (escape.util.nffg.AbstractNFFG method)}

\begin{fulllineitems}
\phantomsection\label{util/nffg:escape.util.nffg.AbstractNFFG.add_sglink}\pysiglinewithargsret{\bfcode{add\_sglink}}{\emph{edge\_sglink}}{}
Add an SG link to the NF-FG

\end{fulllineitems}

\index{add\_req() (escape.util.nffg.AbstractNFFG method)}

\begin{fulllineitems}
\phantomsection\label{util/nffg:escape.util.nffg.AbstractNFFG.add_req}\pysiglinewithargsret{\bfcode{add\_req}}{\emph{edge\_req}}{}
Add a requirement link to the NF-FG

\end{fulllineitems}

\index{del\_node() (escape.util.nffg.AbstractNFFG method)}

\begin{fulllineitems}
\phantomsection\label{util/nffg:escape.util.nffg.AbstractNFFG.del_node}\pysiglinewithargsret{\bfcode{del\_node}}{\emph{id}}{}
Delete a single node from the NF-FG

\end{fulllineitems}

\index{\_init\_from\_json() (escape.util.nffg.AbstractNFFG method)}

\begin{fulllineitems}
\phantomsection\label{util/nffg:escape.util.nffg.AbstractNFFG._init_from_json}\pysiglinewithargsret{\bfcode{\_init\_from\_json}}{\emph{json\_data}}{}
Initialize the NFFG object from JSON data
\begin{quote}\begin{description}
\item[{Parameters}] \leavevmode
\textbf{\texttt{json\_data}} (\href{https://docs.python.org/2.7/library/functions.html\#str}{\emph{str}}) -- NF-FG represented in JSON format

\item[{Returns}] \leavevmode
None

\end{description}\end{quote}

\end{fulllineitems}

\index{load\_from\_file() (escape.util.nffg.AbstractNFFG static method)}

\begin{fulllineitems}
\phantomsection\label{util/nffg:escape.util.nffg.AbstractNFFG.load_from_file}\pysiglinewithargsret{\strong{static }\bfcode{load\_from\_file}}{\emph{filename}}{}
\end{fulllineitems}

\index{to\_json() (escape.util.nffg.AbstractNFFG method)}

\begin{fulllineitems}
\phantomsection\label{util/nffg:escape.util.nffg.AbstractNFFG.to_json}\pysiglinewithargsret{\bfcode{to\_json}}{}{}
Return a JSON string represent this instance
\begin{quote}\begin{description}
\item[{Returns}] \leavevmode
JSON formatted string

\item[{Return type}] \leavevmode
\href{https://docs.python.org/2.7/library/functions.html\#str}{str}

\end{description}\end{quote}

\end{fulllineitems}

\index{\_\_copy\_\_() (escape.util.nffg.AbstractNFFG method)}

\begin{fulllineitems}
\phantomsection\label{util/nffg:escape.util.nffg.AbstractNFFG.__copy__}\pysiglinewithargsret{\bfcode{\_\_copy\_\_}}{}{}
Magic class for creating a shallow copy of actual class using the
copy.copy() function of Python standard library. This means that,
while the instance itself is a new instance, all of its data is referenced.
\begin{quote}\begin{description}
\item[{Returns}] \leavevmode
shallow copy of this instace

\item[{Return type}] \leavevmode
{\hyperref[util/nffg:escape.util.nffg.NFFG]{\emph{\code{NFFG}}}}

\end{description}\end{quote}

\end{fulllineitems}

\index{\_\_deepcopy\_\_() (escape.util.nffg.AbstractNFFG method)}

\begin{fulllineitems}
\phantomsection\label{util/nffg:escape.util.nffg.AbstractNFFG.__deepcopy__}\pysiglinewithargsret{\bfcode{\_\_deepcopy\_\_}}{\emph{memo=\{\}}}{}
Magic class for creating a deep copy of actual class using the
copy.deepcopy() function of Python standard library. The object and its
data are both copied.
\begin{quote}\begin{description}
\item[{Parameters}] \leavevmode
\textbf{\texttt{memo}} (\href{https://docs.python.org/2.7/library/stdtypes.html\#dict}{\emph{dict}}) -- is a cache of previously copied objects

\item[{Returns}] \leavevmode
shallow copy of this instace

\item[{Return type}] \leavevmode
{\hyperref[util/nffg:escape.util.nffg.NFFG]{\emph{\code{NFFG}}}}

\end{description}\end{quote}

\end{fulllineitems}

\index{\_\_abstractmethods\_\_ (escape.util.nffg.AbstractNFFG attribute)}

\begin{fulllineitems}
\phantomsection\label{util/nffg:escape.util.nffg.AbstractNFFG.__abstractmethods__}\pysigline{\bfcode{\_\_abstractmethods\_\_}\strong{ = frozenset({[}'add\_sglink', `add\_req', `add\_link', `del\_node', `add\_infra', `add\_sap', `add\_nf'{]})}}
\end{fulllineitems}

\index{\_abc\_cache (escape.util.nffg.AbstractNFFG attribute)}

\begin{fulllineitems}
\phantomsection\label{util/nffg:escape.util.nffg.AbstractNFFG._abc_cache}\pysigline{\bfcode{\_abc\_cache}\strong{ = \textless{}\_weakrefset.WeakSet object\textgreater{}}}
\end{fulllineitems}

\index{\_abc\_negative\_cache (escape.util.nffg.AbstractNFFG attribute)}

\begin{fulllineitems}
\phantomsection\label{util/nffg:escape.util.nffg.AbstractNFFG._abc_negative_cache}\pysigline{\bfcode{\_abc\_negative\_cache}\strong{ = \textless{}\_weakrefset.WeakSet object\textgreater{}}}
\end{fulllineitems}

\index{\_abc\_negative\_cache\_version (escape.util.nffg.AbstractNFFG attribute)}

\begin{fulllineitems}
\phantomsection\label{util/nffg:escape.util.nffg.AbstractNFFG._abc_negative_cache_version}\pysigline{\bfcode{\_abc\_negative\_cache\_version}\strong{ = 25}}
\end{fulllineitems}

\index{\_abc\_registry (escape.util.nffg.AbstractNFFG attribute)}

\begin{fulllineitems}
\phantomsection\label{util/nffg:escape.util.nffg.AbstractNFFG._abc_registry}\pysigline{\bfcode{\_abc\_registry}\strong{ = \textless{}\_weakrefset.WeakSet object\textgreater{}}}
\end{fulllineitems}


\end{fulllineitems}

\index{NFFG (class in escape.util.nffg)}

\begin{fulllineitems}
\phantomsection\label{util/nffg:escape.util.nffg.NFFG}\pysiglinewithargsret{\strong{class }\code{escape.util.nffg.}\bfcode{NFFG}}{\emph{id=None}, \emph{name=None}, \emph{version=`1.0'}, \emph{json=None}, \emph{file=None}}{}
Bases: {\hyperref[util/nffg:escape.util.nffg.AbstractNFFG]{\emph{\code{escape.util.nffg.AbstractNFFG}}}}, \code{networkx.classes.multigraph.MultiGraph}

NF-FG implementation based on NetworkX.

Implement the AbstractNFFG using NetworkX graph representation
internally.  Implement the high level functions and additionally
expose NetworkX API.
\index{\_\_init\_\_() (escape.util.nffg.NFFG method)}

\begin{fulllineitems}
\phantomsection\label{util/nffg:escape.util.nffg.NFFG.__init__}\pysiglinewithargsret{\bfcode{\_\_init\_\_}}{\emph{id=None}, \emph{name=None}, \emph{version=`1.0'}, \emph{json=None}, \emph{file=None}}{}
Init

\end{fulllineitems}

\index{add\_nf() (escape.util.nffg.NFFG method)}

\begin{fulllineitems}
\phantomsection\label{util/nffg:escape.util.nffg.NFFG.add_nf}\pysiglinewithargsret{\bfcode{add\_nf}}{\emph{node\_nf}}{}
Add a single NF node to the NF-FG

\end{fulllineitems}

\index{add\_sap() (escape.util.nffg.NFFG method)}

\begin{fulllineitems}
\phantomsection\label{util/nffg:escape.util.nffg.NFFG.add_sap}\pysiglinewithargsret{\bfcode{add\_sap}}{\emph{node\_sap}}{}
Add a single SAP node to the NF-FG

\end{fulllineitems}

\index{add\_infra() (escape.util.nffg.NFFG method)}

\begin{fulllineitems}
\phantomsection\label{util/nffg:escape.util.nffg.NFFG.add_infra}\pysiglinewithargsret{\bfcode{add\_infra}}{\emph{node\_infra}}{}
Add a single infrastructure node to the NF-FG

\end{fulllineitems}

\index{add\_link() (escape.util.nffg.NFFG method)}

\begin{fulllineitems}
\phantomsection\label{util/nffg:escape.util.nffg.NFFG.add_link}\pysiglinewithargsret{\bfcode{add\_link}}{\emph{edge\_link}}{}
Add a static or dynamic infrastructure link to the NF-FG

\end{fulllineitems}

\index{add\_sglink() (escape.util.nffg.NFFG method)}

\begin{fulllineitems}
\phantomsection\label{util/nffg:escape.util.nffg.NFFG.add_sglink}\pysiglinewithargsret{\bfcode{add\_sglink}}{\emph{edge\_sglink}}{}
Add an SG link to the NF-FG

\end{fulllineitems}

\index{add\_req() (escape.util.nffg.NFFG method)}

\begin{fulllineitems}
\phantomsection\label{util/nffg:escape.util.nffg.NFFG.add_req}\pysiglinewithargsret{\bfcode{add\_req}}{\emph{edge\_req}}{}
Add a requirement link to the NF-FG

\end{fulllineitems}

\index{del\_node() (escape.util.nffg.NFFG method)}

\begin{fulllineitems}
\phantomsection\label{util/nffg:escape.util.nffg.NFFG.del_node}\pysiglinewithargsret{\bfcode{del\_node}}{\emph{id}}{}
Delete a single node from the NF-FG

\end{fulllineitems}

\index{\_\_abstractmethods\_\_ (escape.util.nffg.NFFG attribute)}

\begin{fulllineitems}
\phantomsection\label{util/nffg:escape.util.nffg.NFFG.__abstractmethods__}\pysigline{\bfcode{\_\_abstractmethods\_\_}\strong{ = frozenset({[}{]})}}
\end{fulllineitems}

\index{\_abc\_cache (escape.util.nffg.NFFG attribute)}

\begin{fulllineitems}
\phantomsection\label{util/nffg:escape.util.nffg.NFFG._abc_cache}\pysigline{\bfcode{\_abc\_cache}\strong{ = \textless{}\_weakrefset.WeakSet object\textgreater{}}}
\end{fulllineitems}

\index{\_abc\_negative\_cache (escape.util.nffg.NFFG attribute)}

\begin{fulllineitems}
\phantomsection\label{util/nffg:escape.util.nffg.NFFG._abc_negative_cache}\pysigline{\bfcode{\_abc\_negative\_cache}\strong{ = \textless{}\_weakrefset.WeakSet object\textgreater{}}}
\end{fulllineitems}

\index{\_abc\_negative\_cache\_version (escape.util.nffg.NFFG attribute)}

\begin{fulllineitems}
\phantomsection\label{util/nffg:escape.util.nffg.NFFG._abc_negative_cache_version}\pysigline{\bfcode{\_abc\_negative\_cache\_version}\strong{ = 25}}
\end{fulllineitems}

\index{\_abc\_registry (escape.util.nffg.NFFG attribute)}

\begin{fulllineitems}
\phantomsection\label{util/nffg:escape.util.nffg.NFFG._abc_registry}\pysigline{\bfcode{\_abc\_registry}\strong{ = \textless{}\_weakrefset.WeakSet object\textgreater{}}}
\end{fulllineitems}


\end{fulllineitems}

\index{main() (in module escape.util.nffg)}

\begin{fulllineitems}
\phantomsection\label{util/nffg:escape.util.nffg.main}\pysiglinewithargsret{\code{escape.util.nffg.}\bfcode{main}}{\emph{argv=None}}{}
\end{fulllineitems}



\subparagraph{\emph{pox\_extension.py} module}
\label{util/pox_extension:pox-extension-py-module}\label{util/pox_extension::doc}
{\hyperref[util/pox_extension:escape.util.pox_extension.OpenFlowBridge]{\emph{\code{OpenFlowBridge}}}} is a special version of OpenFlow event originator class

{\hyperref[util/pox_extension:escape.util.pox_extension.ExtendedOFConnectionArbiter]{\emph{\code{ExtendedOFConnectionArbiter}}}} dispatches incoming OpenFlow connections to
fit ESCAPEv2


\subparagraph{Module contents}
\label{util/pox_extension:module-contents}\label{util/pox_extension:module-escape.util.pox_extension}\index{escape.util.pox\_extension (module)}
Override and extend internal POX components to achieve ESCAPE-desired behaviour
\index{OpenFlowBridge (class in escape.util.pox\_extension)}

\begin{fulllineitems}
\phantomsection\label{util/pox_extension:escape.util.pox_extension.OpenFlowBridge}\pysigline{\strong{class }\code{escape.util.pox\_extension.}\bfcode{OpenFlowBridge}}
Bases: \code{pox.openflow.OpenFlowNexus}

Own class for listening OpenFlow event originated by one of the contained
\code{Connection} and sending OpenFlow messages according to DPID

Purpose of the class mostly fits the Bride design pattern

\end{fulllineitems}

\index{ExtendedOFConnectionArbiter (class in escape.util.pox\_extension)}

\begin{fulllineitems}
\phantomsection\label{util/pox_extension:escape.util.pox_extension.ExtendedOFConnectionArbiter}\pysiglinewithargsret{\strong{class }\code{escape.util.pox\_extension.}\bfcode{ExtendedOFConnectionArbiter}}{\emph{default=False}}{}
Bases: \code{pox.openflow.OpenFlowConnectionArbiter}

Extended connection arbiter class for dispatching incoming OpenFlow
\code{Connection} between registered OF event originators (
\code{OpenFlowNexus}) according to the connection's listening address
\index{\_core\_name (escape.util.pox\_extension.ExtendedOFConnectionArbiter attribute)}

\begin{fulllineitems}
\phantomsection\label{util/pox_extension:escape.util.pox_extension.ExtendedOFConnectionArbiter._core_name}\pysigline{\bfcode{\_core\_name}\strong{ = `OpenFlowConnectionArbiter'}}
\end{fulllineitems}

\index{\_\_init\_\_() (escape.util.pox\_extension.ExtendedOFConnectionArbiter method)}

\begin{fulllineitems}
\phantomsection\label{util/pox_extension:escape.util.pox_extension.ExtendedOFConnectionArbiter.__init__}\pysiglinewithargsret{\bfcode{\_\_init\_\_}}{\emph{default=False}}{}
Init
\begin{quote}\begin{description}
\item[{Parameters}] \leavevmode
\textbf{\texttt{default}} (\code{OpenFlowNexus}) -- inherited param

\end{description}\end{quote}

\end{fulllineitems}

\index{add\_connection\_listener() (escape.util.pox\_extension.ExtendedOFConnectionArbiter method)}

\begin{fulllineitems}
\phantomsection\label{util/pox_extension:escape.util.pox_extension.ExtendedOFConnectionArbiter.add_connection_listener}\pysiglinewithargsret{\bfcode{add\_connection\_listener}}{\emph{address}, \emph{nexus}}{}
Helper function to register connection listeners a.k.a.
\code{OpenFlowNexus}
\begin{quote}\begin{description}
\item[{Parameters}] \leavevmode\begin{itemize}
\item {} 
\textbf{\texttt{address}} (\href{https://docs.python.org/2.7/library/functions.html\#tuple}{\emph{tuple}}) -- listened socket name in form of (address, port)

\item {} 
\textbf{\texttt{nexus}} ({\hyperref[util/pox_extension:escape.util.pox_extension.OpenFlowBridge]{\emph{\code{OpenFlowBridge}}}}) -- registered object

\end{itemize}

\item[{Returns}] \leavevmode
registered listener

\item[{Return type}] \leavevmode
{\hyperref[util/pox_extension:escape.util.pox_extension.OpenFlowBridge]{\emph{\code{OpenFlowBridge}}}}

\end{description}\end{quote}

\end{fulllineitems}

\index{activate() (escape.util.pox\_extension.ExtendedOFConnectionArbiter class method)}

\begin{fulllineitems}
\phantomsection\label{util/pox_extension:escape.util.pox_extension.ExtendedOFConnectionArbiter.activate}\pysiglinewithargsret{\strong{classmethod }\bfcode{activate}}{}{}
Register this component into \code{pox.core} and replace already registered
Arbiter.
\begin{quote}\begin{description}
\item[{Returns}] \leavevmode
registered component

\item[{Return type}] \leavevmode
{\hyperref[util/pox_extension:escape.util.pox_extension.ExtendedOFConnectionArbiter]{\emph{\code{ExtendedOFConnectionArbiter}}}}

\end{description}\end{quote}

\end{fulllineitems}

\index{getNexus() (escape.util.pox\_extension.ExtendedOFConnectionArbiter method)}

\begin{fulllineitems}
\phantomsection\label{util/pox_extension:escape.util.pox_extension.ExtendedOFConnectionArbiter.getNexus}\pysiglinewithargsret{\bfcode{getNexus}}{\emph{connection}}{}
Return registered connection listener or default \code{core.openflow}.

Fires ConnectionIn event.
\begin{quote}\begin{description}
\item[{Parameters}] \leavevmode
\textbf{\texttt{connection}} (\code{Connection}) -- incoming connection object

\item[{Returns}] \leavevmode
OpenFlow event originator object

\item[{Return type}] \leavevmode
\code{OpenFlowNexus}

\end{description}\end{quote}

\end{fulllineitems}


\end{fulllineitems}



\section{Main modules for layers/sublayers}
\label{index:main-modules-for-layers-sublayers}

\subsection{The \emph{unify.py} top module}
\label{unify:module-unify}\label{unify:the-unify-py-top-module}\label{unify::doc}\index{unify (module)}
Basic POX module for ESCAPE

Initiate appropriate APIs

Follows POX module conventions
\index{\_start\_components() (in module unify)}

\begin{fulllineitems}
\phantomsection\label{unify:unify._start_components}\pysiglinewithargsret{\code{unify.}\bfcode{\_start\_components}}{\emph{event}}{}
Initiate and run POX with ESCAPE components
\begin{quote}\begin{description}
\item[{Parameters}] \leavevmode
\textbf{\texttt{event}} (\emph{GoingUpEvent}) -- POX's going up event

\item[{Returns}] \leavevmode
None

\end{description}\end{quote}

\end{fulllineitems}

\index{add\_dependencies() (in module unify)}

\begin{fulllineitems}
\phantomsection\label{unify:unify.add_dependencies}\pysiglinewithargsret{\code{unify.}\bfcode{add\_dependencies}}{}{}
Add dependency directories to PYTHONPATH.
Dependencies are directories besides the escape.py initial script except pox.
\begin{quote}\begin{description}
\item[{Returns}] \leavevmode
None

\end{description}\end{quote}

\end{fulllineitems}

\index{launch() (in module unify)}

\begin{fulllineitems}
\phantomsection\label{unify:unify.launch}\pysiglinewithargsret{\code{unify.}\bfcode{launch}}{\emph{sg\_file='`}, \emph{gui=False}, \emph{full=False}, \emph{debug=True}}{}
Launch function called by POX core when core is up
\begin{quote}\begin{description}
\item[{Parameters}] \leavevmode\begin{itemize}
\item {} 
\textbf{\texttt{sg\_file}} (\href{https://docs.python.org/2.7/library/functions.html\#str}{\emph{str}}) -- Path of the input Service graph (optional)

\item {} 
\textbf{\texttt{gui}} (\href{https://docs.python.org/2.7/library/functions.html\#bool}{\emph{bool}}) -- Signal for initiate GUI (optional)

\item {} 
\textbf{\texttt{full}} (\href{https://docs.python.org/2.7/library/functions.html\#bool}{\emph{bool}}) -- Initiate Infrastructure Layer also

\end{itemize}

\item[{Returns}] \leavevmode
None

\end{description}\end{quote}

\end{fulllineitems}



\subsubsection{Submodules}
\label{unify:submodules}

\paragraph{The \emph{service.py} main module}
\label{service:module-service}\label{service:the-service-py-main-module}\label{service::doc}\index{service (module)}
Basic POX module for ESCAPE Service (Graph Adaptation) sublayer

Initiate appropriate API class which implements U-Sl reference point

Follows POX module conventions
\index{\_start\_layer() (in module service)}

\begin{fulllineitems}
\phantomsection\label{service:service._start_layer}\pysiglinewithargsret{\code{service.}\bfcode{\_start\_layer}}{\emph{event}}{}
Initiate and run Service module
\begin{quote}\begin{description}
\item[{Parameters}] \leavevmode
\textbf{\texttt{event}} (\emph{GoingUpEvent}) -- POX's going up event

\item[{Returns}] \leavevmode
None

\end{description}\end{quote}

\end{fulllineitems}

\index{launch() (in module service)}

\begin{fulllineitems}
\phantomsection\label{service:service.launch}\pysiglinewithargsret{\code{service.}\bfcode{launch}}{\emph{sg\_file='`}, \emph{gui=False}, \emph{standalone=False}}{}
Launch function called by POX core when core is up
\begin{quote}\begin{description}
\item[{Parameters}] \leavevmode\begin{itemize}
\item {} 
\textbf{\texttt{sg\_file}} (\href{https://docs.python.org/2.7/library/functions.html\#str}{\emph{str}}) -- Path of the input Service graph (optional)

\item {} 
\textbf{\texttt{gui}} (\href{https://docs.python.org/2.7/library/functions.html\#bool}{\emph{bool}}) -- Initiate built-in GUI (optional)

\item {} 
\textbf{\texttt{standalone}} (\href{https://docs.python.org/2.7/library/functions.html\#bool}{\emph{bool}}) -- Run layer without dependency checking (optional)

\end{itemize}

\item[{Returns}] \leavevmode
None

\end{description}\end{quote}

\end{fulllineitems}



\paragraph{The \emph{orchestration.py} main module}
\label{orchestration:module-orchestration}\label{orchestration:the-orchestration-py-main-module}\label{orchestration::doc}\index{orchestration (module)}
Basic POX module for ESCAPE Resource Orchestration Sublayer (ROS)

Initiate appropriate API class which implements Sl-Or reference point

Follows POX module conventions
\index{\_start\_layer() (in module orchestration)}

\begin{fulllineitems}
\phantomsection\label{orchestration:orchestration._start_layer}\pysiglinewithargsret{\code{orchestration.}\bfcode{\_start\_layer}}{\emph{event}}{}
Initiate and run Orchestration module
\begin{quote}\begin{description}
\item[{Parameters}] \leavevmode
\textbf{\texttt{event}} (\emph{GoingUpEvent}) -- POX's going up event

\item[{Returns}] \leavevmode
None

\end{description}\end{quote}

\end{fulllineitems}

\index{launch() (in module orchestration)}

\begin{fulllineitems}
\phantomsection\label{orchestration:orchestration.launch}\pysiglinewithargsret{\code{orchestration.}\bfcode{launch}}{\emph{nffg\_file='`}, \emph{standalone=False}}{}
Launch function called by POX core when core is up
\begin{quote}\begin{description}
\item[{Parameters}] \leavevmode\begin{itemize}
\item {} 
\textbf{\texttt{nffg\_file}} (\href{https://docs.python.org/2.7/library/functions.html\#str}{\emph{str}}) -- Path of the NF-FG graph (optional)

\item {} 
\textbf{\texttt{standalone}} (\href{https://docs.python.org/2.7/library/functions.html\#bool}{\emph{bool}}) -- Run layer without dependency checking (optional)

\end{itemize}

\item[{Returns}] \leavevmode
None

\end{description}\end{quote}

\end{fulllineitems}



\paragraph{The \emph{adaptation.py} main module}
\label{adaptation:the-adaptation-py-main-module}\label{adaptation:module-adaptation}\label{adaptation::doc}\index{adaptation (module)}
Basic POX module for ESCAPE Controller Adaptation Sublayer (CAS)

Initiate appropriate API class which implements Or-Ca reference point

Follows POX module conventions
\index{\_start\_layer() (in module adaptation)}

\begin{fulllineitems}
\phantomsection\label{adaptation:adaptation._start_layer}\pysiglinewithargsret{\code{adaptation.}\bfcode{\_start\_layer}}{\emph{event}}{}
Initiate and run Adaptation module
\begin{quote}\begin{description}
\item[{Parameters}] \leavevmode
\textbf{\texttt{event}} (\emph{GoingUpEvent}) -- POX's going up event

\item[{Returns}] \leavevmode
None

\end{description}\end{quote}

\end{fulllineitems}

\index{launch() (in module adaptation)}

\begin{fulllineitems}
\phantomsection\label{adaptation:adaptation.launch}\pysiglinewithargsret{\code{adaptation.}\bfcode{launch}}{\emph{mapped\_nffg\_file='`}, \emph{with\_infr=False}, \emph{standalone=False}}{}
Launch function called by POX core when core is up
\begin{quote}\begin{description}
\item[{Parameters}] \leavevmode\begin{itemize}
\item {} 
\textbf{\texttt{mapped\_nffg\_file}} (\href{https://docs.python.org/2.7/library/functions.html\#str}{\emph{str}}) -- Path of the mapped NF-FG graph (optional)

\item {} 
\textbf{\texttt{with\_infr}} (\href{https://docs.python.org/2.7/library/functions.html\#bool}{\emph{bool}}) -- Set Infrastructure as a dependency

\item {} 
\textbf{\texttt{standalone}} (\href{https://docs.python.org/2.7/library/functions.html\#bool}{\emph{bool}}) -- Run layer without dependency checking (optional)

\end{itemize}

\item[{Returns}] \leavevmode
None

\end{description}\end{quote}

\end{fulllineitems}



\paragraph{The \emph{infrastructure.py} main module}
\label{infrastructure:module-infrastructure}\label{infrastructure:the-infrastructure-py-main-module}\label{infrastructure::doc}\index{infrastructure (module)}
Basic POX module for ESCAPE Infrastructure Layer

Initiate appropriate API class which emulate Co-Rm reference point

Follows POX module conventions
\index{\_start\_layer() (in module infrastructure)}

\begin{fulllineitems}
\phantomsection\label{infrastructure:infrastructure._start_layer}\pysiglinewithargsret{\code{infrastructure.}\bfcode{\_start\_layer}}{\emph{event}}{}
Initiate and run Infrastructure module
\begin{quote}\begin{description}
\item[{Parameters}] \leavevmode
\textbf{\texttt{event}} (\emph{GoingUpEvent}) -- POX's going up event

\item[{Returns}] \leavevmode
None

\end{description}\end{quote}

\end{fulllineitems}

\index{launch() (in module infrastructure)}

\begin{fulllineitems}
\phantomsection\label{infrastructure:infrastructure.launch}\pysiglinewithargsret{\code{infrastructure.}\bfcode{launch}}{\emph{standalone=False}}{}
Launch function called by POX core when core is up
\begin{quote}\begin{description}
\item[{Parameters}] \leavevmode
\textbf{\texttt{standalone}} (\href{https://docs.python.org/2.7/library/functions.html\#bool}{\emph{bool}}) -- Run layer without dependency checking (optional)

\item[{Returns}] \leavevmode
None

\end{description}\end{quote}

\end{fulllineitems}



\section{README}
\label{index:readme}
ESCAPEv2 example commands

\textbf{The simpliest use-case:}

\begin{Verbatim}[commandchars=\\\{\}]
\PYG{n+nv}{\PYGZdl{} }./escape.py
\end{Verbatim}

Usage:

\begin{Verbatim}[commandchars=\\\{\}]
\PYGZdl{} ./escape.py \PYGZhy{}h
usage: escape.py [\PYGZhy{}h] [\PYGZhy{}v] [\PYGZhy{}d] [\PYGZhy{}f] [\PYGZhy{}i]

ESCAPE: Extensible Service ChAin Prototyping Environment using Mininet, Click,
NETCONF and POX

optional arguments:
  \PYGZhy{}h, \PYGZhy{}\PYGZhy{}help         show this help message and exit
  \PYGZhy{}v, \PYGZhy{}\PYGZhy{}version      show program\PYGZsq{}s version number and exit

ESCAPE arguments:
  \PYGZhy{}d, \PYGZhy{}\PYGZhy{}debug        run the ESCAPE in debug mode
  \PYGZhy{}f, \PYGZhy{}\PYGZhy{}full         run the infrastructure layer also
  \PYGZhy{}i, \PYGZhy{}\PYGZhy{}interactive  run an interactive shell for observing internal states
\end{Verbatim}

\textbf{More advanced commands:}

Basic command:

\begin{Verbatim}[commandchars=\\\{\}]
\PYG{n+nv}{\PYGZdl{} }./pox.py unify
\end{Verbatim}

Basic command for debugging:

\begin{Verbatim}[commandchars=\\\{\}]
\PYG{n+nv}{\PYGZdl{} }./pox.py \PYGZhy{}\PYGZhy{}verbose \PYGZhy{}\PYGZhy{}no\PYGZhy{}openflow unify py
\end{Verbatim}

Basic command to initiate a built-in emulated network for testing:

\begin{Verbatim}[commandchars=\\\{\}]
\PYG{c}{\PYGZsh{} Infrastructure layer requires root privileges due to use of Mininet!}
\PYG{n+nv}{\PYGZdl{} }sudo ./pox.py unify \PYGZhy{}\PYGZhy{}full
\end{Verbatim}

Minimal command with explicitly-defined components (components' order is irrelevant):

\begin{Verbatim}[commandchars=\\\{\}]
\PYG{n+nv}{\PYGZdl{} }./pox.py service orchestration adaptation
\end{Verbatim}

Without service layer:

\begin{Verbatim}[commandchars=\\\{\}]
\PYG{n+nv}{\PYGZdl{} }./pox.py orchestration adaptation
\end{Verbatim}

With infrastructure layer:

\begin{Verbatim}[commandchars=\\\{\}]
\PYG{n+nv}{\PYGZdl{} }sudo ./pox.py service orchestration adaptation \PYGZhy{}\PYGZhy{}with\PYGZus{}infr infrastructure
\end{Verbatim}

Long version with debugging and explicitly-defined components (analogous with ./pox.py unify --full):

\begin{Verbatim}[commandchars=\\\{\}]
\PYG{n+nv}{\PYGZdl{}.}/pox.py \PYGZhy{}\PYGZhy{}verbose log.level \PYGZhy{}\PYGZhy{}DEBUG samples.pretty\PYGZus{}log service orchestration adaptation\PYGZhy{}\PYGZhy{}with\PYGZus{}infr infrastructure
\end{Verbatim}

Start layers with graph-represented input contained in a specific file:

\begin{Verbatim}[commandchars=\\\{\}]
\PYG{n+nv}{\PYGZdl{} }./pox.py service \PYGZhy{}\PYGZhy{}sg\PYGZus{}file\PYG{o}{=}\PYGZlt{}path\PYGZgt{} ...
\PYG{n+nv}{\PYGZdl{} }./pox.py unify \PYGZhy{}\PYGZhy{}sg\PYGZus{}file\PYG{o}{=}\PYGZlt{}path\PYGZgt{}

\PYG{n+nv}{\PYGZdl{} }./pox.py orchestration \PYGZhy{}\PYGZhy{}nffg\PYGZus{}file\PYG{o}{=}\PYGZlt{}path\PYGZgt{} ...
\PYG{n+nv}{\PYGZdl{} }./pox.py adaptation \PYGZhy{}\PYGZhy{}mapped\PYGZus{}nffg\PYGZus{}file\PYG{o}{=}\PYGZlt{}path\PYGZgt{} ...
\end{Verbatim}

Start ESCAPEv2 with built-in GUI:

\begin{Verbatim}[commandchars=\\\{\}]
\PYG{n+nv}{\PYGZdl{} }./pox.py service \PYGZhy{}\PYGZhy{}gui ...
\PYG{n+nv}{\PYGZdl{} }./pox.py unify \PYGZhy{}\PYGZhy{}gui
\end{Verbatim}

Start layer in standalone mode (no dependency handling) for test/debug:

\begin{Verbatim}[commandchars=\\\{\}]
\PYG{n+nv}{\PYGZdl{} }./pox.py service \PYGZhy{}\PYGZhy{}standalone
\PYG{n+nv}{\PYGZdl{} }./pox.py orchestration \PYGZhy{}\PYGZhy{}standalone
\PYG{n+nv}{\PYGZdl{} }./pox.py adaptation \PYGZhy{}\PYGZhy{}standalone
\PYG{n+nv}{\PYGZdl{} }sudo ./pox.py infrastructure \PYGZhy{}\PYGZhy{}standalone

\PYG{n+nv}{\PYGZdl{} }./pox.py service orchestration \PYGZhy{}\PYGZhy{}standalone
\end{Verbatim}


\section{REST API}
\label{index:rest-api}
\emph{Content Negotiation:} The Service layer's RESTful API accepts and returns data only in JSON format.

\emph{Operations:}   Every operation need to be called under the \textbf{escape/} path. E.g. \emph{http://localhost/escape/version}

\begin{tabulary}{\linewidth}{|L|L|L|L|}
\hline
\textsf{\relax 
Path
} & \textsf{\relax 
Params
} & \textsf{\relax 
HTTP verbs
} & \textsf{\relax 
Description
}\\
\hline
\emph{/version}
 & 
\code{None}
 & 
GET
 & 
Returns with the current version of ESCAPEv2
\\
\hline
\emph{/echo}
 & 
\code{ANY}
 & 
ALL
 & 
Returns with the given parameters
\\
\hline
\emph{/operations}
 & 
\code{None}
 & 
GET
 & 
Returns with the implemented operations as a list
\\
\hline
\emph{/sg}
 & 
\code{NFFG}
 & 
POST
 & 
Initiate given NFFG. Returns the given NFFG initiation is accepted or not.
\\
\hline\end{tabulary}



\chapter{Indices and tables}
\label{index:indices-and-tables}\begin{itemize}
\item {} 
\DUspan{xref,std,std-ref}{genindex}

\item {} 
\DUspan{xref,std,std-ref}{modindex}

\item {} 
\DUspan{xref,std,std-ref}{search}

\end{itemize}


\renewcommand{\indexname}{Python Module Index}
\begin{theindex}
\def\bigletter#1{{\Large\sffamily#1}\nopagebreak\vspace{1mm}}
\bigletter{a}
\item {\texttt{adaptation}}, \pageref{adaptation:module-adaptation}
\indexspace
\bigletter{e}
\item {\texttt{escape}}, \pageref{escape:module-escape}
\item {\texttt{escape.adapt}}, \pageref{adapt/adapt:module-escape.adapt}
\item {\texttt{escape.adapt.adaptation}}, \pageref{adapt/adaptation:module-escape.adapt.adaptation}
\item {\texttt{escape.adapt.cas\_API}}, \pageref{adapt/cas_API:module-escape.adapt.cas_API}
\item {\texttt{escape.adapt.domain\_adapters}}, \pageref{adapt/domain_adapters:module-escape.adapt.domain_adapters}
\item {\texttt{escape.infr}}, \pageref{infr/infr:module-escape.infr}
\item {\texttt{escape.infr.il\_API}}, \pageref{infr/il_API:module-escape.infr.il_API}
\item {\texttt{escape.infr.topology}}, \pageref{infr/topology:module-escape.infr.topology}
\item {\texttt{escape.orchest}}, \pageref{orchest/orchest:module-escape.orchest}
\item {\texttt{escape.orchest.policy\_enforcement}}, \pageref{orchest/policy_enforcement:module-escape.orchest.policy_enforcement}
\item {\texttt{escape.orchest.ros\_API}}, \pageref{orchest/ros_API:module-escape.orchest.ros_API}
\item {\texttt{escape.orchest.ros\_mapping}}, \pageref{orchest/ros_mapping:module-escape.orchest.ros_mapping}
\item {\texttt{escape.orchest.ros\_orchestration}}, \pageref{orchest/ros_orchestration:module-escape.orchest.ros_orchestration}
\item {\texttt{escape.orchest.virtualization\_mgmt}}, \pageref{orchest/virtualization_mgmt:module-escape.orchest.virtualization_mgmt}
\item {\texttt{escape.service}}, \pageref{service/service:module-escape.service}
\item {\texttt{escape.service.element\_mgmt}}, \pageref{service/element_mgmt:module-escape.service.element_mgmt}
\item {\texttt{escape.service.sas\_API}}, \pageref{service/sas_API:module-escape.service.sas_API}
\item {\texttt{escape.service.sas\_mapping}}, \pageref{service/sas_mapping:module-escape.service.sas_mapping}
\item {\texttt{escape.service.sas\_orchestration}}, \pageref{service/sas_orchestration:module-escape.service.sas_orchestration}
\item {\texttt{escape.util}}, \pageref{util/util:module-escape.util}
\item {\texttt{escape.util.adapter}}, \pageref{util/adapter:module-escape.util.adapter}
\item {\texttt{escape.util.api}}, \pageref{util/api:module-escape.util.api}
\item {\texttt{escape.util.mapping}}, \pageref{util/mapping:module-escape.util.mapping}
\item {\texttt{escape.util.misc}}, \pageref{util/misc:module-escape.util.misc}
\item {\texttt{escape.util.netconf}}, \pageref{util/netconf:module-escape.util.netconf}
\item {\texttt{escape.util.nffg}}, \pageref{util/nffg:module-escape.util.nffg}
\item {\texttt{escape.util.pox\_extension}}, \pageref{util/pox_extension:module-escape.util.pox_extension}
\indexspace
\bigletter{i}
\item {\texttt{infrastructure}}, \pageref{infrastructure:module-infrastructure}
\indexspace
\bigletter{o}
\item {\texttt{orchestration}}, \pageref{orchestration:module-orchestration}
\indexspace
\bigletter{s}
\item {\texttt{service}}, \pageref{service:module-service}
\indexspace
\bigletter{u}
\item {\texttt{unify}}, \pageref{unify:module-unify}
\end{theindex}

\renewcommand{\indexname}{Index}
\printindex
\end{document}
